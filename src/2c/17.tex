\documentclass[a4paper]{article}
\usepackage[leqno]{amsmath}
\usepackage[top=1cm,right=1cm,bottom=1cm,left=1cm]{geometry}
\setlength\parindent{0pt}
\begin{document}
\Large
Consider this counter-example.
\begin{align*}
    U_1 &= \left\{ (a, b, 0) \in \mathbf{F^3} : a, b \in \mathbf{F} \right\} \\
    U_2 &= \left\{ (0, c, c) \in \mathbf{F^3} : c \in \mathbf{F} \right\} \\
    U_3 &= \left\{ (0, 0, d) \in \mathbf{F^3} : d \in \mathbf{F} \right\} 
\end{align*}
The list $(1,0,0), (0,1,0)$ is a basis of $U_1$, so dim $U_1 = 2$.
The list $(0,1,1)$ is a basis of $U_2$, so dim $U_2 = 1$.
The list $(0,0,1)$ is a basis of $U_3$, so dim $U_3 = 1$.
\\
\\
If $v \in (U_1\cap U_2)$ then $v$ has 0 as its first coordinate (to be in $U_2$), 0 as its third coordinate (to be in $U_1$) and its second coordinate must be the same as its third coordinate (to be in $U_2$).
Hence $v = (0, 0, 0)$.
So dim $(U_1\cap U_2) = 0$.
\\
\\
Using similar logic, it is clear that $(0, 0, 0)$ is also the only element in $U_1\cap U_3$, the only element in $U_2\cap U_3$, and thus the only element in $U_1\cap U_2\cap U_3$.
Hence all of the intersection subspaces are 0-dimensional.
\begin{align*}
 \text{RHS} = &\text{ dim }U_1 + \text{dim }U_2 + \text{dim }U_3 \\
        &- \text{dim }(U_1 \cap U_2) - \text{dim }(U_1 \cap U_3) - \text{dim }(U_2 \cap U_3) \\
        &+ \text{dim }(U_1 \cap U_2 \cap U_3)
\\
   = &\text{ }2 + 1 + 1 - 0 - 0 - 0 + 0 \\
   = &\text{ }4
\end{align*}
But $\mathbf{F^3}$ is closed under addition, so every possible sum of three elements of $\mathbf{F^3}$ is in $\mathbf{F^3}$.
Hence $U_1 + U_2 + U_3$ is a subspace of $\mathbf{F^3}$.
Thus,
\begin{align*}
    \text{dim }(U_1 + U_2 + U_3) &\le \text{dim }\mathbf{F^3} \\
    \text{LHS } &\le 3
\end{align*}
Therefore the statement is false.
\end{document}
