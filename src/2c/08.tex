\documentclass[a4paper]{article}
\usepackage[leqno]{amsmath}
\usepackage[top=1cm,right=1cm,bottom=1cm,left=1cm]{geometry}

%\usepackage{xcolor}
%\pagecolor[rgb]{0.1,0.1,0.1}
%\color[rgb]{0.5,0.86,0.86}

\begin{document}
\Large
\begin{align*}
\tag{a}
U &= \left\{ p \in \mathcal{P}_4(\textbf{F}) : \int_{-1}^1 p = 0 \right\} \\
\end{align*}
\begin{align*}
p(z)   &= a_0 + a_1z + a_2z^2 + a_3z^3 + a_4z^4 \\
\int p &= a_0z + \frac{a_1}{2}z^2 + \frac{a_2}{3}z^3 + \frac{a_3}{4}z^4 + \frac{a_4}{5}z^5 + c \\ 
\int_{-1}^1 p &= 
          \left( a_0 + \frac{a_1}{2} + \frac{a_2}{3} + \frac{a_3}{4} + \frac{a_4}{5}  \right) - 
          \left( -a_0 + \frac{a_1}{2} - \frac{a_2}{3} + \frac{a_3}{4} - \frac{a_4}{5}  \right)    \\
\int_{-1}^1 p &= a_0 + \frac{a_1}{2} + \frac{a_2}{3} + \frac{a_3}{4} + \frac{a_4}{5}  + a_0 - \frac{a_1}{2} + \frac{a_2}{3} - \frac{a_3}{4} + \frac{a_4}{5}     
\\
\int_{-1}^1 p &= 2a_0 + \frac{2a_2}{3} + \frac{2a_4}{5} =0
\\
2a_0 &= \frac{-2a_2}{3} - \frac{2a_4}{5} \\
a_0 &= \frac{-a_2}{3} - \frac{a_4}{5} \\
\\
p(z)   &= \frac{-a_2}{3} - \frac{a_4}{5} + a_1z + a_2z^2 + a_3z^3 + a_4z^4 \\
p(z)   &= a_1z + a_2\left(z^2-\frac{1}{3}\right) + a_3z^3 + a_4\left(z^4-\frac{1}{5}\right)\\
\end{align*}
The list $z$, $z^2-\frac{1}{3}$, $z^3$, $z^4-\frac{1}{5}$ is a basis of $U$.
\\
\\
(b) The polynomial $p(z)=1$ is not in $U$, since $\int_{-1}^1p=2\neq 0.$ Hence adjoining $1$ to the basis of $U$ produces a linearly independent list in $\mathcal{P}_4(\textbf{F})$. The resulting list is also the right length and thus a basis of $\mathcal{P}_4(\textbf{F})$.
\\
\begin{align*}
\tag{c}
\mathcal{P}_4(\textbf{F}) &= U \oplus W \\
W &= \text{span}(1) \\
 W &= \left\{ p \in \mathcal{P}_4(\textbf{F}) : p(z) = a : a \in \textbf{F} \right\}
\end{align*}
\end{document}
 
