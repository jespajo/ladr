\documentclass{article}
\usepackage[leqno]{amsmath}
\usepackage[top=0.1cm,right=1cm,bottom=0.1cm,left=1cm]{geometry}
\begin{document}
\Large
\begin{align*}
\tag{a}
U &= \{p \in \mathcal{P}_4(\textbf{F}) : p''(6) = 0\} \\
\\
      p(z) &= a_0 + a_1z + a_2z^2 + a_3z^3 + a_4z^4 \\
     p'(z) &= a_1 + 2a_2z + 3a_3z^2 + 4a_4z^3 \\
    p''(z) &= 2a_2 + 6a_3z + 12a_4z^2 \\
\\
    p''(6) &= 2a_2 + 6a_3\times 6 + 12a_4\times 6^2 = 0 \\
            0 &= 2a_2 + 36a_3 + 432a_4 \\
         2a_2 &= -36a_3 - 432a_4 \\
          a_2 &= -18a_3 - 216a_4 \\
\\
      p(z) &= a_0 + a_1z + (-18a_3 - 216a_4)z^2 + a_3z^3 + a_4z^4 \\
      p(z) &= a_0 + a_1z - 18a_3z^2 + a_3z^3 - 216a_4z^2 + a_4z^4 \\
      p(z) &= a_0 + a_1z + a_3(z^3 - 18z^2) + a_4(z^4 - 216z^2) \\
\end{align*}
The four polynomials $1$, $z$, $z^3 - 18z^2$, $z^4 - 216z^2$ span $U$ because each $p \in U$ can be written as a linear combination of this list.
These polynomials are linearly independent since any choice of $a_0$, $a_1$, $a_2$ and $a_3$ other than all zeroes sums to a polynomial with at least one non-zero coefficient.
Hence the list is a basis of $U$.
\\
\\
(b) $z^2 \notin U$ because taking $a_3$ or $a_4$ as non-zero would result in a polynomial with non-zero coefficients for $z^3$ or $z^2$.
Hence the list $1$, $z$, $z^3 - 18z^2$, $z^4 - 216z^2$, $z^2$ (the basis of $U$ then $z^2$) is linearly independent in $\mathcal{P}_4(\textbf{F})$.
This list has length five, which is the dimension of $\mathcal{P}_4(\textbf{F})$. Hence it is a basis of $\mathcal{P}_4(\textbf{F})$.
\begin{align*}
\tag{c}
\mathcal{P}_4(\textbf{F}) &= U \oplus W \\
W &= \{ p(z) = az^2 : a \in \textbf{F} \} \\
\end{align*}
\end{document}
