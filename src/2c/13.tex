\documentclass[a4paper]{article}
\usepackage[leqno]{amsmath}
\usepackage[top=1cm,right=1cm,bottom=1cm,left=1cm]{geometry}
\setlength\parindent{0pt}
\begin{document}
\Large
\begin{align*}
    \text{dim }(U + W) &= \text{dim }U + \text{dim }W - \text{dim }(U \cap W) \\
    &= 4 + 4 - \text{dim }(U \cap W) \\
    &= 8 - \text{dim }(U \cap W)
\intertext{
Both $U$ and $W$ are subspaces of $\mathbf{C^6}$, so $\mathbf{C^6}$ contains every $u \in U$, every $w \in W$ and also every $u + w$ because it is closed under addition.
Hence $U + W$ is a subspace of $\mathbf{C^6}$.
Thus
}
    \text{dim }\mathbf{C^6} &\ge \text{dim }(U + W) \\
    6 & \ge 8 - \text{dim }(U \cap W) \\
    \text{dim }(U \cap W) &\ge 2
\end{align*}
Therefore there is a basis of $U \cap W$ of length at least two.
The first two vectors in this basis are linearly independent, which means that neither is a scalar multiple of the other.
\end{document}
