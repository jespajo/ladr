\documentclass[a5paper]{article}
\usepackage{amsmath}
\usepackage{amssymb}
\usepackage[top=1cm,right=1cm,bottom=2cm,left=1cm]{geometry}
\setlength\parindent{0pt}
\setlength\parskip{1em}
\begin{document}
\newcommand   \C            {\mathbf{C}}
\newcommand   \R            {\mathbf{R}}
\renewcommand \L            {\mathcal{L}}
\newcommand   \F            {\mathbf{F}}
\renewcommand \P            {\mathcal{P}}
\newcommand   \M            {\mathcal{M}}
\newcommand   \E            {\mathcal{E}}
\newcommand   \op           {\operatorname}
\newcommand   \A            {\mathcal{A}}
\newcommand   \Q            {\mathcal{Q}}
\newcommand   \conj[1]      {\overline{#1}}

    4.4.
    Suppose $m$ is a positive integer.
    Is the set
\begin{align*}
        \{ 0 \} \cup \{ p \in \P(\F) : \op{deg}p = m \}
\end{align*}
    a subspace of $\P(\F)$?

    No, becuase the set is not closed under addition.
    For example, suppose $m = 2$.
    Then define $p,q \in \P(\F)$ by
\begin{align*}
        p(x) = x^2 + 1 \quad\text{and}\quad q(x) = -x^2 + 1
\end{align*}
    for all $x \in \F$.
    Both $p$ and $q$ are elements of the set in question because $\op{deg}p = \op{deg}q = 2$.
    But
\begin{align*}
        p(x) + q(x) &= x^2 + 1 + (-x^2 + 1) \\
                    &= 2 .
\end{align*}
    Thus $p+q$ is not in the set because $\op{deg}(p+q)=0$.
    So the set is not a subspace of $\P(\F)$ because it is not closed under addition.
\end{document}
