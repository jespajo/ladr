\documentclass[a5paper]{article}
\usepackage{amsmath}
\usepackage[top=1cm,right=1cm,bottom=1cm,left=1cm]{geometry}
\setlength\parindent{0pt}
%
%\usepackage{xcolor}
%\pagecolor[rgb]{0.1,0.1,0.1}
%\color[rgb]{1.0,1.0,1.0}
%
\begin{document}

\newcommand   \F { \mathbf{F}  }
\renewcommand \L { \mathcal{L} }

Suppose $S,T,T_1,T_2,T_3 \in \L(V,W)$ and $v\in V$ and $a,b \in \F$. \\

Arithmetic on $\L(V,W)$ is commutative and associative because arithmetic on $W$ is commutative and associative:
\begin{align*}
             (S + T)(v) &= Sv + Tv \\
                        &= Tv + Sv \\
                        &= (T + S)(v) \\
  \\
    \big( (T_1 + T_2) + T_3 \big)(v) &= (T_1 + T_2)(v) + T_3v     \\
                                     &= T_1v + T_2v + T_3v      \\
                                     &= T_1v + (T_2 + T_3)(v)   \\
                                     &= \big( T_1 + (T_2 + T_3) \big)(v) \\
  \\
           \big( (ab)T \big)(v) &= ab(Tv)       \\
                                &= a(bTv)       \\
                                &= \big( a(bT) \big)(v)
\end{align*}

The additive identity in $\L(V,W)$ is the zero map, which takes $v$ to $0\in W$:
\begin{align*}
    Tv + 0(v) = Tv + 0 = Tv
\end{align*}

The additive inverse of $T$ is the function $-T : (-T)(v) = -1\times Tv$.
$-Tv$ is in $W$ because $Tv$ is in $W$ and $W$ is closed under scalar multiplication.
$-T$ added to $T$ produces the zero map:
\begin{align*}
    Tv + (-T)(v) = Tv + T(-v) = T(v - v) = T(0) = 0 = 0(v)
\end{align*}

The multiplicative identity holds for $\L(V,W)$:
\begin{align*}
    1Tv = T(1v) = Tv
\end{align*}

Finally, arithmetic on $\L(V,W)$ has distributive properties:
\begin{align*}
    a(Sv + Tv) &= aSv + aTv            \\
  \\
     (a + b)Tv &= aTv + bTv
\end{align*}
Thus $\L(V,W)$ is a vector space.
\end{document}
