\documentclass[a5paper]{article}
\usepackage{amsmath}
\usepackage{amssymb}
\usepackage[top=1cm,right=1cm,bottom=2cm,left=1cm]{geometry}
\setlength\parindent{0pt}
\setlength\parskip{1em}
%
\begin{document}
\newcommand   \C           {\mathbf{C}}
\newcommand   \R           {\mathbf{R}}
\renewcommand \L           {\mathcal{L}}
\newcommand   \F           {\mathbf{F}}
\renewcommand \P           {\mathcal{P}}
\newcommand   \M           {\mathcal{M}}
\newcommand   \E           {\mathcal{E}}
\newcommand   \op          {\operatorname}
\newcommand   \A           {\mathcal{A}}

    3.A.11.
    Suppose $V$ is finite-dimensional and $T \in \L(V)$.
    Prove that $T$ is a scalar multiple of the identity if and only if $ST = TS$ for every $S \in \L(V)$.

    Suppose $T$ is a scalar multiple of the identity.
    Then $a \in \F$ such that $T = aI$.

    Thus for $v \in V$,
\begin{align*}
        STv = S(Tv) = S(aIv) = S(av) = aSv = aI(Sv) = T(Sv) = TSv .
\end{align*}
    This completes the proof in one direction.

    To prove in the other direction, suppose $ST=TS$ for every $S \in \L(V)$.

    Suppose $v_1,\dots,v_n$ is a basis of $V$.

    $T \in \L(V)$ is defined by some $w_1,\dots,w_n \in V$ such that $Tv_k=w_k$ for $k=1,\dots,n$.
    Let each $w_k$ be defined by $A_{1,k},\dots,A_{n,k} \in \F$ such that
\begin{align*}
        Tv_k = w_k = A_{1,k}v_1 + \dots A_{n,k}v_n .
\end{align*}
    For convenience, let the subscripts of the $v$'s and the $A$'s wrap around at $n$.
    In other words, let $v_{n+1}=v_1$ and let $A_{j,n+1}=A_{j,1}$ and let $A_{n+1,k}=A_{1,k}$.

    Define $S_1,\dots,S_n \in \L(V)$ such that each $S_j$ takes $v_j$ to $v_{j+1}$ and takes all the other $v$'s in the basis of $V$ to $0$.
    In other words, for each $j,k \in \{1,\dots,n\}$,
\begin{align*}
        S_jv_k &=
        \begin{cases}
            \ v_{k+1}   &\text{for}\ j  =   k \\
            \ 0         &\text{for}\ j \neq k .
        \end{cases}
\end{align*}
    Applying $S_jT$ to $v_j$, we have
\begin{align*}
        S_jTv_j &= S_j(A_{1,j}v_1 + \dots + A_{n,j}v_n) \\
                &= A_{1,j}S_jv_1 + \dots + A_{n,j}S_jv_n \\
                &= A_{j,j}v_{j+1} .
\end{align*}
    Since $TS=ST$, we also have
\begin{align*}
        S_jTv_j &= T(S_jv_j) \\
                &= Tv_{j+1} \\
                &= A_{1,j+1}v_1 + \dots + A_{n,j+1}v_n .
\end{align*}
    We have expressed the same vector as two linear combinations of a basis of $V$, so the coefficients must be equal.
    The first equation only has a non-zero coefficient for $v_{j+1}$, so the coefficients of all the other $v$'s in the second equation must equal $0$.
    Thus $A_{1,j+1}=\dots=A_{j,j+1}=A_{j+2,j+1}=A_{n,j+1}=0$.

    Hence we can write the second equation more simply as
\begin{align*}
        S_jTv_j &= A_{j+1,j+1}v_{j+1} .
\end{align*}
    The coefficients of $v_{j+1}$ in the two equations are also equal.
    Hence $A_{j,j} = A_{j+1,j+1}$.

    Let $a = A_{1,1}$.
    Thus $a = A_{1,1} = A_{2,2} = \dots = A_{n,n}$.

    We can now write $Tv_k$ more simply:
\begin{align*}
        Tv_k &= A_{1,k}v_1 + \dots + A_{n,k}v_n \\
             &= A_{k,k}v_k \\
             &= a v_k .
\end{align*}
    Suppose $v \in V$.
    There exists $b_1,\dots,b_n \in \F$ such that $v=b_1v_1+\dots+b_nv_n$.
    Thus
\begin{align*}
        Tv &= T(b_1v_1+\dots+b_nv_n) \\
           &= b_1Tv_1+\dots+b_nTv_n \\
           &= b_1av_1 + \dots + b_nav_n \\
           &= a(b_1v_1 + \dots + b_nv_n) \\
           &= av .
\end{align*}
\end{document}
