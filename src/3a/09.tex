\documentclass[a5paper]{article}
\usepackage{amsmath}
\usepackage[top=1cm,right=1cm,bottom=1cm,left=1cm]{geometry}
\setlength\parindent{0pt}
%
%\usepackage{xcolor}
%\pagecolor[rgb]{0.1,0.1,0.1}
%\color[rgb]{1.0,1.0,1.0}
%
\begin{document}

\newcommand    \F  { \mathbf{F} }
\renewcommand  \L  { \mathcal{L} }
\newcommand    \R  { \mathbf{R} }
\newcommand    \C  { \mathbf{C} }

Define a function $\varphi: \C \rightarrow \C$ as
\begin{align*}
        \varphi(x+yi) &= y+xi
\intertext{
Suppose $w,z\in\C$ such that $w=a+bi$ and $z=c+di$ where $a,b,c,d\in\R$.
}
                \varphi(w+z) &= \varphi( a+bi + c+di )                  \\
                             &= \varphi\big( (a+c) + (b+d)i \big)       \\
                             &= (b+d) + (a+c)i                          \\
                             &= b + ai + d + ci                         \\
                             &= \varphi(a+bi) + \varphi(c+di)           \\
                             &= \varphi(w) + \varphi(z)
\intertext{
Now consider scalar multiplication by $i\in\C$.
}
        i \varphi(w) &= i \varphi(a+bi)                 \\
                     &= i (b + ai)                      \\
                     &= bi + ai^2                       \\
                     &= bi - a                          \\
                                                        \\
         \varphi(iw) &= \varphi\big( i(a+bi) \big)      \\
                     &= \varphi( ai + bi^2 )            \\
                     &= \varphi(ai - b)                 \\
                     &= a - bi                          \\
                                                        \\
        i \varphi(w) &\neq \varphi(iw)
\end{align*}
Thus $\varphi$ is not homogenous.
Hence $\varphi$ is not a linear map.
\end{document}
