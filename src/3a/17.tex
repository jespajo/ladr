\documentclass[a5paper]{article}
\usepackage{amsmath}
\usepackage[top=1cm,right=1cm,bottom=1cm,left=1cm]{geometry}
\setlength\parindent{0pt}
\setlength\parskip{1em}
%
%\usepackage{xcolor}
%\pagecolor[rgb]{0.1,0.1,0.1}
%\color[rgb]{1.0,1.0,1.0}
%
\begin{document}
\newcommand   \C           {\mathbf{C}}
\newcommand   \R           {\mathbf{R}}
\renewcommand \L           {\mathcal{L}}
\newcommand   \F           {\mathbf{F}}
\renewcommand \P           {\mathcal{P}}
\newcommand   \M           {\mathcal{M}}
\newcommand   \op          {\operatorname}
\newcommand \E             {\mathcal{E}}

    3.A.17.
    Suppose $V$ is finite-dimensional.
    Show that the only two-sided ideals of $\L(V)$ are $\{0\}$ and $\L(V)$.

    $\{0\}$ is a two-sided ideal of $\L(V)$ because for $T \in \L(V)$,
\begin{equation*}
        0T = T0 = 0 \in \{0\} .
\end{equation*}
    Suppose $\E$ is a two-sided ideal of $\L(V)$ such that $\E \neq \{0\}$.
    Then there exists $E \in \E$ such that $E \neq 0$.

    Since $E \in \L(V)$, it is defined with respect to a basis of $V$, denoted $v_1,\dots,v_n$, by
\begin{equation*}
        Ev_k = w_k
\end{equation*}
    for $k = 1,\dots,n$.
    Since $E \neq 0$, some $w_k \neq 0$.
    Let $p$ denote the lowest value of $k$ such that $w_k \neq 0$.

    Since $w_p \in V$, we can write it as
\begin{equation*}
        w_p = c_1v_1 + \dots + c_nv_n
\end{equation*}
    with $c_1,\dots,c_n \in \F$.
    Since $w_p \neq 0$, some $c_k \neq 0$.
    Let $q$ denote the lowest value of $k$ such that $c_k \neq 0$.

    Define $S_1,\dots,S_n,T_1,\dots,T_n \in \L(V)$ such that for $j=1,\dots,n$ and $k=1,\dots,n$,
\begin{align*}
        T_jv_k &=
            \begin{cases}
                v_p               \; &\text{for}\ k =    j\\
                0                 \; &\text{for}\ k \neq j
            \end{cases} &\text{and}& &
        S_jv_k &=
            \begin{cases}
                \frac{1}{c_q} v_j \; &\text{for}\ k =    q\\
                0                 \; &\text{for}\ k \neq q .
            \end{cases}
\end{align*}
    Define $I_1,\dots,I_n \in \L(V)$ such that each $I_j$ is $E$ sandwiched between $S_j$ and $T_j$:
\begin{equation*}
        I_j = S_jET_j .
\end{equation*}
    $ET_j \in \E$ because it is the product of $E \in \E$ and $T_j \in \L(V)$.
    Hence $I_j \in \E$ because it is the product of $S_j \in \L(V)$ and $ET_j \in \E$.

    Applying some $I_j$ to some $v_k$, there are two cases.
    If $j=k$, then
\begin{align*}
        I_jv_k &= I_jv_j \\
               &= (S_jE)(T_jv_j) \\
                  &= (S_jE)(v_p)\\
                  &= S_j(Ev_p)\\
                  &= S_jw_p\\
                  &= S_j(c_1v_1+\dots+c_nv_n)\\
                  &= c_1S_jv_1+\dots+c_nS_jv_n\\
                  &= c_q\left(\frac{1}{c_q}v_j\right)\\
                  &= v_j.
\end{align*}
    If $k\neq j$, then
\begin{align*}
        I_jv_k = (S_jE)(T_jv_k) = (S_jE)(0) = 0 .
\end{align*}
    Let $I$ denote the sum of the $I_j$'s, as in
\begin{equation*}
        I = I_1 + \dots + I_n .
\end{equation*}
    $\E$, being a subspace of $\L(V)$, is closed under addition.
    Hence $I \in \E$.

    $I$ is so-called because it is the identity map on $V$.
    To see this, suppose $v \in V$.
    Then there exist $a_1,\dots,a_n \in \F$ such that
\begin{align*}
         v&=a_1v_1+\dots+a_nv_n .
\end{align*}
    So
\begin{align*}
        Iv &= (I_1+\dots+I_n)(v)\\
           &= I_1v + \dots + I_nv\\
           &= I_1(a_1v_1 + \dots + a_nv_n) + \dots + I_n(a_1v_1 + \dots + a_nv_n)\\
           &= a_1I_1v_1 + \dots + a_nI_1v_n + \dots + a_1I_nv_1 + \dots + a_nI_nv_n\\
           &= a_1(I_1v_1 + \dots + I_nv_1) + \dots + a_n(I_1v_n + \dots + I_nv_n)\\
           &= a_1v_1 + \dots + a_nv_n\\
           &= v .
\end{align*}
    Each $I_jv_k$ term disappears unless $j=k$, in which case it equals $v_k$.

    Thus $\E$ contains the identity map on $V$.

    Now suppose $T \in \L(V)$.
    Then
\begin{equation*}
        T = TI
\end{equation*}
    so $T$ is a product of $T \in \L(V)$ and $I \in \E$.
    Hence $T \in \E$.
    Thus $\L(V) \subseteq \E$.

    $\E \subseteq \L(V)$ because $\E$ is a subspace of $\L(V)$.
    So $\E = \L(V)$.

    Hence the only two-sided ideals of $\L(V)$ are $\{0\}$ and $\L(V)$.
\end{document}
