\documentclass[a5paper]{article}
\usepackage{amsmath}
\usepackage[top=1cm,right=1cm,bottom=1cm,left=1cm]{geometry}
\setlength\parindent{0pt}
%
%\usepackage{xcolor}
%\pagecolor[rgb]{0.1,0.1,0.1}
%\color[rgb]{1.0,1.0,1.0}
%
\begin{document}

\newcommand   \F { \mathbf{F}  }
\renewcommand \L { \mathcal{L} }

\subsection*{associativity}

Suppose $T_1$, $T_2$ and $T_3$ are linear maps such that $T_3$ maps to the domain of $T_2$ and $T_2$ maps to the domain of $T_1$.
Then if $v$ is in the domain of $T_3$
\begin{align*}
    \big( (T_1T_2)T_3 \big)(v) &= (T_1T_2)(T_3v)                \\
                               &= T_1 \big( T_2 (T_3v) \big)    \\
                               &= T_1 \big( (T_2T_3)(v) \big)   \\
                               &= \big( T_1(T_2T_3) \big)(v)
\end{align*}

\subsection*{identity}

Suppose $T \in \L(V,W)$.
If $I$ is the identity map in $V$, then $Iv=v$ for $v\in V$.
Hence
\begin{align*}
    (TI)(v) = T(Iv) = Tv
\end{align*}
If $I$ is the identity map in $W$, then $ITv = Tv$ for $v \in V$.
Thus
\begin{align*}
    (IT)(v) = I(Tv) = Tv
\end{align*}

\subsection*{distributive properties}

Suppose $T\in \L(U,V)$ and $S_1,S_2 \in \L(V,W)$.
Then for $u\in U$
\begin{align*}
    \big( (S_1+S_2)T \big)(u) &= (S_1+S_2)(Tu)              \\
                              &= S_1(Tu) + S_2(Tu)          \\
                              &= (S_1T)(u) + (S_2T)(u)      \\
                              &= \big(S_1T + S_2T\big)(u)
\intertext{
Now suppose $T_1,T_2\in \L(U,V)$ and $S \in \L(V,W)$.
Then for $u\in U$
}
    \big( S(T_1+T_2) \big)(u) &= S\big( (T_1+T_2)(u) \big)  \\
                              &= S\big( T_1u + T_2u \big)   \\
                              &= S(T_1u) + S(T_2u)          \\
                              &= (ST_1)(u) + (ST_2)(u)      \\
                              &= (ST_1 + ST_2)(u)         
%
\end{align*}
\end{document}
