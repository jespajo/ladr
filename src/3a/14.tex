\documentclass[a5paper]{article}
\usepackage{amsmath}
\usepackage[top=1cm,right=1cm,bottom=1cm,left=1cm]{geometry}
\setlength\parindent{0pt}
\setlength\parskip{1em}
%
%\usepackage{xcolor}
%\pagecolor[rgb]{0.1,0.1,0.1}
%\color[rgb]{1.0,1.0,1.0}
%
\begin{document}

\newcommand    \C  { \mathbf{C} }
\newcommand    \R  { \mathbf{R} }
\renewcommand  \L  { \mathcal{L} }
\newcommand    \F  { \mathbf{F} }

$V$ is finite-dimensional with $\dim V > 0$.
$W$ is infinite-dimensional.

Because $V$ is finite-dimensional, there exists a basis of $V$, $v_1,\dots,v_m$.
Hence for $v\in V$
\begin{align*}
    v = a_1v_1 + \dots + a_mv_m
\end{align*}
where $a_1,\dots,a_m\in \F$ and $\dim V=m$.

So $T \in \L(V,W)$ can be defined uniquely with respect to this basis as
\begin{align*}
    T(a_1v_1 + \dots + a_mv_m) &= a_1w_1 + \dots + a_mw_m
\end{align*}
with $w_1,\dots,w_m\in W$.

Hence any function in $\text{span\,}(T_1,\dots,T_n)$ can be defined with $b_1,\dots,b_n \in \F$ as
\begin{align*}
    (b_1T_1 + &\dots + b_nT_n)(a_1v_1 + \dots + a_mv_m)                                                \\
           &= b_1T_1(a_1v_1 + \dots + a_mv_m)       + \dots + b_nT_n(a_1v_1 + \dots + a_mv_m)          \\
           &= b_1(a_1w_{1,1} + \dots + a_mw_{m,1})  + \dots + b_n(a_1w_{1,n} + \dots + a_mw_{m,n})     \\
           &= a_1b_1w_{1,1} + \dots + a_mb_1w_{m,1} + \dots + a_1b_nw_{1,n} + \dots + a_mb_nw_{m,n}
\end{align*}
where $T_kv_j = w_{j,k}$ for $j=1,\dots,m$ and $k=1,\dots,n$.

Thus every vector in the range of $b_1T_1+\dots+b_nT_n$ can be written as a linear combination of a list of length $mn$ of vectors in $W$.

Since $W$ is infinite-dimensional, no list spans $W$.
Hence there is some $w \in W$ that is not in the range of $b_1T_1+\dots+b_nT_n$.

We can define $S: V \rightarrow W$ as
\begin{align*}
    Sv = S(a_1v_1 + \dots + a_mv_m) = a_1w
\end{align*}
If there is any doubt that $S$ is a linear map, consider $c \in \F$:
\begin{align*}
    S(cv) &= S(ca_1v_1 + \dots + ca_mv_m)                   \\
          &= ca_1w                                          \\
          &= cSv
\intertext{
Thus $S$ is homogeneous. Next consider $u = c_1v_1 + \dots + c_mv_m \in V$:
}
    S(v + u) &= S\big( (a_1+c_1)v_1 + \dots + (a_m+c_m)v_m \big)   \\
             &= (a_1+c_1)w                                         \\
             &= a_1w + c_1w                                        \\
             &= Sv + Su
\end{align*}
Thus $S$ is additive.
Hence $S$ is a linear map.

$S \notin \text{span\,}(T_1,\dots,T_n)$ because $Sv_1=u$ and $u \notin \text{range span\,}(T_1,\dots,T_n)$.

Hence $\L(V,W)$ is infinite-dimensional because no list spans $\L(V,W)$.
\end{document}
