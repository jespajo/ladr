\documentclass[a5paper]{article}
\usepackage{amsmath}
\usepackage[top=1cm,right=1cm,bottom=1cm,left=1cm]{geometry}
\setlength\parindent{0pt}
\setlength\parskip{1em}
%
%\usepackage{xcolor}
%\pagecolor[rgb]{0.1,0.1,0.1}
%\color[rgb]{1.0,1.0,1.0}
%
\begin{document}

\newcommand    \C  { \mathbf{C} }
\newcommand    \R  { \mathbf{R} }
\renewcommand  \L  { \mathcal{L} }
\newcommand    \F  { \mathbf{F} }

Since $V$ is finite-dimensional with $\dim V \ge 2$, a basis of $V$ exists, $v_1,\dots,v_n$, with $n \ge 2$.
Hence every $v\in V$ can be written as
\begin{align*}
    v &= a_1v_1 + \dots + a_nv_n
\end{align*}
with $a_1,\dots,a_n\in \F$.

Define $T:V\rightarrow V$ by
\begin{align*}
    T(a_1v_1 + \dots + a_nv_n) &= a_2v_1 + \dots + a_nv_{n-1}
\intertext{
Define $S:V\rightarrow V$ by
}
    S(a_1v_1 + \dots + a_nv_n) &= a_1v_2 + \dots + a_{n-1}v_n
\end{align*}
To show that $T$ and $S$ are homogeneous, consider $\lambda \in \F$:
\begin{align*}
    T(\lambda v) &= T \big( \lambda(a_1v_1 + \dots + a_nv_n) \big)      \\
                 &= T ( \lambda a_1v_1 + \dots + \lambda a_nv_n )       \\
                 &= \lambda a_2v_1 + \dots + \lambda a_nv_{n-1}         \\
                 &= \lambda (a_2v_1 + \dots + a_nv_{n-1})               \\
                 &= \lambda Tv                                          \\
                                                                        \\
    S(\lambda v) &= S ( \lambda a_1v_1 + \dots + \lambda a_nv_n )       \\
                 &= \lambda a_1v_2 + \dots + \lambda a_{n-1}v_n         \\
                 &= \lambda (a_1v_2 + \dots + a_{n-1}v_n)               \\
                 &= \lambda Sv
\intertext{
To show that $T$ and $S$ are additive, consider $u = b_1v_1 + \dots + b_nv_n \in V$ with $b_1,\dots,b_n \in \F$:
}
        T(u + v) &= T \big( (a_1 + b_1)v_1 + \dots + (a_n + b_n)v_n \big)       \\
                 &= (a_2 + b_2)v_1 + \dots + (a_n + b_n)v_{n-1}                 \\
                 &= a_2v_1 + b_2v_1 + \dots + a_nv_{n-1} + b_nv_{n-1}           \\
                 &= a_2v_1 + \dots + a_nv_{n-1} + b_2v_1 + \dots + b_nv_{n-1}   \\
                 &= Tu + Tv                                                     \\
                                                                                \\
        S(u + v) &= S \big( (a_1 + b_1)v_1 + \dots + (a_n + b_n)v_n \big)       \\
                 &= (a_1 + b_1)v_2 + \dots + (a_{n-1} + b_{n-1})v_n             \\
                 &= a_1v_2 + b_1v_2 + \dots + a_{n-1}v_n + b_{n-1})v_n          \\
                 &= a_1v_2 + \dots + a_{n-1}v_n + b_1v_2 + \dots + b_{n-1})v_n  \\
                 &= Su + Sv
\end{align*}
Hence $T, S \in \L(V,V)$.
\begin{align*}
    (ST)(v) &= S\big( T(a_1v_1 + \dots + a_nv_n) \big)              \\
            &= S ( a_2v_1 + \dots + a_nv_{n-1} + 0v_n )             \\
            &= a_2v_2 + \dots + a_nv_n                              \\
                                                                    \\
    (TS)(v) &= T\big( S(a_1v_1 + \dots + a_nv_n) \big)              \\
            &= T ( 0v_1 + a_1v_2 + \dots + a_{n-1}v_n )             \\
            &= a_1v_1 + \dots + a_{n-1}v_{n-1}
\end{align*}
Thus $ST \neq TS$ for $\dim V \ge 2$.
(For $\dim V < 2$, $ST$ and $TS$ are both the zero map on $V$.)
\end{document}
