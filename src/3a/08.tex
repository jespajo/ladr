\documentclass[a5paper]{article}
\usepackage{amsmath}
\usepackage[top=1cm,right=1cm,bottom=1cm,left=1cm]{geometry}
\setlength\parindent{0pt}
%
%\usepackage{xcolor}
%\pagecolor[rgb]{0.1,0.1,0.1}
%\color[rgb]{1.0,1.0,1.0}
%
\begin{document}

\newcommand   \F    { \mathbf{F} }
\renewcommand \L    { \mathcal{L} }
\newcommand   \R    { \mathbf{R} }

Suppose $v = (x,y) \in \R^2 : x,y \in \R$.
Define a function $\varphi: \R^2 \rightarrow \R$ as
\begin{align*}
        \varphi(v) &= \sqrt{x^2 + y^2}
\intertext{
Suppose $a \in \R$, then $av = (ax, ay)$.
Hence
}
        \varphi(av) &= \sqrt{(ax)^2 + (ay)^2}           \\
                    &= \sqrt{a^2x^2 + a^2y^2}           \\
                    &= \sqrt{a^2(x^2 + y^2)}            \\
                    &= \sqrt{a^2} \sqrt{x^2 + y^2}      \\
                    &= a\sqrt{x^2 + y^2}                \\
                    &= a\varphi(v)
\intertext{
Thus $\varphi$ is homogenous.
Now suppose $u=(3,4)\in\R^2$ and $w=(4,3)\in\R^2$.
}
    \varphi(u) + \varphi(w) &= \sqrt{3^2+4^2} + \sqrt{4^2+3^2}  \\
                            &= \sqrt{9+16} + \sqrt{16+9}        \\
                            &= \sqrt{25} + \sqrt{25}            \\
                            &= 5 + 5                            \\
                            &= 10                               \\
                                                                \\
               \varphi(u+w) &= \sqrt{(3+4)^2 + (4+3)^2}         \\
                            &= \sqrt{7^2 + 7^2}                 \\
                            &= \sqrt{49 + 49}                   \\
                            &= \sqrt{98}                        \\
                                                                \\
      \varphi(u)+\varphi(w) &\neq \varphi(u+w)
\end{align*}
Thus $\varphi$ is not additive.
Hence $\varphi$ is not a linear map.
\end{document}
