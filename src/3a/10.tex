\documentclass[a5paper]{article}
\usepackage{amsmath}
\usepackage[top=1cm,right=1cm,bottom=1cm,left=1cm]{geometry}
\setlength\parindent{0pt}
%
\usepackage{xcolor}
\pagecolor[rgb]{0.1,0.1,0.1}
\color[rgb]{1.0,1.0,1.0}
%
\begin{document}

\newcommand    \F  { \mathbf{F} }
\renewcommand  \L  { \mathcal{L} }
\newcommand    \R  { \mathbf{R} }
\newcommand    \C  { \mathbf{C} }

$U$ is a subspace of $V$ with $U \neq V$.
$S \in \L(U,W)$ and $S \neq 0$.

$T: V \rightarrow W$ is defined by
\begin{align*}
  Tv =
    \begin{cases}
        Sv  &\text{if } v \in U, \\
        0   &\text{if } v \in V \text{ and } v \notin U.
    \end{cases}
\end{align*}

Since $S$ is not the zero map on $U$, there is some $u \in U$ for which $Su \neq 0$ and since $Tu = Su$ for $u \in U$,
\begin{align*}
    Tu &\neq 0
\intertext{
Now because $U$ is a subset of $V$ with $U \neq V$, there exists some $w \in V such that w\notin U$.
Hence
}
         Tw &= 0        \\
      Tu+Tw &= Tu       \\
      Tu+Tw &\neq 0
\intertext{
$U$ does not contain $u+w$ because otherwise could write $w$ as a sum of elements of $U$, namely $w=(u+w)-u$.
So
}
        T(u+w) &= 0
\intertext{
Thus
}
    Tu + Tw &\neq T(u+w)
\end{align*}
Therefore $T$ is not a linear map since it is not additive.
\end{document}
