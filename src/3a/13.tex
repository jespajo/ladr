\documentclass[a5paper]{article}
\usepackage{amsmath}
\usepackage[top=1cm,right=1cm,bottom=1cm,left=1cm]{geometry}
\setlength\parindent{0pt}
\setlength\parskip{1em}
%
%\usepackage{xcolor}
%\pagecolor[rgb]{0.1,0.1,0.1}
%\color[rgb]{1.0,1.0,1.0}
%
\begin{document}

\newcommand    \C  { \mathbf{C} }
\newcommand    \R  { \mathbf{R} }
\renewcommand  \L  { \mathcal{L} }
\newcommand    \F  { \mathbf{F} }

$U$ is a subspace of $V$ and $S \in \L(U,W)$.

$V$ is finite-dimensional so $U$ is finite-dimensional.
Thus a basis of $U$ exists, which we'll denote $u_1,\dots,u_m$.

Hence for $u \in U$
\begin{align*}
     u &= a_1u_1 + \dots + a_mu_m       \\
    Su &= S(a_1u_1 + \dots + a_mu_m)
\end{align*}
where $a_1,\dots,a_m \in \F$.

Seen as a linearly independent list in $V$, the basis of $U$ can be extended to a basis of $V$.
Thus
\begin{align*}
    u_1,\dots,u_m,v_1,\dots,v_n
\end{align*}
is a basis of $V$.

By 3.5, there exists a linear map $T \in \L(V,W)$ such that $Tu_j=Su_j$ for $j=1,\dots,m$ and $Tv_k=w_k$ for $k=1,\dots,n$ where $w_k$ is some vector in $W$.

Hence for $u\in V$
\begin{align*}
    Tu  &= T(b_1u_1 + \dots + b_mu_m + c_1v_1 + \dots + c_nv_n)     \\
        &= b_1Tu_1 + \dots + b_mTu_m + c_1Tv_1 + \dots + c_nTv_n    \\
        &= b_1Su_1 + \dots + b_mSu_m + c_1w_1 + \dots + c_nw_n
\end{align*}
where $b_1,\dots,b_m,c_1,\dots,c_n\in\F$.

If $u \in U$, then $u$ can be written uniquely as a linear combination of $u_1,\dots,u_m,v_1,\dots,v_n$ by taking its representation as a linear combination of the basis of $U$ and adding each $v_k$ multiplied by 0.
Hence
\begin{align*}
    Tu  &= T(a_1u_1 + \dots + a_mu_m + 0v_1 + \dots + 0v_n)     \\
        &= a_1Su_1 + \dots + a_mSu_m + 0w_1 + \dots + 0w_n      \\
        &= a_1Su_1 + \dots + a_mSu_m                            \\
        &= S(a_1u_1 + \dots + a_mu_m)
\end{align*}
Thus $Tu=Su$ for all $u\in U$.
\end{document}
