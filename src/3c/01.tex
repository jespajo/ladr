\documentclass[a5paper]{article}
\usepackage{amsmath}
\usepackage[top=1cm,right=1cm,bottom=1cm,left=1cm]{geometry}
\setlength\parindent{0pt}
\setlength\parskip{1em}
%\usepackage{xcolor}
%\pagecolor[rgb]{0.1,0.1,0.1}
%\color[rgb]{1.0,1.0,1.0}
\begin{document}
\newcommand   \C           {\mathbf{C}}
\newcommand   \R           {\mathbf{R}}
\renewcommand \L           {\mathcal{L}}
\newcommand   \F           {\mathbf{F}}
\renewcommand \P           {\mathcal{P}}
\newcommand   \question[1] {\textbf{\boldmath#1\unboldmath}\par}
\newcommand   \op          {\operatorname}

\question{
    3.C.1.
    Suppose $T \in \L(V, W)$.
    Show with respect to each choice of bases of $V$ and $W$, the matrix of $T$ has at least $\op{dim} \op{range} T$ nonzero entries.
}

    Suppose $v_1,\dots,v_n$ is a basis of $V$ and $w_1,\dots,w_m$ is a basis of $W$.

    For $k \in \{1,\dots,n\}$,
\begin{equation*}
        Tv_k = A_{1,k}w_1 + \dots + A_{m,k}w_m
\end{equation*}
    where each $A_{j,k}$ is the entry in row $j$, column $k$ of the matrix of $T$.

    Suppose we reduce $v_1,\dots,v_n$ to a new list of vectors, including each $v_k$ only if every entry in the $k$th column of the matrix of $T$ is zero.
    Let $p$ denote the length of this new list.

    $T$ sends every vector in the new list to zero.
    So the new list would be a linearly independent list of vectors in the null space of $T$.
    Thus
\begin{equation*}
        p \le \op{dim}\op{null}T .
\end{equation*}

    From the Fundamental Theorem of Linear Maps,
\begin{equation*}
        \op{dim}V - \op{dim}\op{null}T = \op{dim}\op{range}T .
\end{equation*}
    $\op{dim}V = n$, so the left side of the equation can only get larger if we replace it with $n - p$:
\begin{equation*}
        n - p \ge \op{dim}\op{range}T .
\end{equation*}
    Since $n - p$ is the number of columns of the matrix of $T$ with at least one nonzero entry, we have the desired result.
\end{document}
