\documentclass[a5paper]{article}
\usepackage{amsmath}
\usepackage{amssymb}
\usepackage[top=1cm,right=1cm,bottom=1.5cm,left=1cm]{geometry}
\setlength\parindent{0pt}
\setlength\parskip{1em}
%
%\usepackage{xcolor}
%\pagecolor[rgb]{0.1,0.1,0.1}
%\color[rgb]{1.0,1.0,1.0}
%
\begin{document}
\newcommand   \C           {\mathbf{C}}
\newcommand   \R           {\mathbf{R}}
\renewcommand \L           {\mathcal{L}}
\newcommand   \F           {\mathbf{F}}
\renewcommand \P           {\mathcal{P}}
\newcommand   \M           {\mathcal{M}}
\newcommand   \op          {\operatorname}


    3.C.12.
    Prove that matrix multiplication is associative.
    In other words, suppose $A$, $B$, and $C$ are matrices whose sizes are such that $(AB)C$ makes sense.
    Explain why $A(BC)$ makes sense and prove that
\begin{align*}
        (AB)C = A(BC) .
\end{align*}

    Since $(AB)C$ makes sense, we know the number of columns in $AB$ equals the number of rows in $C$.
    $B$ has the same number of columns as $AB$, so the number of columns in $B$ also equals the number of rows in $C$.
    Thus $BC$ makes sense.

    Since $AB$ makes sense, the number of columns in $A$ equals the number of rows in $B$.
    $BC$ has the same number of rows as $B$.
    Thus the number of columns in $A$ equals the number of rows in $BC$.
    Hence $A(BC)$ makes sense.

    Let $m$ denote the number of rows in $A$. Let $n$ denote the number of columns in $C$.
    Both $(AB)C$ and $A(BC)$ are $m$-by-$n$ matrices.

%    For each $j \in \{1,\dots,m\}$, row $j$ of $A(BC)$ equals (row $j$ of $A$) times $BC$.
%    This is a linear combination of the rows of $BC$, with the coefficients for each row coming from the corresponding entry in row $j$ of $A$.
%
%    For each $j \in \{1,\dots,m\}$, row $j$ of $(AB)C$ equals (row $j$ of $AB$) times $C$.
%    Row $j$ of $AB$ equals (row $j$ of $A$) times $B$.
%    Thus row $j$ of $(AB)C$ equals ((row $j$ of $A$) times $B$) times $C$.
%
%    For each $k \in \{1,\dots,n\}$, column $k$ of $(AB)C$ equals $AB$ times (column $k$ of $C$).

    Let $p$ denote the number of rows in $B$.
    Let $q$ denote the number of columns in $B$.

    For each $j \in \{1,\dots,m\}$,  and each $k \in \{1,\dots,n\}$,
\begin{align*}
        ((AB)C)_{j,k} &= \sum_{x=1}^q (AB)_{j,x} C_{x,k} \\
                      &= \sum_{x=1}^q ( \sum_{y=1}^p A_{j,y} B_{y,x} ) C_{x,k} \\
                      &= \sum_{x=1}^q \sum_{y=1}^p A_{j,y} B_{y,x} C_{x,k} \\
                      &= \sum_{y=1}^p A_{j,y} \sum_{x=1}^q B_{y,x} C_{x,k} \\
                      &= \sum_{y=1}^p A_{j,y} (BC)_{y,k} \\
                      &= (A(BC))_{j,k} .
\end{align*}
    Thus $(AB)C$ and $A(BC)$ are matrices of the same size with the same entries.
\end{document}
