\documentclass[a5paper]{article}
\usepackage{amsmath}
\usepackage{amssymb}
\usepackage[top=1cm,right=1cm,bottom=2cm,left=1cm]{geometry}
\setlength\parindent{0pt}
\setlength\parskip{1em}
%
\begin{document}
\newcommand   \C           {\mathbf{C}}
\newcommand   \R           {\mathbf{R}}
\renewcommand \L           {\mathcal{L}}
\newcommand   \F           {\mathbf{F}}
\renewcommand \P           {\mathcal{P}}
\newcommand   \M           {\mathcal{M}}
\newcommand   \op          {\operatorname}


    3.C.15.
    Prove that if $A$ is an $m$-by-$n$ matrix and $C$ is an $n$-by-$p$ matrix, then
\begin{align*}
        (AC)^t = C^t A^t .
\end{align*}

    $AC$ is an $m$-by-$p$ matrix.
    So the value on the left is a $p$-by-$m$ matrix.

    The value on the right is a $p$-by-$n$ matrix multiplied by an $n$-by-$m$ matrix, which is also a $p$-by-$m$ matrix.

    For each $j \in \{1,\dots\,m\}$ and each $k \in \{1,\dots,p\}$,
\begin{align*}
        (C^t A^t)_{k,j} &= \sum_{r=1}^n (C^t)_{k,r} (A^t)_{r,j} \\
                        &= \sum_{r=1}^n (A^t)_{r,j} (C^t)_{k,r} \\
                        &= \sum_{r=1}^n A_{j,r} C_{r,k} \\
                        &= (AC)_{j,k} \\
                        &= ((AC)^t)_{k,j} .
\end{align*}
    %The first and fourth equalities hold due to the definition of matrix multiplication.
    %The second equality holds due to the commutativity of multiplication of scalars.
    %The third and fifth equalities hold due to the definition of the transpose of a matrix.

    Thus $(AC)^t$ and $C^tA^t$ are matrices of the same size with the same entries.
\end{document}
