\documentclass[a5paper]{article}
\usepackage{amsmath}
\usepackage{amssymb}
\usepackage[top=1cm,right=1cm,bottom=1cm,left=1cm]{geometry}
\setlength\parindent{0pt}
\setlength\parskip{1em}
%
%\usepackage{xcolor}
%\pagecolor[rgb]{0.1,0.1,0.1}
%\color[rgb]{1.0,1.0,1.0}
%
\begin{document}
\newcommand   \C           {\mathbf{C}}
\newcommand   \R           {\mathbf{R}}
\renewcommand \L           {\mathcal{L}}
\newcommand   \F           {\mathbf{F}}
\renewcommand \P           {\mathcal{P}}
\newcommand   \M           {\mathcal{M}}
\newcommand   \op          {\operatorname}

    3.C.6.
    Suppose $v_1,\dots,v_m$ is a basis of $V$ and $W$ is finite-dimensional.
    Suppose $T \in \L(V,W)$.
    Prove that there exists a basis $w_1,\dots,w_n$ of $W$ such that all entries in the first column of $\M(T)$ [with respect to the bases $v_1,\dots,v_m$ and $w_1,\dots,w_n$] are 0 except for possibly a 1 in the first row, first column.

    Suppose $Tv_1 = 0$.
    Then suppose $w_1,\dots,w_n$ is any basis of $W$.
    The only linear combination of $w_1,\dots,w_n$ equal to $Tv_1$ has 0 for all scalars.
    Thus all entries in the first column of $\M(T)$ are 0, so we are done.

    Suppose $Tv_1 \neq 0$.
    Then let $w_1$ denote $Tv_1$.
    Since $w_1$ is a nonzero vector in $W$, we can extend it to a basis of $W$.
    Let $w_2,\dots,w_n$ denote the vectors added for this extension.

    Since $w_1,\dots,w_n$ is a basis of $W$, the only linear combination of this list equal to $w_1$ has $w_1$ multiplied by 1 and all the other $w$'s multiplied by 0.
    Hence the first column of $\M(T)$ defined with respect to these bases has 1 in the first entry and 0 in the other entries.
\end{document}
