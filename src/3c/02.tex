\documentclass[a5paper]{article}
\usepackage{amsmath}
\usepackage[top=1cm,right=1cm,bottom=1cm,left=1cm]{geometry}
\setlength\parindent{0pt}
\setlength\parskip{1em}
%
\usepackage{xcolor}
\pagecolor[rgb]{0.1,0.1,0.1}
\color[rgb]{1.0,1.0,1.0}
%
\begin{document}
\newcommand   \C           {\mathbf{C}}
\newcommand   \R           {\mathbf{R}}
\renewcommand \L           {\mathcal{L}}
\newcommand   \F           {\mathbf{F}}
\renewcommand \P           {\mathcal{P}}
\newcommand   \M           {\mathcal{M}}
\newcommand   \question[1] {\textbf{\boldmath#1\unboldmath}\par}
\newcommand   \op          {\operatorname}

\question{
    3.C.2.
    Suppose $V$ and $W$ are finite-dimensional and $T \in \L(V,W)$.
    Prove that $\op{dim}\op{range}T = 1$ if and only if there exist a basis of $V$ and a basis of $W$ such that, with respect to these bases, all entries of $\M(T)$ equal 1.
}

    To prove in one direction, suppose there exist a basis of $V$ and a basis of $W$ such that, with respect to these bases, all entries of $\M(T)$ equal 1.
    Let $v_1,\dots,v_n$ denote the basis of $V$ and let $w_1,\dots,w_m$ denote the basis of $W$.

    Then for $k \in \{1,\dots,n\}$,
\begin{align*}
        Tv_k &= A_{1,k}w_1 + \dots + A_{m,k}w_m  \\
             &= w_1 + \dots + w_m .
\end{align*}
    For $v \in V$, there exist $a_1,\dots,a_n \in \F$ such that
\begin{align*}
        Tv &= T(a_1v_1 + \dots + a_nv_n) \\
           &= a_1Tv_1 + \dots + a_nTv_n \\
           &= a_1(w_1 + \dots + w_m) + \dots + a_n(w_1 + \dots + w_m) \\
           &= a_1w_1 + \dots + a_1w_m + \dots + a_nw_1 + \dots + a_nw_m \\
           &= (a_1+\dots+a_n)w_1 + \dots + (a_1+\dots+a_n)w_m \\
           &= (a_1 + \dots + a_n)(w_1 + \dots + w_m) .
\end{align*}
    Hence any $w \in \op{range}T$ is the vector $w_1+\dots+w_m$ multiplied by some scalar.
    Thus the vector $w_1+\dots+w_m$ spans $\op{range}T$.

    In fact the vector $w_1+\dots+w_m$ is a basis of $\op{range}T$.
    (A list containing only this vector is linearly independent because the vector is not 0, which we know because the vector is itself a linear combination of a linearly independent list with nonzero scalars.)
    Hence $\op{dim}\op{range}T = 1$.

    To prove in the other direction, suppose $\op{dim}\op{range}T = 1$.

    Suppose $u_1,\dots,u_n$ is a basis of $\op{null}T$.
    Extend this list to a basis of $V$.
    From the Fundamental Theorem of Linear Maps, we know that $\op{dim}V = \op{dim}\op{null}T + 1$.
    So the expansion only requires adding one vector, which we'll call $v$.
    So we have the list $u_1, \dots, u_n, v$, a basis of $V$.

    Modify this list by adding $v$ to each vector except the last one.
    This creates a new list:
\begin{equation*}
        u_1+v, \dots, u_n+v, v .
\end{equation*}
    The new list is also a basis of $V$, because it is a linearly independent list of the right length.
    To see that it is linearly independent, consider a linear combination of the list equal to 0.
    With $a_1,\dots,a_{n+1} \in \F$,
\begin{align*}
        0 &= a_1(u_1 + v) + \dots + a_n(u_n + v) + a_{n+1}v \\
          &= a_1u_1 + a_1v + \dots + a_nu_n + a_nv + a_{n+1}v \\
          &= a_1u_1 + \dots + a_nu_n + (a_1 + \dots + a_{n+1})v .
\end{align*}
    Since the final line is a linear combination of a linearly independent list, $a_1=\dots=a_{n+1}=0$.
    Hence we have a basis of $V$.

    Applying $T$ to any vector in this basis results in $Tv$.
    Clearly this is true for the final vector $v$.
    For the others,
\begin{equation*}
        T(u_k + v) = Tu_k + Tv = Tv .
\end{equation*}
    The $Tu_k$ term disappears because $u_k$ is in the null space of $T$.

    $Tv$ is a basis of $\op{range}T$.
    This is true because any nonzero vector in $\op{range}T$ is a linearly independent list of length 1, which is the right length because $\op{dim}\op{range}T = 1$.
    We know that $Tv$ is nonzero because $v \notin \op{null}T$ (by definition, otherwise $v$ would be expressible as a linear combination of the $u$'s).

    Suppose we extend $Tv$ to a basis of $W$ by appending $w_1,\dots,w_m \in W$.
    This creates the list $Tv, w_1, \dots, w_m$.

    Modify this list by subtracting from each vector the vector to its right, if there is one.
    This creates a new list:
\begin{equation*}
        Tv-w_1, w_1-w_2, \dots, w_{m-1}-w_m, w_m .
\end{equation*}
    This new list is also a basis of $W$.
    We will show this by showing, as before, that it is a linearly independent list of the right length.
    Suppose $b_1,\dots,b_{m+1} \in \F$ such that
\begin{align*}
        0 &= b_1(Tv-w_1) + b_2(w_1-w_2) + \dots + b_m(w_{m-1}-w_m) + b_{m+1}w_m \\
          &= b_1Tv - b_1w_1 + b_2w_1 - b_2w_2 + \dots + b_mw_{m-1} - b_mw_m + b_{m+1}w_m \\
          &= b_1Tv + (b_2 - b_1)w_1 + \dots + (b_{m+1} - b_m)w_m .
\end{align*}
    Since the final line is a linear combination of a linearly independent list, $b_1=\dots=b_{m+1}=0$.
    Hence we have a basis of $W$.

    We now have a basis of $V$ and a basis of $W$ with respect to which we can define $\M(T)$.
    In particular,
\begin{equation*}
        \M \big(T,  (u_1+v, \dots, u_n+v, v),  (Tv-w_1, w_1-w_2, \dots, w_{m-1}-w_m, w_m) \big) .
\end{equation*}
    We have already shown that $T$ sends every vector in the basis of $V$ to the same vector, $Tv$.
    Hence all the columns in $\M(T)$ are identical, since each column is determined by the result of applying $T$ to the corresponding vector in the basis of $V$.

    In particular, every column of $\M(T)$ is made up of the scalars in the unique linear combination of the basis of $W$ that equals $Tv$.
    Thus, to complete our proof, we need to show that these scalars are all 1.
\begin{align*}
        Tv &= Tv + (w_1 - w_1) + \dots + (w_m - w_m) \\
           &= Tv - w_1 + w_1 - \dots - w_m + w_m \\
           &= (Tv - w_1) + (w_1 - w_2) + \dots + (w_{m-1} - w_m) + w_m ,
\end{align*}
    as desired.
\end{document}
%
%  for example consider
%      T(x,y,z) = (x+y+z, 2x+2y+2z)
%      null T = {(x,y,z) in F3 : x+y+z = 0}
%      dim null T = 2
%      dim range T = 1
%      basis of null T = (1,0,-1), (1,1,-2)
%      v = (1,0,0)
%      basis of V = (2,0,-1), (2,1,-2), (1,0,0)
%      Tv = (1,2)
%      w1 = (0,1)
%      basis of W = (1,1), (0,1)
%
%  now consider vector (1,1,1)
%  defined in terms of the above basis of V this would be (-3,1,5)
%  so the result of the matrix multiplication would be
%   (1 1 1) (-3) (3)
%   (1 1 1)x( 1)=(3)
%           ( 5)
%  so we have 3 times each vector in the basis of W
%  this is 3*(1,1) + 3*(0,1) = (3,3) + (0,3) = (3,6)
%  which is correct because that's (1+1+1, 2+2+2) !
%
