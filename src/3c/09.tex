\documentclass[a5paper]{article}
\usepackage{amsmath}
\usepackage{amssymb}
\usepackage[top=1cm,right=1cm,bottom=1cm,left=1cm]{geometry}
\setlength\parindent{0pt}
\setlength\parskip{1em}
%
%\usepackage{xcolor}
%\pagecolor[rgb]{0.1,0.1,0.1}
%\color[rgb]{1.0,1.0,1.0}
%
\begin{document}
\newcommand   \C           {\mathbf{C}}
\newcommand   \R           {\mathbf{R}}
\renewcommand \L           {\mathcal{L}}
\newcommand   \F           {\mathbf{F}}
\renewcommand \P           {\mathcal{P}}
\newcommand   \M           {\mathcal{M}}
\newcommand   \op          {\operatorname}


    3.C.9.
    Suppose $a = \begin{pmatrix} a_1 & \dots & a_n \end{pmatrix}$ is a 1-by-$n$ matrix and $B$ is an $n$-by-$p$ matrix.
    Prove that
\begin{align*}
        aB = a_1B_{1,\cdot} + \dots + a_nB_{n,\cdot} .
\end{align*}
    In other words, show that $aB$ is a linear combination of the rows of $B$, with the scalars that multiply the rows coming from $a$.

\rule{\linewidth}{0.4pt}


    Considering the left side of the equation, the definition of matrix multiplication implies that it is a 1-by-$p$ matrix such that for $k = 1,\dots,p$, the entry in its $k$th column is
\begin{align*}
        a_1B_{1,k} + \dots + a_nB_{n,k} .
\end{align*}
    On the right side of the equation in the question, each term is a scalar multiple of a 1-by-$p$ matrix.
    From the definitions of matrix scalar multiplication and matrix addition, the right side of the equation as a whole is also a 1-by-$p$ matrix such that for $k = 1,\dots,p$, the entry in its $k$th column is the number above.

    Hence both sides of the equation are the same.
\end{document}
