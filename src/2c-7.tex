\documentclass[a4paper]{article}
\usepackage[leqno]{amsmath}
\usepackage[top=1cm,right=1cm,bottom=1cm,left=1cm]{geometry}
\begin{document}
\large
\begin{align*}
\tag{a}
U &= \{p \in \mathcal{P}_4(\textbf{F}) : p(2) = p(5) = p(6)\} \\
\intertext{From the previous question, }
  p(2) &= p(5) \\
  p(z) &= a_0 + a_2(z^2 - 7z) + a_3(z^3 - 39z) + a_4(z^4 - 203z)
\intertext{Now, }
  p(2) &= a_0 + a_2(2^2 - 7(2)) + a_3(2^3 - 39(2)) + a_4(2^4 - 203(2)) \\
  p(2) &= a_0 + a_2(4 - 14) + a_3(8 - 78) + a_4(16 - 406) \\
  p(2) &= a_0 - 10a_2       - 70a_3       - 390a_4 \\
\\
  p(6) &= a_0 + a_2(6^2 - 7(6)) + a_3(6^3 - 39(6)) + a_4(6^4 - 203(6)) \\
  p(6) &= a_0 + a_2(36 - 42) + a_3(216 - 234) + a_4(1296 - 1218) \\
  p(6) &= a_0 - 6a_2 - 18a_3 + 78a_4 \\
\\
  p(2) &= p(6) \\
  a_0 - 10a_2 - 70a_3 - 390a_4 &= a_0 - 6a_2 - 18a_3 + 78a_4 \\
       -10a_2 - 70a_3 - 390a_4 &=     - 6a_2 - 18a_3 + 78a_4 \\
               -70a_3 - 390a_4 &=       4a_2 - 18a_3 + 78a_4 \\
                       -390a_4 &=       4a_2 + 52a_3 + 78a_4 \\
                             0 &=       4a_2 + 52a_3 + 468a_4 \\
                          4a_2 &=             -52a_3 - 468a_4 \\
                           a_2 &=             -13a_3 - 117a_4
\end{align*}
\begin{align*}
  p(z) &= a_0 + (-13a_3 - 117a_4)(z^2 - 7z) + a_3(z^3 - 39z) + a_4(z^4 - 203z) \\
  p(z) &= a_0 - 13a_3z^2 + 13a_3(7z) - 117a_4z^2 + 117a_4(7z) + a_3(z^3 - 39z) + a_4(z^4 - 203z) \\
  p(z) &= a_0 + a_3(-13z^2 + 13(7z)) + a_4(-117z^2 + 117(7z)) + a_3(z^3 - 39z) + a_4(z^4 - 203z) \\
  p(z) &= a_0 + a_3(-13z^2 + 91z) + a_4(-117z^2 + 819z) + a_3(z^3 - 39z) + a_4(z^4 - 203z) \\
  p(z) &= a_0 + a_3(-13z^2 + 91z + z^3 - 39z) + a_4(-117z^2 + 819z + z^4 - 203z) \\
  p(z) &= a_0 + a_3(z^3 - 13z^2 + 52z) + a_4(z^4 - 117z^2 + 616z) \\
\end{align*}
The list $1$, $z^3-13z^2+52z$, $z^4-117z^2+616z$ spans $U$ because, as shown above, every $p\in U$ can be represented as a linear combination of this list.
The list is linearly independent because any choice of scalars other than all zeroes would sum to a non-zero polynomial.
Hence the list is a basis of $U$.
\\
\\
(b) The polynomial $p(z)=z$ is not in $U$ because $p(2)\neq p(5)$.
Adjoining $z$ to the list above results in a list of length four.
The span of that list does not include $z^2$ because any linear combination of the list with a non-zero coefficient in $z^2$ must also have a non-zero coefficient in $z^3$ or $z^4$.
Therefore we can adjoin $z$ and then $z^2$ to the basis of $U$ to get a linearly independent list of polynomials in $\mathcal{P}_4(\textbf{F})$. This list is a basis of $\mathcal{P}_4(\textbf{F})$ because it has length five.
\\
\begin{align*}
\tag{c}
\mathcal{P}_4(\textbf{F}) &= U \oplus W \\
W &= \text{span}(z, z^2) \\
W &= \{ p \in \mathcal{P}_4(\textbf{F}) : p(z) = az + bz^2 : a,b \in \textbf{F} \}
\end{align*}
\end{document}
 
