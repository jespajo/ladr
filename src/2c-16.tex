\documentclass[a4paper]{article}
\usepackage[leqno]{amsmath}
\usepackage[top=1cm,right=1cm,bottom=1cm,left=1cm]{geometry}
\setlength\parindent{0pt}
\begin{document}
\Large
Suppose $v \in U_1 \oplus \dots \oplus U_m$. Then, 
\begin{align*}
    v = u_1 + \dots + u_m \text{ for some } u_1 \in U_1, \dots, u_m \in U_m
\end{align*}
In the equation above, each $U_j$ is finite-dimensional, so it has a basis, and $u_j$ is some linear combination of the basis.
Hence $v$ is a sum of these linear combinations.
This sum is itself a linear combination of a list of vectors: the list you'd get by joining all the bases into one big list.
Therefore the subspace $U_1\oplus\dots\oplus U_m$ is finite-dimensional since this joined list of bases spans the subspace.
\\
\\
Now suppose $v=0$.
Because $U_1\oplus\dots\oplus U_m$ is a direct sum, $u_1=\dots=u_m=0$.
Furthermore, if we expand each $u_j$ into a linear combination of a basis of $U_j$ like this:
\begin{align*}
    u_j &= a_1v_1 + \dots + a_nv_n = 0
\end{align*}
where $v_1,\dots,v_n$ is a basis of $U_j$, then every $a$ in the equation is zero.
\\
\\
In other words, if we join the bases of $U_1,\dots,U_m$ into one list, the only linear combination of the list equal to 0 has every vector multiplied by zero.
Hence this list is a basis of the subspace $U_1\oplus\dots\oplus U_m$.
\\
\\
We have shown that bases of $U_1,\dots,U_m$ can be joined together to create a basis of the direct sum.
Therefore,
\begin{align*}
    \text{dim }U_1\oplus\dots\oplus U_m = \text{dim }U_1+\dots+\text{dim }U_m
\end{align*}
because the length of the joined list (the left side of the equation) is equal to the sum of the lengths of the lists that went into it (the right side of the equation).
\end{document}
