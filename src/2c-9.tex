\documentclass[a4paper]{article}
\usepackage[leqno]{amsmath}
\usepackage[top=1cm,right=1cm,bottom=1cm,left=1cm]{geometry}
\setlength\parindent{0pt}
\begin{document}
\Large
Suppose $U = \text{span}(v_1 + w, ..., v_m + w)$.
If the list $v_1 + w, ..., v_m + w$ is linearly independent then it is a basis of $U$ and $\text{dim }U = m$.
\\
\\
Otherwise the list is linearly dependent and some element at the $j$th position in the list is a linear combination of the previous elements in the list:
\begin{align*}
  v_j + w &= a_1(v_1 + w) + ... + a_{j-1}(v_{j-1} + w) \\
  v_j + w &= a_1v_1 + a_1w + ... + a_{j-1}v_{j-1} + a_{j-1}w \\
  v_j + w &= a_1v_1 + ... + a_{j-1}v_{j-1} + (a_1 + ... + a_{j-1})w \\
 w - (a_1 + ... + a_{j-1})w &= a_1v_1 + ... + a_{j-1}v_{j-1} - v_j \\
 (1 - a_1 - ... - a_{j-1})w &= a_1v_1 + ... + a_{j-1}v_{j-1} - v_j
\intertext{
The right side of the equation is a linear combination of the list $v_1$, ..., $v_j$.
This list is linearly independent and not all of the scalars are zero (the coefficient of $v_j$ is $-1$).
Hence the right side of the equation is non-zero.
Thus the coefficient of $w$ on the left side is non-zero.
Call it $k$:
}
 kw &= a_1v_1 + ... + a_{j-1}v_{j-1} - v_j \\
 w &= \frac{a_1}{k}v_1 + ... + \frac{a_{j-1}}{k}v_{j-1} - \frac{1}{k}v_j
\end{align*}
Thus $w$ can be expressed as a linear combination of the list $v_1$, ..., $v_m$ by taking the above coefficients for the vectors $v_1$, ..., $v_j$ and zero for the remaining coefficients.
Furthermore, because the list $v_1$, ..., $v_m$ is linearly independent, it is a basis of the subspace it spans, so this representation is unique.
\\
\\
Now say we remove the $j$th element from the list $v_1 + w$, ..., $v_m + w$.
The resulting list must be linearly independent.
If it was linearly dependent, we could follow the same steps above and express $w$ as a linear combination of $v_1$, ..., $v_{j-1}$, $v_{j+1}$, ..., $v_m$.
Then we could add $0v_j$ to express $w$ as a linear combination of $v_1$, ..., $v_m$.
This cannot be possible because the unique representation of $w$ as a linear combination of $v_1$, ..., $v_m$ has $\frac{-1}{k}$ as a coefficient for $v_j$, not zero.
\\
\\
Therefore we can reduce $v_1 + w$, ..., $v_m + w$ to a linearly independent list (and a basis of $U$) by removing at most one element.
Hence dim $U$ is either $m$ or $m - 1$.
\end{document}
 
