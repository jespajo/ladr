\documentclass[a5paper]{article}
\usepackage{amsmath}
\usepackage{amssymb}
\usepackage[top=1cm,right=1cm,bottom=1cm,left=1cm]{geometry}
\setlength\parindent{0pt}
\setlength\parskip{1em}
%
%\usepackage{xcolor}
%\pagecolor[rgb]{0.1,0.1,0.1}
%\color[rgb]{1.0,1.0,1.0}
%
\begin{document}
\newcommand   \C           {\mathbf{C}}
\newcommand   \R           {\mathbf{R}}
\renewcommand \L           {\mathcal{L}}
\newcommand   \F           {\mathbf{F}}
\renewcommand \P           {\mathcal{P}}
\newcommand   \M           {\mathcal{M}}
\newcommand   \op          {\operatorname}

    3.B.30.
    Suppose $\varphi \in \L(V,\F)$ and $\varphi \neq 0$.
    Suppose $u \in V$ is not in $\op{null}\varphi$.
    Prove that
\begin{equation*}
        V = \op{null}\varphi \oplus \{au : a \in \F\} .
\end{equation*}

    Suppose $v \in V$.

    Since $u \notin \op{null}\varphi$, there exists $b \in \F$ with $b \neq 0$ such that
\begin{equation*}
        \varphi u = b .
\end{equation*}
    There also exists $c \in \F$ (with $c$ possibly being zero) such that
\begin{equation*}
        \varphi v = c .
\end{equation*}

    Consider the vector $v - \frac{c}{b} u \in V$.
    Applying $\varphi$,
\begin{align*}
        \varphi \left( v - \frac{c}{b} u \right) &= \varphi v - \frac{c}{b} \varphi u \\
                                                &= c - \frac{c}{b} b \\
                                                &= c - c \\
                                                &= 0 .
\end{align*}
    Thus $v - \frac{c}{b} u \in \op{null}\varphi$.

    So we can write $v$ as the sum of a vector in $\op{null}\varphi$ and a vector in $\{au: a \in \F\}$:
\begin{equation*}
        v = \left( v - \frac{c}{b} u \right) + \frac{c}{b} u .
\end{equation*}
    Hence $V \subseteq \op{null}\varphi + \{au : a \in \F\}$.

    To show inclusion in the other direction, both $\op{null}\varphi$ and $\{au : a \in \F\}$ are subspaces of $V$, so the sum of these subspaces is the smallest subspace of $V$ containing both subspaces.

    Thus $V = \op{null}\varphi + \{au : a \in \F\}$.

    To show that this is a direct sum, suppose that $w \in \op{null}\varphi$ and $du \in \{au : a \in \F\}$ such that
\begin{equation}
        0 = w + du .
\end{equation}
    Then
\begin{equation*}
        -w = du .
\end{equation*}
    Applying $\varphi$ to both sides,
\begin{equation*}
         -\varphi w = d \varphi u .
\end{equation*}
    The left side of the equation is zero because $w \in \op{null}\varphi$.
    On the right side, $\varphi u$ is nonzero because $u \notin \op{null}\varphi$.
    Thus $d = 0$.

    So (1) becomes
\begin{equation*}
        0 = w + 0u = w.
\end{equation*}

    Hence the only choice of vectors from each subspace that sum to zero is to take zero from each subspace.
    Thus
\begin{equation*}
        V = \op{null}\varphi \oplus \{au : a \in \F\} .
\end{equation*}
\end{document}
