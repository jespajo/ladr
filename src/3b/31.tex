\documentclass[a5paper]{article}
\usepackage{amsmath}
\usepackage{amssymb}
\usepackage[top=1cm,right=1cm,bottom=1cm,left=1cm]{geometry}
\setlength\parindent{0pt}
\setlength\parskip{1em}
%
%\usepackage{xcolor}
%\pagecolor[rgb]{0.1,0.1,0.1}
%\color[rgb]{1.0,1.0,1.0}
%
\begin{document}
\newcommand   \C           {\mathbf{C}}
\newcommand   \R           {\mathbf{R}}
\renewcommand \L           {\mathcal{L}}
\newcommand   \F           {\mathbf{F}}
\renewcommand \P           {\mathcal{P}}
\newcommand   \M           {\mathcal{M}}
\newcommand   \op          {\operatorname}

    3.B.31.
    Suppose $V$ is finite-dimensional, $X$ is a subspace of $V$, and $Y$ is a finite-dimensional subspace of $W$.
    Prove that there exists $T \in \L(V,W)$ such that $\op{null}T = X$ and $\op{range}T = Y$ if and only if $\op{dim}X + \op{dim}Y = \op{dim}V$.

    To prove in one direction, suppose there exists $T \in \L(V,W)$ such that $\op{null}T = X$ and $\op{range}T = Y$.
    Then, from the Fundamental Theorem of Linear Maps,
\begin{align*}
        \op{dim}V &= \op{dim}\op{null}T + \op{dim}\op{range}T \\
                  &= \op{dim}X + \op{dim}Y .
\end{align*}

    To prove in the other direction, suppose $\op{dim}X + \op{dim}Y = \op{dim}V$.
    Suppose $x_1,\dots,x_m$ is a basis of $X$ and $y_1,\dots,y_n$ is a basis of $Y$.
    Thus $\op{dim}V = m + n$.

    Since $X$ is a subspace of $V$, we can extend our basis of $X$ to a basis of $V$.
    Since $\op{dim}V = m + n$, this process involves adding an extra $n$ vectors to the list.
    Denote the extra vectors $v_1,\dots,v_n$.
    In other words,
\begin{equation*}
        x_1,\dots,x_m,v_1,\dots,v_n
\end{equation*}
    is a basis of $V$.

    Define $T \in \L(V,W)$ in terms of this basis of $V$ such that
\begin{align*}
        Tx_j&=0&    \text{for}\ j&=1,\dots,m&\text{and} \\
        Tv_k&=y_k&  \text{for}\ k&=1,\dots,n.&
\end{align*}

    We will complete our proof by showing that $\op{null}T = X$ and $\op{range}T = Y$.

    To prove $X \subseteq \op{null}T$, suppose $x \in X$.
    Then there exist $a_1,\dots,a_m \in \F$ such that 
\begin{align*}
        Tx &= T(a_1x_1 + \dots + a_mx_m) \\
           &= a_1Tx_1 + \dots + a_mTx_m  \\
           &= 0 .
\end{align*}
    So $x \in \op{null}T$.

    To prove $\op{null}T \subseteq X$, suppose $u \in \op{null}T$.
    Then since $u \in V$, there exist $b_1,\dots,b_{m+n}$ such that
\begin{equation*}
        u = b_1x_1 + \dots + b_mx_m + b_{m+1}v_1 + \dots + b_{m+n}v_n .
\end{equation*}
    Since $u \in \op{null}T$,
\begin{align*}
        0 &= Tu \\
          &= T(b_1x_1 + \dots + b_mx_m + b_{m+1}v_1 + \dots + b_{m+n}v_n) \\
          &= b_1Tx_1 + \dots + b_mTx_m + b_{m+1}Tv_1 + \dots + b_{m+n}Tv_n \\
          &= b_{m+1}y_1 + \dots + b_{m+n}y_n .
\end{align*}
    The $b_jTx_j$ terms disappear because each $Tx_j = 0$.
    We are left with a linear combination of $y_1,\dots,y_n$ equal to zero.
    Since the $y$'s are linearly independent, this implies $b_{m+1}=\dots=b_{m+n}=0$.
    Thus
\begin{equation*}
        u = b_1x_1 + \dots + b_mx_m .
\end{equation*}
    So $u \in X$.

    Thus $\op{null}T = X$.

    To prove $Y \subseteq \op{range}T$, suppose $y \in Y$.
    Then there exist $c_1,\dots,c_n \in \F$ such that
\begin{align*}
        y &= c_1y_1 + \dots + c_ny_n \\
          &= c_1Tv_1 + \dots + c_nTv_n \\
          &= T(c_1v_1 + \dots + c_nv_n) .
\end{align*}
    Thus $y \in \op{range}T$.

    To prove $\op{range}T \subseteq Y$, suppose $w \in \op{range}T$.
    Then there exists $v \in V$ such that $Tv = w$.
    Since $v \in V$, there exist $d_1,\dots,d_{m+n} \in \F$ such that
\begin{align*}
        w &= Tv \\
          &= T(d_1x_1 + \dots + d_mx_m + d_{m+1}v_1 + \dots + d_{m+n}v_n) \\
          &= d_1Tx_1 + \dots + d_mTx_m + d_{m+1}Tv_1 + \dots + d_{m+n}Tv_n \\
          &= d_{m+1}y_1 + \dots + d_{m+n}y_n .
\end{align*}
    Thus $w \in Y$.

    Thus $\op{range}T = Y$.
\end{document}
