\documentclass[a4paper]{article}
\usepackage{amsmath}
\usepackage[top=1cm,right=1cm,bottom=1cm,left=1cm]{geometry}
\setlength\parindent{0pt}
\setlength\parskip{1em}
%
%\usepackage{xcolor}
%\pagecolor[rgb]{0.1,0.1,0.1}
%\color[rgb]{1.0,1.0,1.0}
%
\begin{document}

\newcommand    \C  { \mathbf{C} }
\newcommand    \R  { \mathbf{R} }
\renewcommand  \L  { \mathcal{L} }
\newcommand    \F  { \mathbf{F} }
\newcommand    \nullspace { \text{null\;} }
\newcommand    \range     { \text{range\;} }

By the Fundamental Theorem of Linear Maps, for any $T \in \L(\R^5, W)$
\begin{equation*}
    \dim\nullspace T + \dim\range T  = \dim \R^5 = 5
\end{equation*}
The two dimensions on the left side of the equation are nonnegative integers.
The sum of these integers is an odd number.
Hence one integer must be even and the other odd.
Thus
\begin{align*}
    \dim\nullspace T &\neq \dim\range T             \\
        \nullspace T &\neq \range T
\end{align*}
\end{document}
