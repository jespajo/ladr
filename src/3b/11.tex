\documentclass[a5paper]{article}
\usepackage{amsmath}
\usepackage[top=1cm,right=1cm,bottom=1cm,left=1cm]{geometry}
\usepackage[pdfstartview=FitH]{hyperref}
\setlength\parindent{0pt}
\setlength\parskip{1em}
%\usepackage{xcolor}
%\pagecolor[rgb]{0.1,0.1,0.1}
%\color[rgb]{1.0,1.0,1.0}
\begin{document}
\newcommand    \C          {\mathbf{C}}
\newcommand    \R          {\mathbf{R}}
\renewcommand  \L          {\mathcal{L}}
\newcommand    \F          {\mathbf{F}}
\renewcommand  \P          {\mathcal{P}}
\newcommand    \nullspace  {\text{null\;}}
\newcommand    \range      {\text{range\;}}
\newcommand    \linspan    {\text{span\;}}

    $V$ is finite-dimensional and $T \in \L(V,W)$.

    $\nullspace T$ is a subspace of $V$.
    Hence there is a subspace $U$ of $V$ such that
\begin{align*}
        V &= \nullspace T \oplus U .
\end{align*}
    Since this is a direct sum, $\nullspace T \,\cap\, U = \{0\}$.
    So we only need to prove that $\range T = \{ Tu : u \in U \}$.

    $U$ and $\nullspace T$ are subspaces of $V$, so they both have bases.
    Suppose $u_1,\dots,u_m$ is a basis of $U$ and $w_1,\dots,w_n$ is a basis of $\nullspace T$.

    Then $u_1,\dots,u_m,w_1,\dots,w_n$ is a basis of $V$, a result we proved in 2.B.8.

    So for every $v \in V$ there is some $a_1,\dots,a_m,b_1,\dots,b_n \in \F$ such that
\begin{align*}
         v &= a_1u_1 + \dots + a_mu_m + b_1w_1 + \dots + b_nw_n                 \\
                                                                                \\
        Tv &= T(a_1u_1 + \dots + a_mu_m + b_1w_1 + \dots + b_nw_n)              \\
           &= a_1(Tu_1) + \dots + a_m(Tu_m) + b_1(Tw_1) + \dots + b_n(Tw_n)     \\
           &= a_1(Tu_1) + \dots + a_m(Tu_m)                                     \\
           &= T(a_1u_1 + \dots + a_mu_m)
\end{align*}
    Thus $\range T = \{ Tu : u \in U \}$, as desired.
\end{document}
