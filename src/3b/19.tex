\documentclass[a5paper]{article}
\usepackage{amsmath}
\usepackage[top=1cm,right=1cm,bottom=1cm,left=1cm]{geometry}
\setlength\parindent{0pt}
\setlength\parskip{1em}
%\usepackage{xcolor}
%\pagecolor[rgb]{0.1,0.1,0.1}
%\color[rgb]{1.0,1.0,1.0}
\begin{document}
\newcommand   \C           {\mathbf{C}}
\newcommand   \R           {\mathbf{R}}
\renewcommand \L           {\mathcal{L}}
\newcommand   \F           {\mathbf{F}}
\renewcommand \P           {\mathcal{P}}
\newcommand   \nullspace   {\text{null\:}}
\newcommand   \range       {\text{range\:}}
\newcommand   \linspan     {\text{span\:}}
\newcommand   \question[1] {\textbf{\boldmath#1\unboldmath}\par}

\question{
    3.B.20.
    Suppose $W$ is finite-dimensional and that $T \in \L(V,W)$.
    Prove that $T$ is injective if and only if there exists $S \in \L(W,V)$ such that $ST$ is the identity map on $V$.
}
    To prove in one direction, suppose $S \in \L(W,V)$ exists such that $ST$ is the identity map on $V$.

    Suppose $v \in V$ such that $Tv = 0$.
    Then
\begin{equation*}
        (ST)(v) = S(Tv) = S(0) = 0 .
\end{equation*}
    So the identity map on $V$ takes $v$ to 0.
    Thus $v=0$.
    Hence $T$ is injective.

    To prove in the other direction, suppose $T$ is injective.

    $\range T$ has a basis because it is a subspace of $W$.
    Denote this basis $w_1,\dots,w_m$.
    So $\dim \range T = m$.

    Clearly there exist $v_1,\dots,v_m \in V$ such that $Tv_j = w_j$ for each $j = 1,\dots,m$.
    In fact $v_1,\dots,v_m$ is a basis of $V$.

    To show this, suppose $a_1,\dots,a_m \in \F$ such that
\begin{align*}
        a_1v_1 + \dots + a_mv_m &= 0 .
\intertext{
    Applying $T$ to both sides of the equation,
}
        a_1Tv_1 + \dots + a_mTv_m &= T(0) \\
          a_1w_1 + \dots + a_mw_m &= 0 .
\end{align*}
    Then $a_1=\dots=a_m=0$ since $w_1,\dots,w_m$ is linearly independent.
    Hence $v_1,\dots,v_m$ is also linearly independent.

    So we can prove that $v_1,\dots,v_m$ is a basis of $V$ by proving that $V$ is finite-dimensional and that $v_1,\dots,v_m$ has has length equal to $\dim V$.

    No linear map to a smaller-dimensional subspace is injective, so the fact that $T$ is injective means $\dim V \le \dim W$.
    Thus $V$ is finite-dimensional.

    Hence we can use the Fundamental Theorem of Linear Maps to show
\begin{align*}
        \dim V &= \dim \range T + \dim \nullspace T \\
               &= \dim \range T + \dim \{ 0 \} \\
               &= \dim \range T \\
               &= m .
\end{align*}
    So $v_1,\dots,v_m$ has length equal to $\dim V$.
    Hence this list is a basis of $V$.

    Thus every $v \in V$ can be written as
\begin{equation*}
        v = b_1v_1 + \dots + b_mv_m
\end{equation*}
    with $b_1,\dots,b_m \in \F$.

    We can extend $w_1,\dots,w_m$ to a basis of $W$ by appending $w_{m+1},\dots,w_{m+n} \in W$.

    Then there exists $S \in \L(W,V)$ such that
\begin{align*}
        Sw_k =
            \begin{cases}
                \ v_k \ &\text{if}\quad k \le m \\
                \ 0   \ &\text{if}\quad k  >  m
            \end{cases}
\end{align*}
    for each $k = 1,\dots,m+n$.
    (The second case doesn't matter and 0 could be any vector in $V$.)

    Hence
\begin{align*}
        (ST)(v) &= S(Tv) \\
                &= S \big( T(b_1v_1 + \dots + b_mv_m) \big) \\
                &= S (b_1Tv_1 + \dots + b_mTv_m ) \\
                &= S(b_1w_1 + \dots + b_mw_m) \\
                &= b_1Sw_1 + \dots + b_mSw_m \\
                &= b_1v_1 + \dots + b_mv_m \\
                &= v .
\end{align*}
    Thus $ST$ is the identity map on $V$.
\end{document}
