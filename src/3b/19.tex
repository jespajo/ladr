\documentclass[a5paper]{article}
\usepackage{amsmath}
\usepackage[top=1cm,right=1cm,bottom=1cm,left=1cm]{geometry}
\setlength\parindent{0pt}
\setlength\parskip{1em}
%\usepackage{xcolor}
%\pagecolor[rgb]{0.1,0.1,0.1}
%\color[rgb]{1.0,1.0,1.0}
\begin{document}
\newcommand   \C           {\mathbf{C}}
\newcommand   \R           {\mathbf{R}}
\renewcommand \L           {\mathcal{L}}
\newcommand   \F           {\mathbf{F}}
\renewcommand \P           {\mathcal{P}}
\newcommand   \nullspace   {\text{null\,}}
\newcommand   \range       {\text{range\,}}
\newcommand   \linspan     {\text{span\,}}
\newcommand   \question[1] {\textbf{\boldmath#1\unboldmath}\par}

\question{
    3.B.19. Suppose $V$ and $W$ are finite-dimensional and that $U$ is a subspace of $V$.
    Prove that there exists $T \in \L(V,W)$ such that $\nullspace T = U$ if and only if $\dim U \ge \dim V - \dim W$.
}
    To prove in one direction, suppose $T \in \L(V,W)$ exists such that $\nullspace T = U$.
    Then applying the Fundamental Theorem of Linear Maps,
\begin{align*}
        \dim U &= \dim \nullspace T \\
            &= \dim V - \dim \range T .
\intertext{
    Since $\dim W \ge \dim \range T$, if we replace $\dim \range T$ in the equation above with $\dim W$, the right side of the equation gets smaller or stays the same.
    Thus
}
        \dim U &\ge \dim V - \dim W .
\end{align*}
    To prove in the other direction, suppose $\dim U \ge \dim V - \dim W$.

    Let $u_1,\dots,u_m$ be a basis of $U$.
    Extend this list to a basis of $V$ by appending $v_1,\dots,v_n \in V$.

    Then $\dim U = m$ and $\dim V = m + n$.
    So
\begin{align*}
             m &\ge m + n - \dim W \\
             0 &\ge n - \dim W \\
        \dim W &\ge n .
\end{align*}
    Hence there exists a list of length $n$ of linearly independent vectors in $W$.
    Denote this list $w_1,\dots,w_n$.

    Since $u_1,\dots,u_m, v_1,\dots,v_n$ is a basis of $V$, there exists $T \in \L(V,W)$ such that
\begin{alignat*}{3}
        Tu_j\,&=\,0  \quad &&\text{for}\quad j\;& =&\;1,\dots,m\quad\text{and} \\
        Tv_k\,&=\,w_k\quad &&\text{for}\quad k\;& =&\;1,\dots,n .
\end{alignat*}
    So we need to prove that $\nullspace T = U$.
    We do so by showing $U \subset \nullspace T$ and then $\nullspace T \subset U$.
    
    First suppose $u \in U$.
    Then we can write $u$ as
\begin{align*}
        u &= a_1u_1 + \dots + a_mu_m
\intertext{
    with $a_1,\dots,a_m \in \F$.
    So
}
        Tu &= T(a_1u_1 + \dots + a_mu_m) \\
           &= a_1(Tu_1) + \dots + a_m(Tu_m) \\
           &= a_1(0) + \dots + a_m(0) \\
           &= 0 .
\end{align*}
    Thus $u$ is in the null space of $T$.
    Hence $U \subset \nullspace T$.

    Now to show $\nullspace T \subset U$.

    Suppose $v \in V$ such that $Tv = 0$.
    We can write $v$ as
\begin{equation*}
        v = b_1u_1+\dots+b_mu_m + c_1v_1+\dots+c_nv_n
\end{equation*}
    with $b_1,\dots,b_m,c_1,\dots,c_n \in \F$. So
\begin{align*}
        T(b_1u_1+\dots+b_mu_m + c_1v_1+\dots+c_nv_n) &= 0 \\
        T(b_1u_1+\dots+b_mu_m) + T(c_1v_1+\dots+c_nv_n) &= 0 \\
        T(c_1v_1+\dots+c_nv_n) &= -T(b_1u_1+\dots+b_mu_m) \\
        c_1(Tv_1) +\dots+ c_n(Tv_n) &= -b_1(Tu_1) -\dots- b_m(Tu_m) \\
        c_1w_1 +\dots+ c_nw_n &= -b_1(0) -\dots- b_m(0) \\
        c_1w_1 +\dots+ c_nw_n &= 0 .
\end{align*}
    Then since $w_1,\dots,w_n$ is linearly independent, $c_1=\dots=c_n=0$.
    Hence
\begin{align*}
        v &= b_1u_1+\dots+b_mu_m + 0v_1+\dots+0v_n \\
          &= b_1u_1+\dots+b_mu_m
\end{align*}
    Thus $v \in U$ because $U = \linspan(u_1,\dots,u_m)$.

    Hence $\nullspace T = U$.
\end{document}
