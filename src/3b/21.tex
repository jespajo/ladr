\documentclass[a5paper]{article}
\usepackage{amsmath}
\usepackage[top=1cm,right=1cm,bottom=1cm,left=1cm]{geometry}
\setlength\parindent{0pt}
\setlength\parskip{1em}
%
%\usepackage{xcolor}
%\pagecolor[rgb]{0.1,0.1,0.1}
%\color[rgb]{1.0,1.0,1.0}
%
\begin{document}
\newcommand   \C           {\mathbf{C}}
\newcommand   \R           {\mathbf{R}}
\renewcommand \L           {\mathcal{L}}
\newcommand   \F           {\mathbf{F}}
\renewcommand \P           {\mathcal{P}}
\newcommand   \M           {\mathcal{M}}
\newcommand   \op          {\operatorname}

    3.B.21.
    Suppose $V$ is finite-dimensional, $T \in \L(V,W)$, and $U$ is a subspace of $W$.
    Prove that $\{v \in V : Tv \in U\}$ is a subspace of $V$ and
\begin{equation*}
        \op{dim}\{v \in V : Tv \in U\} = \op{dim}\op{null}T + \op{dim}(U \cap \op{range}T) .
\end{equation*}

    In this proof, we sometimes refer to $\{v \in V : Tv \in U\}$ as ``our set''.

    Our set contains 0 because $T(0)=0$.
    The 0 on the right side is an element of $U$ because $U$ is a subspace of $W$.
    
    To see that our set is closed under addition, suppose $u,w \in \{v \in V : Tv \in U\}$.
    Then $Tu,Tw \in U$.
    Hence $Tu+Tw \in U$ because $U$ is closed under addition.
    $T(u+w)=Tu+Tw$ so $u+w$ is an element of our set.

    To see that our set is closed under scalar multiplication, suppose $u \in \{v \in V : Tv \in U\}$ and $a \in \F$.
    Then $Tu \in U$.
    Hence $aTu \in U$ because $U$ is closed under scalar multiplication.
    $T(au)=aTu$ so $au$ is an element of our set.

    Hence our set is a subspace of $V$.

    $\op{null}T$ is a subspace of our set.
    To see this, suppose $u \in \op{null}T$.
    Then $Tu = 0$.
    This implies $u$ is in our set because $0 \in U$.
    Thus $\op{null}T \subseteq \{v \in V : Tv \in U\}$.

    Suppose $u_1,\dots,u_m$ is a basis of $\op{null}T$.
    Suppose we extend this list to a basis of our set by appending $v_1,\dots,v_n$.
    In other words,
\begin{equation*}
        u_1,\dots,u_m,v_1,\dots,v_n
\end{equation*}
    is a basis of $\{v \in V : Tv \in U\}$.

    Our task then is to show that
\begin{align*}
        \op{dim}\{v \in V : Tv \in U\} &= \op{dim}\op{null}T + \op{dim}(U \cap \op{range}T) \\
                                 m + n &= m + \op{dim}(U \cap \op{range}T) \\
                                     n &= \op{dim}(U \cap \op{range}T) .
\end{align*}
    Suppose $w \in U \cap \op{range}T$.
    Because $w \in \op{range}T$, there exists $u \in V$ such that $Tu = w$.
    $u$ must be in our set because $w \in U$.
    So there exist $a_1,\dots,a_m,b_1,\dots,b_n \in \F$ such that
\begin{align*}
         w &= Tu \\
           &= T(a_1u_1 + \dots + a_mu_m + b_1v_1 + \dots + b_nv_n) \\
           &= a_1Tu_1 + \dots + a_mTu_m + b_1Tv_1 + \dots + b_nTv_n \\
           &= b_1Tv_1 + \dots + b_nTv_n .
\end{align*}
    Hence the list $Tv_1,\dots,Tv_n$ spans $U \cap \op{range}T$.
    (The $a_jTu_j$ terms disappeared because each $u_j \in \op{null}T$.)

    Now suppose $c_1,\dots,c_n \in \F$ such that $c_1Tv_1 + \dots + c_nTv_n = 0$.
    Then
\begin{equation*}
        T(c_1v_1 + \dots + c_nv_n) = 0.
\end{equation*}
    Thus $c_1v_1+\dots+c_nv_n \in \op{null}T$.
    So there exist $d_1,\dots,d_m \in \F$ such that
\begin{align*}
        d_1u_1 + \dots + d_mu_m &= c_1v_1 + \dots + c_nv_n \\
        0 &= c_1v_1 + \dots + c_nv_n - d_1u_1 - \dots - d_mu_m .
\end{align*}
    Here is a linear combination of a linearly independent list equal to 0, so $c_1=\dots=c_n=d_1=\dots=d_m=0$.
    Thus $Tv_1,\dots,Tv_n$ is linearly independent.

    Hence $Tv_1,\dots,Tv_n$ is a basis of $U \cap \op{range}T$.
    So $\op{dim}(U \cap \op{range}T) = n$, as desired.
\end{document}
