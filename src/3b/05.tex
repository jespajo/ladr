\documentclass[a5paper]{article}
\usepackage{amsmath}
\usepackage[top=1cm,right=1cm,bottom=1cm,left=1cm]{geometry}
\setlength\parindent{0pt}
\setlength\parskip{1em}
%
%\usepackage{xcolor}
%\pagecolor[rgb]{0.1,0.1,0.1}
%\color[rgb]{1.0,1.0,1.0}
%
\begin{document}

\newcommand    \C  { \mathbf{C} }
\newcommand    \R  { \mathbf{R} }
\renewcommand  \L  { \mathcal{L} }
\newcommand    \F  { \mathbf{F} }
\newcommand    \nullspace { \text{null\;} }
\newcommand    \range     { \text{range\;} }

Suppose $T \in \L(\R^4, \R^4)$ is defined by
\begin{equation*}
    T(x_1,x_2,x_3,x_4) = (x_3, x_4, 0, 0)
\end{equation*}
Then
\begin{align*}
    \nullspace T &= \left\{ (x_1, x_2, 0, 0) \in \R^4 : x_1, x_2 \in \R \right\}    \\
        \range T &= \left\{ (x_3, x_4, 0, 0) \in \R^4 : x_3, x_4 \in \R \right\}
\end{align*}
Thus
\begin{align*}
    \nullspace T &= \range T
\end{align*}
\end{document}
