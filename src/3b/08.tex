\documentclass[a5paper]{article}
\usepackage{amsmath}
\usepackage[top=1cm,right=1cm,bottom=1cm,left=1cm]{geometry}
\setlength\parindent{0pt}
\setlength\parskip{1em}
% 
% \usepackage{xcolor}
% \pagecolor[rgb]{0.1,0.1,0.1}
% \color[rgb]{1.0,1.0,1.0}
% 
\begin{document}

\newcommand    \C  { \mathbf{C} }
\newcommand    \R  { \mathbf{R} }
\renewcommand  \L  { \mathcal{L} }
\newcommand    \F  { \mathbf{F} }
\newcommand    \nullspace { \text{null\;} }
\newcommand    \range     { \text{range\;} }

$V$ and $W$ are finite-dimensional with $\dim V \ge \dim W \ge 2$.

Suppose $v_1,\dots,v_n$ is a basis of $V$ and $w_1,\dots,w_m$ is a basis of $W$.
Thus $n \ge m$.

Define $S_1,S_2 \in \L(V,W)$ such that
\begin{align*}
  \begin{split}
    S_1v_j &=
        \begin{cases}
            w_1  &\text{if } j = 1                      \\
            0    &\text{if } j \in \{2,\dots,n\}
        \end{cases}
  \end{split}
  \begin{split}
    S_2v_j &=
        \begin{cases}
            0    &\text{if } j = 1                      \\
            w_j  &\text{if } j \in \{2,\dots,m\}        \\
            0    &\text{if } j \in \{m+1,\dots,n\}
        \end{cases}
  \end{split}
\end{align*}

Every $v \in V$ can be written with $a_1,\dots,a_n\in\F$ as
\begin{align*}
    v    &= a_1v_1 + \dots + a_nv_n
\intertext{
So
}
    S_1v &= S_1(a_1v_1 + \dots + a_nv_n)            \\
         &= S_1(a_1v_1) + \dots + S_1(a_nv_n)       \\
         &= a_1(S_1v_1) + \dots + a_n(S_1v_n)       \\
         &= a_1w_1                                  \\
                                                    \\
    S_2v &= S_2(a_1v_1 + \dots + a_nv_n)            \\
         &= S_2(a_1v_1) + \dots + S_2(a_nv_n)       \\
         &= a_1(S_2v_1) + \dots + a_n(S_2v_n)       \\
         &= a_2w_2 + \dots + a_mw_m
\end{align*}
$w_1,\dots,w_m$ is a linearly independent list of at least two vectors, so $w_2$ exists and cannot be written as a scalar multiple of $w_1$.
Hence $w_2 \notin \range S_1$.
Similarly $w_1$ exists and cannot be written as a linear combination of $w_2,\dots,w_m$.
Hence $w_1 \notin \range S_2$.
Thus neither $S_1$ nor $S_2$ is surjective.

But
\begin{align*}
    (S_1 + S_2)(v) &= S_1v + S_2v                       \\
                   &= a_1w_1 + \dots + a_mw_m
\end{align*}
Thus $\range (S_1 + S_2) = \text{span\;}(w_1,\dots,w_m) = W$.
So $S_1 + S_2$ is surjective.

Hence $\left\{ T \in \L(V,W) : T \text{ is not surjective} \right\}$ is not a subspace because it is not closed under addition.
\end{document}
