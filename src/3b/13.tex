\documentclass[a5paper]{article}
\usepackage{amsmath}
\usepackage[top=1cm,right=1cm,bottom=1cm,left=1cm]{geometry}
\usepackage[pdfstartview=FitH]{hyperref}
\setlength\parindent{0pt}
\setlength\parskip{1em}
%\usepackage{xcolor}
%\pagecolor[rgb]{0.1,0.1,0.1}
%\color[rgb]{1.0,1.0,1.0}
\begin{document}
\newcommand    \C          {\mathbf{C}}
\newcommand    \R          {\mathbf{R}}
\renewcommand  \L          {\mathcal{L}}
\newcommand    \F          {\mathbf{F}}
\renewcommand  \P          {\mathcal{P}}
\newcommand    \nullspace  {\text{null\;}}
\newcommand    \range      {\text{range\;}}
\newcommand    \linspan    {\text{span\;}}

    We have a linear map $T : \F^4 \rightarrow \F^2$ such that
\begin{align*}
        \nullspace T = \left\{ (x_1,x_2,x_3,x_4) \in \F^4 : x_1 = 5x_2 \text{ and } x_3 = 7x_4 \right\}
\end{align*}
    A basis of $\nullspace T$ is $(5,1,0,0),(0,0,7,1)$.
    Thus $\dim \nullspace T = 2$.

    From the Fundamental Theorem of Linear Maps,
\begin{align*}
           \dim \F^4 &= \dim\nullspace T + \dim\range T     \\
                   4 &= 2 + \dim\range T                    \\
        \dim\range T &= 2
\intertext{
    $\range T$ is therefore a subspace of $\F^2$ with the same dimension as $\F^2$.
    Hence
}
                               \range T  &=  \F^2
\end{align*}
    Thus $T$ is surjective.
\end{document}
