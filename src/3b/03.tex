\documentclass[a5paper]{article}
\usepackage{amsmath}
\usepackage[top=1cm,right=1cm,bottom=1cm,left=1cm]{geometry}
\setlength\parindent{0pt}
\setlength\parskip{1em}
%
%\usepackage{xcolor}
%\pagecolor[rgb]{0.1,0.1,0.1}
%\color[rgb]{1.0,1.0,1.0}
%
\begin{document}

\newcommand    \C  { \mathbf{C} }
\newcommand    \R  { \mathbf{R} }
\renewcommand  \L  { \mathcal{L} }
\newcommand    \F  { \mathbf{F} }
\renewcommand \P           {\mathcal{P}}
\newcommand   \question[1] {\textbf{\boldmath#1\unboldmath}\par}
\newcommand   \op          {\operatorname}

\question{
    Suppose $v_1,\dots,v_n$ is a list of vectors in $V$.
    Define $T \in \L(\F^m, V)$ by
}
\begin{equation*}
        T(z_1,\dots,z_m) = z_1v_1 + \dots + z_mv_m .
\end{equation*}

\question{(a) What property of $T$ corresponds to $v_1,\dots,v_n$ spanning $V$?}

Surjectivity of $T$ implies $v_1,\dots,v_m$ spans $V$.
If $T$ is surjective, $\text{range\;}T = V$.
Hence every vector in $V$ can be written as a linear combination of $v_1,\dots,v_m$.

\question{(b) What property of $T$ corresponds to the list $v_1,\dots,v_n$ being linearly independent?}

Injectivity of $T$ implies $v_1,\dots,v_m$ is linearly independent.
If $T$ is injective, then $T(z_1,\dots,z_m)=0$ implies $z_1 = \dots = z_m = 0$.
Hence the only linear combination of $v_1,\dots,v_m$ equal to 0 has all the $v$'s multiplied by 0.
\end{document}
