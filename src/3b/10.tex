\documentclass[a5paper]{article}
\usepackage{amsmath}
\usepackage[top=1cm,right=1cm,bottom=1cm,left=1cm]{geometry}
\setlength\parindent{0pt}
\setlength\parskip{1em}
% 
% \usepackage{xcolor}
% \pagecolor[rgb]{0.1,0.1,0.1}
% \color[rgb]{1.0,1.0,1.0}
% 
\begin{document}

\newcommand    \C  { \mathbf{C} }
\newcommand    \R  { \mathbf{R} }
\renewcommand  \L  { \mathcal{L} }
\newcommand    \F  { \mathbf{F} }
\renewcommand  \P  { \mathcal{P} }
\newcommand    \nullspace { \text{null\;} }
\newcommand    \range     { \text{range\;} }
\newcommand    \linspan   { \text{span\;} }

We have $T \in \L(V,W)$ and $v_1,\dots,v_n$ spans $V$.

Every $v \in V$ can be written 
\begin{align*}
    v &= a_1v_1 + \dots + a_nv_n
\intertext{
with $a_1,\dots,a_n \in \F$.
Hence
}
    Tv &= T(a_1v_1 + \dots + a_nv_n)        \\
       &= T(a_1v_1) + \dots + T(a_nv_n)     \\
       &= a_1(Tv_1) + \dots + a_n(Tv_n)
\end{align*}
Thus $\range T = \linspan(Tv_1,\dots,Tv_n)$.
\end{document}
