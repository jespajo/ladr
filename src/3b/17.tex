\documentclass[a5paper]{article}
\usepackage{amsmath}
\usepackage[top=1cm,right=1cm,bottom=1cm,left=1cm]{geometry}
\usepackage[pdfstartview=FitH]{hyperref}
\setlength\parindent{0pt}
\setlength\parskip{1em}
%\usepackage{xcolor}
%\pagecolor[rgb]{0.1,0.1,0.1}
%\color[rgb]{1.0,1.0,1.0}
\begin{document}
\newcommand    \C          {\mathbf{C}}
\newcommand    \R          {\mathbf{R}}
\renewcommand  \L          {\mathcal{L}}
\newcommand    \F          {\mathbf{F}}
\renewcommand  \P          {\mathcal{P}}
\newcommand    \nullspace  {\text{null\;}}
\newcommand    \range      {\text{range\;}}
\newcommand    \linspan    {\text{span\;}}

    $V$ and $W$ are finite-dimensional.

    Let $v_1,\dots,v_m$ be a basis of $V$ and $w_1,\dots,w_n$ be a basis of $W$.
    Thus $\dim V = m$ and $\dim W = n$.

    First suppose there exists a surjective linear map from $V$ to $W$.
    No linear map to a larger-dimensional space is surjective so $\dim V \ge \dim W$.

    To prove the other direction, suppose $\dim V \ge \dim W$.
    Then $m \ge n$.

    Hence there exists $T \in \L(V,W)$ such that
\begin{align*}
        Tv_j =
            \begin{cases}
                    w_j \ &\text{for} \quad j = 1,\dots,n     \\
                    0   \ &\text{for} \quad j = n+1,\dots,m
            \end{cases}
\end{align*}
    (The second case doesn't matter and 0 could be any vector in $W$.)

    Every $w \in W$ can be written with $a_1,\dots,a_n \in \F$ as
\begin{align*}
        w   &= a_1w_1 + \dots + a_nw_n          \\
            &= a_1(Tv_1) + \dots + a_n(Tv_n)    \\
            &= T(a_1v_1 + \dots + a_nv_n)
\end{align*}
    $V$ contains $a_1v_1 + \dots + a_nv_n$ because $V$ contains all linear combinations of its basis.

    So for every $w \in W$ there is some $v \in V$ such that $Tv = w$.
    Thus $T$ is surjective.
\end{document}
