\documentclass[a5paper]{article}
\usepackage{amsmath}
\usepackage[top=1cm,right=1cm,bottom=1cm,left=1cm]{geometry}
\usepackage[pdfstartview=FitH]{hyperref}
\setlength\parindent{0pt}
\setlength\parskip{1em}
%\usepackage{xcolor}
%\pagecolor[rgb]{0.1,0.1,0.1}
%\color[rgb]{1.0,1.0,1.0}
\begin{document}
\newcommand    \C          {\mathbf{C}}
\newcommand    \R          {\mathbf{R}}
\renewcommand  \L          {\mathcal{L}}
\newcommand    \F          {\mathbf{F}}
\renewcommand  \P          {\mathcal{P}}
\newcommand    \nullspace  {\text{null\;}}
\newcommand    \range      {\text{range\;}}
\newcommand    \linspan    {\text{span\;}}

    $V$ and $W$ are finite-dimensional.

    Let $v_1,\dots,v_m$ be a basis of $V$ and $w_1,\dots,w_n$ be a basis of $W$.
    Thus $\dim V = m$ and $\dim W = n$.

    First suppose there exists an injective linear map from $V$ to $W$.
    No linear map to a larger dimensional space is injective, so $\dim V \le \dim W$.

    To prove the other direction, suppose $\dim V \le \dim W$.
    Then $m \le n$.
    Hence there exists $T \in \L(V,W)$ such that
\begin{align*}
        Tv_j = w_j \quad \text{for} \quad j = 1, \dots, m
\end{align*}
    For $v \in V$, there are $a_1,\dots,a_m$ such that
\begin{align*}
         v &=   a_1v_1 + \dots + a_mv_m             \\
        Tv &= T(a_1v_1 + \dots + a_mv_m)            \\
        Tv &= a_1(Tv_1) + \dots + a_m(Tv_m)         \\
        Tv &= a_1w_1 + \dots + a_mw_m
\end{align*}
    Suppose $Tv = 0$.
    Because $w_1,\dots,w_m$ is linearly independent, $a_1=\dots=a_m=0$.
    That means $v=0$.
    Hence $T$ is injective.
\end{document}
