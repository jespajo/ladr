\documentclass[a5paper]{article}
\usepackage{amsmath}
\usepackage[top=1cm,right=1cm,bottom=1cm,left=1cm]{geometry}
\usepackage[pdfstartview=FitH]{hyperref}
\setlength\parindent{0pt}
\setlength\parskip{1em}
%\usepackage{xcolor}
%\pagecolor[rgb]{0.1,0.1,0.1}
%\color[rgb]{1.0,1.0,1.0}
\begin{document}
\newcommand    \C          {\mathbf{C}}
\newcommand    \R          {\mathbf{R}}
\renewcommand  \L          {\mathcal{L}}
\newcommand    \F          {\mathbf{F}}
\renewcommand  \P          {\mathcal{P}}
\newcommand    \nullspace  {\text{null\;}}
\newcommand    \range      {\text{range\;}}
\newcommand    \linspan    {\text{span\;}}

    We know there exists a linear map on $V$ whose null space and range are finite-dimensional.
    Denote this map $T$.

    Suppose $u_1,\dots,u_m$ is a basis of $\nullspace T$.
    Suppose $Tw_1,\dots,Tw_n$ is a basis of $\range T$.

    For $v \in V$, there are $a_1,\dots,a_n \in \F$ such that
\begin{align*}
        Tv &= a_1(Tw_1) + \dots + a_n(Tw_n)     \\
        Tv &= T(a_1w_1 + \dots + a_nw_n) \\
        0 &= T(a_1w_1 + \dots + a_nw_n) - Tv \\
        0 &= T(a_1w_1 + \dots + a_nw_n - v) 
\end{align*}
    Hence $a_1w_1 + \dots + a_nw_n - v \in \nullspace T$.

    So there are $b_1,\dots,b_m \in \F$ such that
\begin{align*}
        b_1u_1 + \dots + b_mu_m &= a_1w_1 + \dots + a_nw_n - v 
\end{align*}
    Hence
\begin{align*}
        v &= b_1u_1 + \dots + b_mu_m - a_1w_1 - \dots - a_nw_n
\end{align*}
    So the list $u_1,\dots,u_m,w_1,\dots,w_n$ spans $V$.

    Thus $V$ is finite-dimensional.
\end{document}
