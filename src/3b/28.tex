\documentclass[a5paper]{article}
\usepackage{amsmath}
\usepackage[top=1cm,right=1cm,bottom=1cm,left=1cm]{geometry}
\setlength\parindent{0pt}
\setlength\parskip{1em}
%
%\usepackage{xcolor}
%\pagecolor[rgb]{0.1,0.1,0.1}
%\color[rgb]{1.0,1.0,1.0}
%
\begin{document}
\newcommand   \C           {\mathbf{C}}
\newcommand   \R           {\mathbf{R}}
\renewcommand \L           {\mathcal{L}}
\newcommand   \F           {\mathbf{F}}
\renewcommand \P           {\mathcal{P}}
\newcommand   \M           {\mathcal{M}}
\newcommand   \op          {\operatorname}

    3.B.28.
    Suppose $D \in \L\big(\P(\R)\big)$ is such that $\op{deg}Dp = (\op{deg}P) - 1$ for every non-constant polynomial $p \in \P(\R)$.
    Prove that $D$ is surjective.

    Suppose $p \in \P(\R)$.
    Let $m$ denote the degree of $p$.
    Then $p \in \P_m(\R)$.

    Consider this list of polynomials:
\begin{equation*}
        Dz, Dz^2, \dots, Dz^{m+1} .
\end{equation*}
    The polynomial with the highest degree in the list is the last one:
\begin{align*}
        \op{deg}Dz^{m+1}&=\op{deg}z^{m+1} - 1 \\
                          &=m .
\end{align*}
    Hence every polynomial in the list is an element of $P_m(\R)$.

    Since $\op{dim}P_m(\R) = m+1$, our list has the right length to be a basis of $\P_m(\R)$.
    So we can show that it is a basis by showing that it is linearly independent.

    Consider a linear combination of the list with the scalars $a_1,\dots,a_{m+1} \in \R$.
    Suppose some scalar is nonzero.
    Let $k$ be the largest value for which $a_k \neq 0$.
    Then we can write the combination as
\begin{equation*}
        a_1Dz + \dots + a_kDz^k .
\end{equation*}
    The last term is a polynomial with a nonzero coefficient for $z^{k-1}$, because $\op{deg}Dz^k = k-1$.
    No other term has a nonzero coefficient for $z^{k-1}$.
    This implies that the sum as a whole is nonzero.

    Hence if the combination is zero, $a_1=\dots=a_{m+1}=0$.
    Thus the list $Dz,\dots,Dz^{m+1}$ is linearly independent and therefore a basis of $P_m(\R)$.
    In particular the list spans $P_m(\R)$.

    Since $p \in \P_m(\R)$, this means there exist $b_1,\dots,b_{m+1}\in\R$ such that
\begin{align*}
        p &= b_1Dz + \dots + b_{m+1}Dz^{m+1} \\
          &= D(b_1z + \dots + b_{m+1}z^{m+1}) .
\end{align*}
    Thus for $p \in \P(\R)$ there exists some $q \in \P(\R)$ such that $Dq = p$.
    In other words, $D$ is surjective.

%    First we show that $D$ takes every constant polynomial to 0.
%
%    Consider two functions, $z$ and $z+1$.
%    Both are first-degree polynomials, so there exist constant functions $a, b \in \P_0(\R)$ such that
%\begin{equation*}
%        Dz = a \quad \text{and} \quad D(z+1) = b .
%\end{equation*}
%    In fact $a = b$.
%    To see this, suppose we have another function, $(b - a)z - a$.
%    If $a \neq b$, this would be a first-degree polynomial, so applying $D$ would send the function to a constant function.
%    But, applying $D$,
%\begin{align*}
%        D\left((b - a)z - a\right) &= D(bz - az - a) \\
%                           &= D\left(bz - a(z + 1)\right) \\
%                           &= bDz - aD(z + 1) \\
%                           &= ba - ab \\
%                           &= 0 .
%\intertext{
%    Therefore $a = b$.
%    So
%}
%        Dz &= D(z+1) \\
%        Dz &= Dz+D1 \\
%        0  &= D1 .
%\end{align*}
%    Hence for $k \in \P_0(\R)$,
%\begin{equation*}
%        Dk = kD1 = k0 = 0 .
%\end{equation*}
\end{document}
