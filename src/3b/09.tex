\documentclass[a5paper]{article}
\usepackage{amsmath}
\usepackage[top=1cm,right=1cm,bottom=1cm,left=1cm]{geometry}
\setlength\parindent{0pt}
\setlength\parskip{1em}
% 
% \usepackage{xcolor}
% \pagecolor[rgb]{0.1,0.1,0.1}
% \color[rgb]{1.0,1.0,1.0}
% 
\begin{document}

\newcommand    \C  { \mathbf{C} }
\newcommand    \R  { \mathbf{R} }
\renewcommand  \L  { \mathcal{L} }
\newcommand    \F  { \mathbf{F} }
\renewcommand  \P  { \mathcal{P} }
\newcommand    \nullspace { \text{null\;} }
\newcommand    \range     { \text{range\;} }
\newcommand    \linspan   { \text{span\;} }

We have $T \in \L(V,W)$ is injective and $v_1,\dots,v_n$ is linearly independent in $V$.

Suppose $a_1,\dots,a_n \in \F$ such that
\begin{align*}
    0 &= a_1(Tv_1) + \dots + a_n(Tv_n)
\intertext{
Then because linear maps take 0 to 0,
}
    T(0) &= a_1(Tv_1) + \dots + a_n(Tv_n)       \\
         &= T(a_1v_1) + \dots + T(a_nv_n)       \\
         &= T(a_1v_1  + \dots + a_nv_n)   
\intertext{
Hence the injectivity of $T$ implies
}
       0 &= a_1v_1 + \dots + a_nv_n
\end{align*}
Therefore $a_1=\dots=a_n=0$ since $v_1,\dots,v_n$ is linearly independent in $V$.
Thus $Tv_1,\dots,Tv_n$ is linearly independent in $W$.
\end{document}
