\documentclass[a5paper]{article}
\usepackage{amsmath}
\usepackage[top=1cm,right=1cm,bottom=1cm,left=1cm]{geometry}
\setlength\parindent{0pt}
\setlength\parskip{1em}
%\usepackage{xcolor}
%\pagecolor[rgb]{0.1,0.1,0.1}
%\color[rgb]{1.0,1.0,1.0}
\begin{document}
\newcommand   \C           {\mathbf{C}}
\newcommand   \R           {\mathbf{R}}
\renewcommand \L           {\mathcal{L}}
\newcommand   \F           {\mathbf{F}}
\renewcommand \P           {\mathcal{P}}
\newcommand   \nullspace   {\text{null\:}}
\newcommand   \range       {\text{range\:}}
\newcommand   \linspan     {\text{span\:}}
\newcommand   \question[1] {\textbf{\boldmath#1\unboldmath}\par}

\question{
    3.B.21.
    Suppose $V$ is finite-dimensional and $T \in \L(V,W)$.
    Prove that $T$ is surjective if and only if there exists $S \in \L(W,V)$ such that $TS$ is the identity map on $W$.
}
    To prove in one direction, suppose there exists $S \in \L(W,V)$ such that $TS$ is the identity map on $W$.

    Suppose $w \in W$.
    Then
\begin{align*}
        w = (TS)(w) = T(Sw) .
\end{align*}
    Thus $w \in \range T$.
    So $T$ is surjective.

    To prove in the other direction, suppose $T$ is surjective.

    No linear map to a larger dimensional subspace is surjective, so $\dim V \ge \dim W$.
    In particular $W$ is finite-dimensional.
    Thus $\range T$ has a basis.

    Suppose $Tv_1,\dots,Tv_n$ is a basis of $\range T$ with $v_1,\dots,v_n \in V$.
    Of course $Tv_1,\dots,Tv_n$ is also a basis of $W$ because $\range T = W$.

    Then we can write any $w \in W$ as
\begin{align*}
        w &= a_1Tv_1 + \dots + a_nTv_n
\end{align*}
    with $a_1,\dots,a_n \in \F$.

    There exists $S \in \L(W,V)$ such that
\begin{equation*}
        S(Tv_j) = v_j
\end{equation*}
    for each $j = 1,\dots,n$.
    Hence
\begin{align*}
        Sw &= S \big( a_1Tv_1 + \dots + a_nTv_n \big) \\
           &= a_1 \big( S(Tv_1) \big) + \dots + a_n \big( S(Tv_n) \big) \\
           &= a_1v_1 + \dots + a_nv_n .
\end{align*}
    Thus
\begin{align*}
        (TS)(w) &= T (Sw) \\
                &= T (a_1v_1 + \dots + a_nv_n) \\
                &= a_1Tv_1 + \dots + a_nTv_n \\
                &= w .
\end{align*}
    So $TS$ is the identity map on $W$.
\end{document}
