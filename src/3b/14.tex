\documentclass[a5paper]{article}
\usepackage{amsmath}
\usepackage[top=1cm,right=1cm,bottom=1cm,left=1cm]{geometry}
\usepackage[pdfstartview=FitH]{hyperref}
\setlength\parindent{0pt}
\setlength\parskip{1em}
%\usepackage{xcolor}
%\pagecolor[rgb]{0.1,0.1,0.1}
%\color[rgb]{1.0,1.0,1.0}
\begin{document}
\newcommand    \C          {\mathbf{C}}
\newcommand    \R          {\mathbf{R}}
\renewcommand  \L          {\mathcal{L}}
\newcommand    \F          {\mathbf{F}}
\renewcommand  \P          {\mathcal{P}}
\newcommand    \nullspace  {\text{null\;}}
\newcommand    \range      {\text{range\;}}
\newcommand    \linspan    {\text{span\;}}

    $U$ is a 3-dimensional subspace of $\R^8$.
    $T \in \L(\R^8, \R^5)$.
    $\nullspace T = U$.

    From the Fundamental Theorem of Linear Maps,
\begin{align*}
            \dim \R^8 &= \dim \nullspace T + \dim \range T      \\
                    8 &= 3 + \dim \range T                      \\
        \dim \range T &= 5
\intertext{
    So $\range T$ is a 5-dimensional subspace of $\R^5$.
    Hence
}
             \range T &= \R^5
\end{align*}
    Thus $T$ is surjective.
\end{document}
