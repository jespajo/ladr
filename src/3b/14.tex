\documentclass[a5paper]{article}
\usepackage{amsmath}
\usepackage[top=1cm,right=1cm,bottom=1cm,left=1cm]{geometry}
\usepackage[pdfstartview=FitH]{hyperref}
\setlength\parindent{0pt}
\setlength\parskip{1em}
%\usepackage{xcolor}
%\pagecolor[rgb]{0.1,0.1,0.1}
%\color[rgb]{1.0,1.0,1.0}
\begin{document}
\newcommand    \C          {\mathbf{C}}
\newcommand    \R          {\mathbf{R}}
\renewcommand  \L          {\mathcal{L}}
\newcommand    \F          {\mathbf{F}}
\renewcommand  \P          {\mathcal{P}}
\newcommand    \nullspace  {\text{null\;}}
\newcommand    \range      {\text{range\;}}
\newcommand    \linspan    {\text{span\;}}

    Suppose $U$ is a subspace of $\F^5$ such that
\begin{align*}
        U = \left\{ (x_1,x_2,x_3,x_4,x_5) \in \F^5 : x_1=3x_2 \text{ and } x_3=x_4=x_5 \right\}
\end{align*}
    Then for any $u \in U$,
\begin{align*}
        u &= (x_1, x_2, x_3, x_4, x_4)          \\
          &= (3x_2, x_2, x_3, x_3, x_3)         \\
          &= x_2(3,1,0,0,0) + x_3(0,0,1,1,1)
\end{align*}
    So the list $(3,1,0,0,0), (0,0,1,1,1)$ spans $U$.

    This list is also linearly independent because if $u=(0,0,0,0,0)$ then $x_2=x_3=0$.

    Hence this list is a basis of $U$.
    Thus $\dim U = 2$.

    Now suppose $T \in \L(\F^5,W)$ with $\nullspace T = U$.
    From the Fundamental Theorem of Linear Maps,
\begin{align*}
                   \dim U + \dim \range T &= \dim \F^5     \\
                        2 + \dim \range T &= 5             \\
                            \dim \range T &= 3
\end{align*}
    $\range T$ is a subspace of $W$, so we know $\dim W \ge 3$.

    Hence $W$ cannot be $\F^2$ because $\dim \F^2=2$.
\end{document}
