\documentclass[a5paper]{article}
\usepackage{amsmath}
\usepackage[top=1cm,right=1cm,bottom=1cm,left=1cm]{geometry}
\setlength\parindent{0pt}
\setlength\parskip{1em}
%\usepackage{xcolor}
%\pagecolor[rgb]{0.1,0.1,0.1}
%\color[rgb]{1.0,1.0,1.0}
\begin{document}
\newcommand   \C           {\mathbf{C}}
\newcommand   \R           {\mathbf{R}}
\renewcommand \L           {\mathcal{L}}
\newcommand   \F           {\mathbf{F}}
\renewcommand \P           {\mathcal{P}}
\newcommand   \question[1] {\textbf{\boldmath#1\unboldmath}\par}
\newcommand   \op          {\operatorname}

\question{
    3.B.24.
    Suppose $W$ is finite-dimensional and $T_1,T_2 \in \L(V, W)$.
    Prove that $\op{null}T_1 \subset \op{null}T_2$ if and only if there exists $S \in \L(W, W)$ such that $T_2=ST_1$.
}

    To prove in one direction, suppose there exists $S \in \L(W, W)$ such that $T_2=ST_1$.

    Suppose $v \in \op{null}T_1$.
    Then $T_1v = 0$.
    So
\begin{equation*}
        T_2v = (ST_1)(v) = S(T_1v) = S(0) = 0 .
\end{equation*}
    Thus $v \in \op{null}T_2$.
    Hence $\op{null}T_1 \subset \op{null}T_2$.

    To prove in the other direction, suppose $\op{null}T_1 \subset \op{null}T_2$.

    $\op{range}T_1$ has a basis because it is a subspace of $W$.
    Suppose $u_1,\dots,u_m \in V$ such that $T_1u_1,\dots,T_1u_m$ is a basis of $\op{range}T_1$.
    
    Suppose $v \in V$.
    There exist $a_1,\dots,a_m \in \F$ such that
\begin{align*}
        T_1v &= a_1T_1u_1 + \dots + a_mT_1u_m .
\intertext{
    Taking $T_1v$ away from both sides,
}
        0 &= a_1T_1u_1 + \dots + a_mT_1u_m - T_1v \\
          &= T_1(a_1u_1 + \dots + a_mu_m - v) .
\end{align*}
    So $a_1u_1 + \dots + a_mu_m - v \in \op{null}T_1$.

    Hence this vector is also in $\op{null}T_2$ because $\op{null}T_1 \subset \op{null}T_2$.
    Thus
\begin{align*}
        0 &= T_2(a_1u_1 + \dots + a_mu_m - v) \\
          &= a_1T_2u_1 + \dots + a_mT_2u_m - T_2v .
\intertext{
    Adding $T_2v$ to both sides,
}
        T_2v &= a_1T_2u_1 + \dots + a_mT_2u_m \\
             &= T_2(a_1u_1 + \dots + a_mu_m) .
\end{align*}
    Now, since $T_1u_1,\dots,T_1u_m$ is a linearly independent list in $W$, we can extend it to a basis of $W$ by appending $w_1,\dots,w_n \in W$.

    Then because $T_1u_1,\dots,T_1u_m,w_1,\dots,w_n$ is a basis of $W$ and $T_2u_1,\dots,T_2u_m \in W$, we can find $S \in \L(W,W)$ such that
\begin{alignat*}{3}
        S(T_1u_j) &= T_2u_j && \quad\text{for}\quad j=\{1,\dots,m\} \quad\text{and} \\
             Sw_k &= 0      && \quad\text{for}\quad k=\{1,\dots,n\} .
\end{alignat*}
    Hence
\begin{align*}
        (ST_1)(v) &= S( T_1v ) \\
                  &= S( a_1T_1u_1 + \dots + a_mT_1u_m ) \\
                  &= a_1 S(T_1u_1) + \dots + a_m S(T_1u_m) \\
                  &= a_1T_2u_1 + \dots + a_mT_2u_m \\
                  &= T_2(a_1u_1 + \dots + a_mu_m) \\
                  &= T_2v .
\end{align*}
\end{document}
%
%
%    Suppose $w_1,\dots,w_m$ is a basis of $\op{range} T_1$.
%
%    Clearly there exist $v_1,\dots,v_m \in V$ such that $T_1v_j = w_j$ for each $j=1,\dots,m$.
%    In fact we will prove that
%\begin{equation*}
%        V = \op{null} T_1 \oplus \op{span}(v_1,\dots,v_m) .
%\end{equation*}
%    This takes a few steps.
%    First we show that $V = \op{null} T_1 + \op{span}(v_1,\dots,v_m)$.
%
%    $\op{null} T_1 + \op{span}(v_1,\dots,v_m) \subset V$ because the sum of subspaces is the smallest containing subspace, and $V$ contains both $\op{null} T_1$ and $\op{span}(v_1,\dots,v_m)$.
%
%    To see inclusion in the other direction, suppose $v \in V$.
%
%    Then because $T_1v \in \op{range} T_1$, there exist $a_1,\dots,a_m \in \F$ such that
%\begin{equation*}
%        a_1T_1v_1 + \dots + a_mT_1v_m = T_1v .
%\end{equation*}
%    Moving each vector from the left side to the right side of the equation,
%\begin{align*}
%        0 &= T_1v - a_1T_1v_1 - \dots - a_mT_1v_m \\
%          &= T_1(v - a_1v_1 - \dots - a_mv_m) .
%\end{align*}
%    Thus $(v - a_1v_1 - \dots - a_mv_m) \in \op{null} T_1$.
%
%    Then we can write $v$ as a sum of a vector in $\op{null} T_1$ and a vector in $\op{span}(v_1,\dots,v_m)$:
%\begin{equation*}
%        v = (v - a_1v_1 - \dots - a_mv_m) + (a_1v_1 + \dots + a_mv_m) .
%\end{equation*}
%    This shows $V \subset \op{null} T_1 + \op{span}(v_1,\dots,v_m)$.
%
%    Hence $V = \op{null} T_1 + \op{span}(v_1,\dots,v_m)$.
%    To see that this sum is a direct sum, suppose $u_1 \in \op{null} T_1$ and $u_2 \in \op{span}(v_1,\dots,v_m)$ such that $u_1+u_2=0$.
%
%    There exist $b_1,\dots,b_m \in \F$ such that $u_2 = b_1v_1+\dots+b_mv_m$.
%    So
%\begin{equation*}
%        u_1 + b_1v_1+\dots+b_mv_m = 0 .
%\end{equation*}
%    Applying $T_1$ to both sides, we get
%\begin{equation*}
%        b_1T_1v_1+\dots+b_mT_1v_1 = 0
%\end{equation*}
%    where $T_1u_1$ disappears because $u_1 \in \op{null} T_1$.
%
%    Then $b_1=\dots=b_m=0$ because $T_1v_1,\dots,T_1v_m$ is linearly independent.
%    Thus
%\begin{equation*}
%        u_2 = 0v_1+\dots+0v_m = 0 .
%\end{equation*}
%    Then $u_1+0=0$ so $u_1=0$.
%
%    Hence the sum of $\op{null} T_1$ and $\op{span}(v_1,\dots,v_m)$ is a direct sum.
%
%%%
%
%    The list $v_1,\dots,v_m$ is linearly independent.
%    To see this, suppose $a_1,\dots,a_m \in \F$ such that $a_1v_1 + \dots + a_mv_m = 0$.
%    Applying $T_1$ to both sides yields $a_1w_1 + \dots + a_mw_m = 0$, so $a_1=\dots=a_m=0$.
%
