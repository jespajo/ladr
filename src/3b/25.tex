\documentclass[a5paper]{article}
\usepackage{amsmath}
\usepackage[top=1cm,right=1cm,bottom=1cm,left=1cm]{geometry}
\setlength\parindent{0pt}
\setlength\parskip{1em}
%\usepackage{xcolor}
%\pagecolor[rgb]{0.1,0.1,0.1}
%\color[rgb]{1.0,1.0,1.0}
\begin{document}
\newcommand   \C           {\mathbf{C}}
\newcommand   \R           {\mathbf{R}}
\renewcommand \L           {\mathcal{L}}
\newcommand   \F           {\mathbf{F}}
\renewcommand \P           {\mathcal{P}}
\newcommand   \question[1] {\textbf{\boldmath#1\unboldmath}\par}
\newcommand   \op          {\operatorname}

\question{
    3.B.25.
    Suppose $V$ is finite-dimensional and $T_1,T_2 \in \L(V, W)$.
    Prove that $\op{range}T_1 \subset \op{range}T_2$ if and only if there exists $S \in \L(V, V)$ such that $T_1=T_2S$.
}

    To prove in one direction, suppose there exists $S \in \L(V, V)$ such that $T_1=T_2S$.

    Suppose $v \in \op{range}T_1$.
    There exists $u \in V$ such that
\begin{equation*}
        v = T_1u = (T_2S)(u) = T_2(Su) .
\end{equation*}
    Thus $v \in \op{range}T_2$.
    Hence $\op{range}T_1 \subset \op{range}T_2$.

    To prove in the other direction, suppose $\op{range}T_1 \subset \op{range}T_2$.

    Suppose $v_1,\dots,v_n$ is a basis of $V$.
    Suppose $v \in V$.
    Then
\begin{equation*}
        v = a_1v_1 + \dots + a_nv_n
\end{equation*}
    with $a_1,\dots,a_n \in \F$.

    For $j=\{1,\dots,n\}$, since $T_1v_j$ is in $\op{range}T_1$, it is also in $\op{range}T_2$ because $\op{range}T_1 \subset \op{range}T_2$.
    So there exists $u_j \in V$ such that $T_2u_j = T_1v_j$.

    Hence we can define $S \in \L(V,V)$ on the basis of $v_1,\dots,v_n$ such that $Sv_j = u_j$ for each $j=\{1,\dots,n\}$.

    Thus
\begin{align*}
        Sv &= S(a_1v_1 + \dots + a_nv_n) \\
           &= a_1Sv_1 + \dots + a_nSv_n \\
           &= a_1u_1 + \dots + a_nu_n .
\end{align*}
    Hence
\begin{align*}
        (T_2S)(v) &= T_2(Sv) \\
                  &= T_2(a_1u_1 + \dots + a_nu_n) \\
                  &= a_1T_2u_1 + \dots + a_nT_2u_n \\
                  &= a_1T_1v_1 + \dots + a_nT_1v_n \\
                  &= T_1(a_1v_1 + \dots + a_nv_n) \\
                  &= T_1v .
\end{align*}


\end{document}
%
%       $V$ is finite-dimensional, so $\op{null}T_1$ is finite-dimensional.
%       The Fundamental Theorem of Linear Maps says that
%   \begin{equation*}
%           \dim\op{range}T_1 = \dim V - \dim\op{null}T_1
%   \end{equation*}
%       so $\op{range}T_1$ is finite-dimensional and has a basis.
%   
%       Say $u_1,\dots,u_m \in V$ such that $T_1u_1,\dots,T_1u_m$ is a basis of $\op{range}T_1$.
%       
%       Then $u_1,\dots,u_m$ are linearly independent in $V$.
%       To see this...
%   
%       We can extend $u_1,\dots,u_m$ to a basis of $V$ by appending $v_1,\dots,v_n \in V$.
%   
%       Suppose $v \in V$.
%       We can write $v$ as
%   \begin{equation*}
%           v = a_1u_1+\dots+a_mu_m+b_1v_1+\dots+b_nv_n
%   \end{equation*}
%       with $a_1,\dots,a_m,b_1,\dots,b_n \in \F$.
%   
%       We can find a linear map $S \in \L(V,V)$ such that
%   \begin{alignat*}{3}
%           Su_j &= u_j && \quad\text{for}\quad j=\{1,\dots,m\} \quad\text{and} \\
%           Sv_k &= 0   && \quad\text{for}\quad k=\{1,\dots,n\} .
%   \end{alignat*}
%       Hence
%   \begin{align*}
%           Sv &= S(a_1u_1+\dots+a_mu_m+b_1v_1+\dots+b_nv_n) \\
%              &= a_1Su_1 + \dots + a_mSu_m + b_1Sv_1 + \dots + b_nSv_n \\
%              &= a_1u_1 + \dots + a_mu_m .
%   \end{align*}
%       Thus
%   \begin{align*}
%           (T_2S)(v) &= T_2(Sv) \\
%                     &= T_2(a_1u_1+\dots+a_mu_m) \\
%                     &= a_1T_2u_1 + \dots + a_mT_2u_m .
%   \end{align*}
%       So we may complete our proof by showing that $T_1v = a_1T_2u_1 + \dots + a_mT_2u_m$.
%   
%   
%   \end{document}
%       
%       
%   
%   %    Since $\op{null}T_2$ is a subspace of $V$, it has a basis.
%   %    Say $u_1,\dots,u_m$ is a basis of $\op{null}T_2$.
%   %
%   %    We can extend this list to a basis of $V$ by appending $v_1,\dots,v_n \in V$.
%   %    Thus $u_1,\dots,u_m,v_1,\dots,v_n$ is a basis of $V$.
%   %
%   %    Suppose $v \in V$.
%   %    Then
%   %\begin{equation*}
%   %        v = a_1u_1 + \dots + a_mu_m + b_1v_1 + \dots + b_nv_n .
%   %\end{equation*}
%   %    Thus
%   %\begin{align*}
%   %        T_2v &= T_2(a_1u_1 + \dots + a_mu_m + b_1v_1 + \dots + b_nv_n) \\
%   %             &= a_1T_2u_1 + \dots + a_mT_2u_m + b_1T_2v_1 + \dots + b_nT_2v_n \\
%   %             &= b_1T_2v_1 + \dots + b_nT_2v_n .
%   %\end{align*}
%   %    Hence the list $T_2v_1,\dots,T_2v_n \in W$ spans $\op{range}T_2$.
%   %
%   %    We can write $T_1v$ as
%   %\begin{align*}
%   %        T_1v &= T_1(a_1u_1 + \dots + a_mu_m + b_1v_1 + \dots + b_nv_n) \\
%   %             &= a_1T_1u_1 + \dots + a_mT_1u_m + b_1T_1v_1 + \dots + b_nT_1v_n .
%   %\end{align*}
%   %    Since $\op{range}T_1 \subset \op{range}T_2$, we can also write $T_1v$ as a linear combination of $T_2v_1,\dots,T_2v_n$:
%   %\begin{equation*}
%   %        v = a_1u_1 + \dots + a_mu_m + b_1v_1 + \dots + b_nv_n .
%   %\end{equation*}
%   %
%   %\end{document}
%   %    
%   %
%   %%       To prove in the other direction, suppose $\op{range}T_1 \subset \op{range}T_2$.
%   %%   
%   %%       Since $\op{null}T_1$ is a subspace of $V$, it has a basis.
%   %%       Say $u_1,\dots,u_m$ is a basis of $\op{null}T_1$.
%   %%   
%   %%       We can extend this list to a basis of $V$ by appending $v_1,\dots,v_n \in V$.
%   %%       Thus $u_1,\dots,u_m,v_1,\dots,v_n$ is a basis of $V$.
%   %%   
%   %%       Suppose $v \in V$.
%   %%       Then
%   %%   \begin{equation*}
%   %%           v = a_1u_1 + \dots a_mu_m + b_1v_1 + \dots + b_nv_n
%   %%   \end{equation*}
%   %%       with $a_1,\dots,a_m,b_1,\dots,b_n \in \F$.
%   %%       Hence
%   %%   \begin{align*}
%   %%           T_1v &= T_1(a_1u_1 + \dots a_mu_m + b_1v_1 + \dots + b_nv_n) \\
%   %%                &= b_1T_1v_1 + \dots + b_nT_1v_n .
%   %%   \end{align*}
%   %%       Then since $T_1v \in \op{range}T_2$, there exists $u \in V$ such that $T_1v = T_2u$.
%   %%       Hence there exist $c_1,\dots,c_m,d_1,\dots,d_n \in \F$ such that
%   %%   \begin{align*}
%   %%           T_1v &= T_2u \\
%   %%                &= T_2(c_1u_1 + \dots c_mu_m + d_1v_1 + \dots + d_nv_n) \\
%   %%                &= c_1T_2u_1 + \dots c_mT_2u_m + d_1T_2v_1 + \dots + d_nT_2v_n .
%   %%   \end{align*}
%   %%     
%   %%   
%   %%   
%   %%   
%   %%       
%   %%   
%   %%       We can find $S \in \L(V,V)$ such that
%   %%   \begin{alignat*}{3}
%   %%           Su_j &= 0   && \quad\text{for}\quad j=\{1,\dots,m\} \quad\text{and} \\
%   %%           Sv_k &= v_k && \quad\text{for}\quad k=\{1,\dots,n\} .
%   %%   \end{alignat*}
%   %%       Thus
%   %%   \begin{align*}
%   %%           Sv &= S(a_1u_1 + \dots a_mu_m + b_1v_1 + \dots + b_nv_n) \\
%   %%              &= a_1Su_1 + \dots a_mSu_m + b_1Sv_1 + \dots + b_nSv_n \\
%   %%              &= b_1v_1 + \dots + b_nv_n .
%   %%   \end{align*}
%   %%       Hence
%   %%   \begin{align*}
%   %%           (T_2S)(v) &= T_2( Sv ) \\
%   %%                     &= T_2(b_1v_1 + \dots + b_nv_n) \\
%   %%                     &= b_1T_2v_1 + \dots + b_nT_2v_n \\
%   %%                     &= T_1v .
%   %%   \end{align*}
%   %%       (We haven't proved this yet but that's the goal.)
%   %%   \end{document}
%   %%   %    Say $v_1,\dots,v_m$ is a basis of $V$.
%   %%   %
%   %%   %    Suppose $v \in V$.
%   %%   %    There exist $a_1,\dots,a_m \in \F$ such that
%   %%   %\begin{equation*}
%   %%   %        v = a_1v_1 + \dots + a_mv_m .
%   %%   %\end{equation*}
%   %%   %    Hence
%   %%   %\begin{align*}
%   %%   %        T_1v &= T_1(a_1v_1 + \dots + a_mv_m) = a_1T_1v_1 + \dots + a_mT_1v_m \quad\text{and} \\
%   %%   %        T_2v &= T_2(a_1v_1 + \dots + a_mv_m) = a_1T_2v_1 + \dots + a_mT_2v_m .
%   %%   %\end{align*}
%   %%   %    Since $\op{range}T_1 \subset \op{range}T_2$, there also exist $b_1,\dots,b_m \in \F$ such that
%   %%   %\begin{equation*}
%   %%   %        T_1v = T_2(b_1v_1 + \dots + b_mv_m) = b_1T_2v_1 + \dots + b_mT_2v_m .
%   %%   %\end{equation*}
%   %%   %    Consider
%   %%   %\begin{align*}
%   %%   %        T_1v + T_2v &= b_1T_2v_1 + \dots + b_mT_2v_m + a_1T_2v_1 + \dots + a_mT_2v_m \\
%   %%   %                    &= (b_1 + a_1)T_2v_1 + \dots + (b_m + a_m)T_2v_m \\
%   %%   %                    &= T_2\big( (b_1 + a_1)v_1 + \dots + (b_m + a_m)v_m \big) . \\
%   %%   %\end{align*}
