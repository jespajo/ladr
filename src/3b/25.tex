\documentclass[a5paper]{article}
\usepackage{amsmath}
\usepackage{amssymb}
\usepackage[top=1cm,right=1cm,bottom=2cm,left=1cm]{geometry}
\setlength\parindent{0pt}
\setlength\parskip{1em}
%
\begin{document}
\newcommand   \C           {\mathbf{C}}
\newcommand   \R           {\mathbf{R}}
\renewcommand \L           {\mathcal{L}}
\newcommand   \F           {\mathbf{F}}
\renewcommand \P           {\mathcal{P}}
\newcommand   \M           {\mathcal{M}}
\newcommand   \op          {\operatorname}

    3.B.25.
    Suppose $W$ is finite-dimensional and $S,T \in \L(V,W)$.
    Prove that $\op{null}S \subseteq \op{null}T$ if and only if there exists $E \in \L(W)$ such that $T=ES$.

    To prove in one direction, suppose there exists $E \in \L(W)$ such that $T=ES$.

    Suppose $v \in \op{null}S$.
    Then $Sv = 0$.
    So
\begin{equation*}
        Tv = (ES)(v) = E(Sv) = E(0) = 0 .
\end{equation*}
    Thus $v \in \op{null}T$.
    Hence $\op{null}S \subseteq \op{null}T$.

    To prove in the other direction, suppose $\op{null}S \subseteq \op{null}T$.

    $\op{range}S$ has a basis because it is a subspace of the finite-dimensional vector space $W$.
    Suppose $Su_1,\dots,Su_m$ is a basis of $\op{range}S$ with some $u_1,\dots,u_m \in V$.

    We can extend this list to a basis of $W$.
    Suppose we do so by appending $w_1,\dots,w_n \in W$.
    Our basis of $W$ is
\begin{align*}
        Su_1,\dots,Su_m,\ w_1,\dots,w_n .
\end{align*}
    Define $E$ as a linear map from $W$ to itself such that
\begin{alignat*}{3}
        E(Su_j) &= Tu_j && \quad\text{for}\quad j=\{1,\dots,m\} \quad\text{and} \\
             Ew_k &= 0      && \quad\text{for}\quad k=\{1,\dots,n\} .
\end{alignat*}
    This is a valid definition of a linear map because it uses the structure from the linear map lemma to define a linear map on a basis of $W$ and because $Tu_1,\dots,Tu_m,0 \in W$.

    Now we must show that $T = ES$.
    To see this, suppose $v \in V$.
    We cannot define $v$ as a linear combination of a basis of $V$, because we do not know whether $V$ is finite-dimensional.
    But we can find $Sv$ and $Tv$ in terms of our basis of $W$, as follows.

    Since $Su_1,\dots,Su_m$ is a basis of $\op{range}S$, there exist $a_1,\dots,a_m \in \F$ such that
\begin{align*}
        Sv &= a_1Su_1 + \dots + a_mSu_m .
\intertext{
    Taking $Sv$ away from both sides,
}
        0 &= a_1Su_1 + \dots + a_mSu_m - Sv \\
          &= S(a_1u_1 + \dots + a_mu_m - v) .
\end{align*}
    So $a_1u_1 + \dots + a_mu_m - v \in \op{null}S$.

    Hence this vector is also in $\op{null}T$ because $\op{null}S \subseteq \op{null}T$.
    Thus
\begin{align*}
        0 &= T(a_1u_1 + \dots + a_mu_m - v) \\
          &= a_1Tu_1 + \dots + a_mTu_m - Tv .
\intertext{
    Adding $Tv$ to both sides,
}
        Tv &= a_1Tu_1 + \dots + a_mTu_m .
\end{align*}
    Hence
\begin{align*}
        (ES)(v) &= E( Sv ) \\
                  &= E( a_1Su_1 + \dots + a_mSu_m ) \\
                  &= a_1 E(Su_1) + \dots + a_m E(Su_m) \\
                  &= a_1Tu_1 + \dots + a_mTu_m \\
                  &= Tv .
\end{align*}
    Thus $T = ES$, as desired.
\end{document}
