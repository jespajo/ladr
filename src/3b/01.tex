\documentclass[a5paper]{article}
\usepackage{amsmath}
\usepackage[top=1cm,right=1cm,bottom=1cm,left=1cm]{geometry}
\setlength\parindent{0pt}
\setlength\parskip{1em}
%
%\usepackage{xcolor}
%\pagecolor[rgb]{0.1,0.1,0.1}
%\color[rgb]{1.0,1.0,1.0}
%
\begin{document}

\newcommand    \C  { \mathbf{C} }
\newcommand    \R  { \mathbf{R} }
\renewcommand  \L  { \mathcal{L} }
\newcommand    \F  { \mathbf{F} }
\renewcommand \P           {\mathcal{P}}
\newcommand   \question[1] {\textbf{\boldmath#1\unboldmath}\par}
\newcommand   \op          {\operatorname}

\question{
    3.B.1.
    Give an example of a linear map $T$ with $\op{dim}\op{null}T = 3$ and $\op{dim}\op{range}T = 2$.
}

Define $T \in \L(\R^5,\R^2)$ by
\begin{equation*}
    T(x_1,x_2,x_3,x_4,x_5) = (x_1,x_2)
\end{equation*}
Then the range of $T$ is $\left\{ (x_1,x_2) \in \R^2 : x_1,x_2 \in \R \right\}$ which is equal to $\R^2$.
Thus
\begin{align*}
    \dim\text{range\;}T &= \dim \R^2        \\
                        &= 2
\intertext{
From the Fundamental Theorem of Linear Maps,
}
    \dim \text{null\;} T &= \dim \R^5 - \dim \text{range\;}T    \\
                         &= 5 - 2                               \\
                         &= 3
\end{align*}
\end{document}
