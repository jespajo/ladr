\documentclass[a5paper]{article}
\usepackage{amsmath}
\usepackage[top=1cm,right=1cm,bottom=1cm,left=1cm]{geometry}
\usepackage[pdfstartview=FitH]{hyperref}
\setlength\parindent{0pt}
\setlength\parskip{1em}
%\usepackage{xcolor}
%\pagecolor[rgb]{0.1,0.1,0.1}
%\color[rgb]{1.0,1.0,1.0}
\begin{document}
\newcommand    \C          {\mathbf{C}}
\newcommand    \R          {\mathbf{R}}
\renewcommand  \L          {\mathcal{L}}
\newcommand    \F          {\mathbf{F}}
\renewcommand  \P          {\mathcal{P}}
\newcommand    \nullspace  {\text{null\;}}
\newcommand    \range      {\text{range\;}}
\newcommand    \linspan    {\text{span\;}}

    $S_1,\dots,S_n$ are injective linear maps such that $S_1S_2\dots S_n$ makes sense.

    Suppose $u$ is an element in the domain of $S_n$ such that
\begin{align*}
        (S_1 S_2 \dots S_n)(u) &= 0
\intertext{
    Then follow this multi-step process for $j = 1,\dots,n$.
}
        0 &= (S_j S_{j+1} \dots S_n)(u)             \\
          &= S_j \big( (S_{j+1} \dots S_n)(u) \big)
\end{align*}
    At the end of each step, we know $(S_{j+1} \dots S_n)(u) = 0$ because $S_j$ is injective.

    At the $n$th step, we have $S_nu = 0$ and thus $u = 0$.

    Hence $S_1S_2\dots S_n$ is injective because its null space equals $\{0\}$.
\end{document}
