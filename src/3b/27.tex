\documentclass[a5paper]{article}
\usepackage{amsmath}
\usepackage[top=1cm,right=1cm,bottom=1cm,left=1cm]{geometry}
\setlength\parindent{0pt}
\setlength\parskip{1em}
%
%\usepackage{xcolor}
%\pagecolor[rgb]{0.1,0.1,0.1}
%\color[rgb]{1.0,1.0,1.0}
%
\begin{document}
\newcommand   \C           {\mathbf{C}}
\newcommand   \R           {\mathbf{R}}
\renewcommand \L           {\mathcal{L}}
\newcommand   \F           {\mathbf{F}}
\renewcommand \P           {\mathcal{P}}
\newcommand   \M           {\mathcal{M}}
\newcommand   \question[1] {\textbf{\boldmath#1\unboldmath}\par}
\newcommand   \op          {\operatorname}

\question{
    3.B.27.
    Suppose $P \in \L(V)$ and $P^2 = P$.
    Prove that $V = \op{null}P \oplus \op{range}P$.
}

    Suppose $w_1,\dots,w_m$ is a basis of $\op{range}P$.
    Suppose $u_1,\dots,u_n$ is a basis of $\op{null}P$.

    Consider a list with these bases joined together:
\begin{equation*}
        w_1,\dots,w_m, u_1,\dots,u_n .
\end{equation*}
    We will show that this list is a basis of $V$.
    First, from the Fundamental Theorem of Linear Maps,
\begin{align*}
        \op{dim}V &= \op{dim}\op{range}P + \op{dim}\op{null}P \\
                  &= m + n .
\end{align*}
    Hence our list has the correct length to be a basis of $V$.
    So to show that it is a basis, we only need to show that it is linearly independent.

    Suppose $a_1,\dots,a_m,b_1,\dots,b_n \in \F$ such that
\begin{equation*}
        a_1w_1 + \dots + a_mw_m + b_1u_1 + \dots + b_nu_n = 0 .
\end{equation*}
    Then
\begin{equation*}
        a_1w_1 + \dots + a_mw_m = -b_1u_1 - \dots - b_nu_n .
\end{equation*}
    Applying $P$ to both sides of the equation:
\begin{align*}
        P(a_1w_1 + \dots + a_mw_m) &= P(-b_1u_1 - \dots - b_nu_n) \\
         a_1Pw_1 + \dots + a_mPw_m &= -b_1Pu_1 - \dots - b_nPu_n  \\
                                   &= 0 .
\end{align*}
    The right side of the equation becomes 0 because each $Pu_j = 0$, because $u_j \in \op{null}P$.

    Now, for $k \in \{ 1,\dots,m \}$, since $w_k \in \op{range}P$, there exists some $v_k \in V$ such that
\begin{equation*}
        w_k = Pv_k
\end{equation*}
    Applying $P$ to both sides of the equation:
\begin{align*}
        Pw_k &= P^2v_k \\
             &= Pv_k \\
             &= w_k
\end{align*}
    where the second equality holds because $P = P^2$.

    Thus we can write each $Pw_k$ more simply as $w_k$.
    Hence the equation becomes
\begin{equation*}
        a_1w_1 + \dots + a_mw_m = 0 .
\end{equation*}
    This is a linear combination of a linearly independent list, so $a_1 = \dots = a_m = 0$.
    So the equation further up becomes
\begin{align*}
        -b_1u_1 - \dots - b_nu_n &= a_1w_1 + \dots + a_mw_m \\
                                 &= 0 .
\end{align*}
    This is also a linear combination of a linearly independent list, so $b_1 = \dots = b_n = 0$.

    So our list $w_1,\dots,w_m,u_1,\dots,u_n$ is indeed a basis of $V$.
    Thus any $v \in V$ has a unique representation as a linear combination of this list:
\begin{equation*}
        v = c_1w_1 + \dots + c_mw_m + d_1u_1 + \dots + d_nu_n
\end{equation*}
    with $c_1,\dots,c_m,d_1,\dots,d_n \in \F$.

    Hence any vector in $V$ has a unique representation as a sum of a vector in $\op{range}P$ and a vector in $\op{null}P$.
    Thus
\begin{equation*}
        V = \op{null}P \oplus \op{range}P .
\end{equation*}
\end{document}
