\documentclass[a5paper]{article}
\usepackage{amsmath}
\usepackage[top=1cm,right=1cm,bottom=1cm,left=1cm]{geometry}
\setlength\parindent{0pt}
\setlength\parskip{1em}
%
%\usepackage{xcolor}
%\pagecolor[rgb]{0.1,0.1,0.1}
%\color[rgb]{1.0,1.0,1.0}
%
\begin{document}
\newcommand   \C           {\mathbf{C}}
\newcommand   \R           {\mathbf{R}}
\renewcommand \L           {\mathcal{L}}
\newcommand   \F           {\mathbf{F}}
\renewcommand \P           {\mathcal{P}}
\newcommand   \M           {\mathcal{M}}
\newcommand   \question[1] {\textbf{\boldmath#1\unboldmath}\par}
\newcommand   \op          {\operatorname}

\question{
    3.B.27.
    Suppose $P \in \L(V)$ and $P^2 = P$.
    Prove that $V = \op{null}P \oplus \op{range}P$.
}

    For $v \in V$,
\begin{align*}
        Pv        &= P^2v \\
        Pv - P^2v &= 0 \\
       P(v - Pv)  &= 0 .
\end{align*}
    Thus $v - Pv \in \op{null}P$.

    Hence we can write $v$ as a sum of a vector in $\op{range}P$ and a vector in $\op{null}P$:
\begin{equation*}
        v = Pv + (v - Pv) .
\end{equation*}
    Thus $V = \op{null}P + \op{range}P$.

    To complete our proof, suppose $w \in \op{range}P \cap \op{null}P$.

    Because $w \in \op{range}P$, there exists $u \in V$ such that
\begin{align*}
        w &= Pu .
\intertext{
    Applying $P$ to both sides,
}
        Pw &= P^2u \\
           &= Pu   \\
           &= w .
\end{align*}
    Thus $Pw = w$.

    $Pw = 0$ because $w \in \op{null}P$.
    Hence $w = 0$.

    Hence $\op{range}P \cap \op{null}P = \{ 0 \}$.
    Thus $V = \op{range}P \oplus \op{null}P$ (by 1.46).
\end{document}
