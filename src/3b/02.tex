\documentclass[a5paper]{article}
\usepackage{amsmath}
\usepackage[top=1cm,right=1cm,bottom=1cm,left=1cm]{geometry}
\setlength\parindent{0pt}
\setlength\parskip{1em}
%
%\usepackage{xcolor}
%\pagecolor[rgb]{0.1,0.1,0.1}
%\color[rgb]{1.0,1.0,1.0}
%
\begin{document}

\newcommand    \C  { \mathbf{C} }
\newcommand    \R  { \mathbf{R} }
\renewcommand  \L  { \mathcal{L} }
\newcommand    \F  { \mathbf{F} }
\renewcommand \P           {\mathcal{P}}
\newcommand   \question[1] {\textbf{\boldmath#1\unboldmath}\par}
\newcommand   \op          {\operatorname}

\question{
    Suppose $S, T \in \L(V,V)$ are such that $\op{range}S \subseteq \op{null}T$.
    Prove that $(ST)^2 = 0$.
}

    Since $\op{range}S \subseteq \op{null}T$, we have $(TS)(v) = 0$ for all $v \in V$.

    Hence,
\begin{align*}
    (ST)^2(v) &= (STST)(v)                      \\
              &= S \big( (TS)(Tv) \big)         \\
              &= S (0)                          \\
              &= 0
\end{align*}
    with the last equality holding because all linear maps take 0 to 0.
\end{document}
