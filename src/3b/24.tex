\documentclass[a5paper]{article}
\usepackage{amsmath}
\usepackage{amssymb}
\usepackage[top=1cm,right=1cm,bottom=1cm,left=1cm]{geometry}
\setlength\parindent{0pt}
\setlength\parskip{1em}
%
%\usepackage{xcolor}
%\pagecolor[rgb]{0.1,0.1,0.1}
%\color[rgb]{1.0,1.0,1.0}
%
\begin{document}
\newcommand   \C           {\mathbf{C}}
\newcommand   \R           {\mathbf{R}}
\renewcommand \L           {\mathcal{L}}
\newcommand   \F           {\mathbf{F}}
\renewcommand \P           {\mathcal{P}}
\newcommand   \M           {\mathcal{M}}
\newcommand   \op          {\operatorname}

    3.B.24.
    (a) Suppose $\op{dim}V = 5$ and $S,T \in \L(V)$ are such that $ST=0$.
    Prove that $\op{dim}\op{range}TS \le 2$.

    To show $\op{range}TS$ is a subspace of $\op{range}T$, suppose $v \in \op{range}TS$.
    Then there exists $u \in V$ such that $(TS)u = v$.
    Then
\begin{equation*}
        v = (TS)u = T(Su) .
\end{equation*}
    Thus $v \in \op{range}T$.
    Hence $\op{range}TS \subseteq \op{range}T$ and so
\begin{equation}
        \op{dim}\op{range}TS \le \op{dim}\op{range}T .
\end{equation}

    To show $\op{range}T$ is a subspace of $\op{null}S$, suppose $v \in \op{range}T$.
    Then there exists $u \in V$ such that $Tu = v$.
    Then
\begin{equation*}
        Sv = S(Tu) = (ST)u = 0 .
\end{equation*}
    Thus $v \in \op{null}S$.
    Hence $\op{range}T \subseteq \op{null}S$ and so
\begin{equation}
        \op{dim}\op{range}T \le \op{dim}\op{null}S .
\end{equation}

    By (1) and (2),
\begin{equation}
        \op{dim}\op{range}TS \le \op{dim}\op{null}S .
\end{equation}

    To show that $\op{null}S$ is a subspace of $\op{null}TS$, suppose $v \in \op{null}S$.
    Then
\begin{equation*}
        (TS)v = T(Sv) = T(0) = 0 .
\end{equation*}
    Thus $v \in \op{null}TS$.
    Hence $\op{null}S \subseteq \op{null}TS$ and so
\begin{equation}
        \op{dim}\op{null}S \le \op{dim}\op{null}TS .
\end{equation}

    By (3) and (4),
\begin{equation}
        \op{dim}\op{range}TS \le \op{dim}\op{null}TS .
\end{equation}

    From the Fundamental Theorem of Linear Maps,
\begin{align*}
        \op{dim}\op{range}TS + \op{dim}\op{null}TS &= 5  \\
        \op{dim}\op{null}TS &= 5 - \op{dim}\op{range}TS .
\end{align*}

    Then by (5),
\begin{align*}
           \op{dim}\op{range}TS &\le 5 - \op{dim}\op{range}TS  \\
        2(\op{dim}\op{range}TS) &\le 5                         \\
           \op{dim}\op{range}TS &\le 2.5 
\intertext{
    so since $\op{dim}\op{range}TS$ is an integer
}
        \op{dim}\op{range}TS &\le 2
\end{align*}
    as desired. $\blacksquare$

    (b) Give an example of $S,T \in \L(\F^5)$ with $ST=0$ and $\op{dim}\op{range}TS=2$.

    Define $S,T \in \L(\F^5)$ by
\begin{align*}
        S(x_1, x_2, x_3, x_4, x_5) &= (0, x_3, x_5, 0, 0) \quad\text{and}\\
        T(x_1, x_2, x_3, x_4, x_5) &= (0, x_2, 0, x_3, 0) .
\end{align*}
    Then
\begin{equation*}
        (ST)(x_1, x_2, x_3, x_4, x_5) = S(0, x_2, 0, x_3, 0) = (0, 0, 0, 0, 0)
\end{equation*}
    so $ST=0$ and
\begin{equation*}
        (TS)(x_1, x_2, x_3, x_4, x_5) = T(0, x_3, x_5, 0, 0) = (0, x_3, 0, x_5, 0) 
\end{equation*}
    so $\op{dim}\op{range}TS=2$.
\end{document}
