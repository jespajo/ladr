\documentclass[a4paper]{article}
\usepackage{amsmath}
\usepackage[top=1cm,right=1cm,bottom=1cm,left=1cm]{geometry}
\setlength\parindent{0pt}
\begin{document}
\large
Let $a \in \mathbf{R}$.
Let $u, v\in \mathbf{R^3} : u=(x_1,y_1,z_1)$ and $v=(x_2,y_2,z_2)$.
We will omit the subscripts when dealing with $u$ alone.
\begin{align*}
\intertext{
First suppose $b=c=0$.
}
    Tu + Tv &= (2x_1-4y_1+3z_1, \; 6x_1) + (2x_2-4y_2+3z_2, \; 6x_2) \\
            &= (2x_1-4y_1+3z_1+2x_2-4y_2+3z_2, \  6x_1+6x_2)     \\
            &= (2(x_1+x_2)-4(y_1+y_2)+3(z_1+z_2), \ 6(x_1+x_2))    \\
            &= T(u + v) \\
\\
    T(au) &= (2(ax) - 4(ay) + 3(az), \  6(ax)) \\
          &= (2ax - 4ay + 3az, \  6ax)  \\
          &= (a(2x - 4y + 3z), \ a(6x))  \\
          &= aTu
\intertext{
Then $T$ is a linear map. Now suppose $T \in \mathcal{L}(\mathbf{R^3}, \mathbf{R^2})$.
}
    Tu + Tv &= T(x_1,y_1,z_1) + T(x_2,y_2,z_2) \\
      &= (2x_1-4y_1+3z_1+b,\; 6x_1+cx_1y_1z_1)\;+\;(2x_2-4y_2+3z_2+b,\; 6x_2+cx_2y_2z_2)  \\
      &= (2x_1-4y_1+3z_1+b + 2x_2-4y_2+3z_2+b,\quad  6x_1+cx_1y_1z_1+6x_2+cx_2y_2z_2) \\
      &= (2(x_1+x_2)-4(y_1+y_2)+3(z_1+z_2)+2b,\quad  6(x_1+x_2)+c(x_1y_1z_1+x_2y_2z_2)) \tag{1} \\
\\
    T(u+v) &= T((x_1,y_1,z_1) + (x_2,y_2,z_2)) \\
        &= T(x_1+x_2,\; y_1+y_2,\; z_1+z_2) \\
    &= (2(x_1+x_2)-4(y_1+y_2)+3(z_1+z_2)+b,\quad 6(x_1+x_2)+c(x_1+x_2)(y_1+y_2)(z_1+z_2)) \tag{2}
\\
\\
    T(au) &= T(ax, ay, az) \\
          &= (2(ax)-4(ay)+3(az)+b, \  6(ax)+c(ax)(ay)(az)) \\
          &= (2ax-4ay+3az+b, \  6ax+ca^3xyz) \tag{3} \\
\\
    aTu &= a(2x-4y+3z+b, \   6x+cxyz) \\
        &= (a(2x-4y+3z+b), \   a(6x+cxyz)) \\
        &= (2ax-4ay+3az+ab, \   6ax+caxyz) \tag{4}
\end{align*}
The additivity of $T$ implies that the first coordinates of (1) and (2) are the same:
\begin{align*}
    2(x_1+x_2)-4(y_1+y_2)+3(z_1+z_2)+2b &= 2(x_1+x_2)-4(y_1+y_2)+3(z_1+z_2)+b \\
        2b &= b \\
        b &= 0
\intertext{
The homogeneity of $T$ implies that the second coordinates of (3) and (4) are the same:
}
    6ax+ca^3xyz &= 6ax+caxyz \\
        ca^3xyz &= caxyz
\intertext{
which is true for any $c \in \mathbf{R}$ if any one of $a, x, y$ or $z$ is 0.
But the equation is true for all $a$, $x$, $y$, $z \in \mathbf{R}$.
If none of $a$, $x$, $y$, $z$ is 0:
}
      \frac{ca^3xyz}{axyz} &= \frac{caxyz}{axyz} \\
            ca^2 &= c \\
            c &= 0
\end{align*}
\end{document}










