\documentclass[a5paper]{article}
\usepackage{amsmath}
\usepackage{amssymb}
\usepackage[top=1cm,right=1cm,bottom=2cm,left=1cm]{geometry}
\setlength\parindent{0pt}
\setlength\parskip{1em}
%
\begin{document}
\newcommand   \C           {\mathbf{C}}
\newcommand   \R           {\mathbf{R}}
\renewcommand \L           {\mathcal{L}}
\newcommand   \F           {\mathbf{F}}
\renewcommand \P           {\mathcal{P}}
\newcommand   \M           {\mathcal{M}}
\newcommand   \op          {\operatorname}

    3.D.8.
    Suppose $V$ and $W$ are finite-dimensional and $S,T \in \L(V,W)$.
    Prove that there exist invertible $E_1 \in \L(V)$ and $E_2 \in \L(W)$ such that $S = E_2TE_1$ if and only if $\op{dim}\op{null}S = \op{dim}\op{null}T$.
    \\

    First, suppose there exist invertible $E_1 \in \L(V)$ and $E_2 \in \L(W)$ such that $S = E_2 T E_1$.

    Using the result of exercise 3.B.22, we have
\begin{align*}
        \op{dim}\op{null}(E_2TE_1) &\le \op{dim}\op{null}(E_2 T) + \op{dim}\op{null}E_1 \\
                                   &=   \op{dim}\op{null}(E_2 T) .
\end{align*}
    $\op{dim}\op{null}E_1$ disappears because $E_1$ is invertible so its null space is $\{0\}$.

    Replacing $E_2TE_1$ with $S$, and applying the same results again, we then have
\begin{align*}
        \op{dim}\op{null}S &\le \op{dim}\op{null}(E_2 T) \\
                           &\le \op{dim}\op{null}E_2 + \op{dim}\op{null}T \\
                           &= \op{dim}\op{null}T .
\end{align*}
    Next, take our original equation, $S = E_2TE_1$, right-multiply both sides by $E_1^{-1}$, and left-multiply both sides by $E_2^{-1}$ to get
\begin{align*}
        E_2^{-1} S E_1^{-1} &= T .
\end{align*}
    Since this is equivalent to our original equation with the roles of $S$ and $T$ interchanged, we can apply the steps above with the roles of $S$ and $T$ interchanged to find that
\begin{align*}
        \op{dim}\op{null}T \le \op{dim}\op{null}S.
\end{align*}
    Thus $\op{dim}\op{null}S = \op{dim}\op{null}T$, completing the proof in one direction.

    To complete the proof in the other direction, suppose $\op{dim}\op{null}S = \op{dim}\op{null}T$.

    Then suppose $u_1,\dots,u_m$ is a basis of $\op{null}S$.
    Extend this list to a basis of $V$ by appending $v_1,\dots,v_n$.

    Suppose $w_1,\dots,w_m$ is a basis of $\op{null}T$.
    Extend this list to a basis of $V$ by appending $x_1,\dots,x_n$.


    Define $E_1 \in \L(V)$ as a linear map such that for $j=1,\dots,m$ and for $k=1,\dots,n$,
\begin{align*}
        E_1u_j = w_j \quad\text{and}\quad E_1v_k = x_k .
\end{align*}
    $E_1$ is invertible because it takes each vector in one basis of $V$ to a vector in another basis of $V$ (see the end of the previous exercise for a more in-depth proof of this lemma).


    We will now show that $Sv_1,\dots,Sv_n$ is a basis of $\op{range}S$ and that $Tx_1,\dots,Tx_n$ is a basis of $\op{range}T$.

    From the fundamental theorem of linear maps,
\begin{align*}
        \op{dim}V   &= \op{dim}\op{null}S + \op{dim}\op{range}S \\
                    &= \op{dim}\op{null}T + \op{dim}\op{range}T .
\end{align*}
    Since the null spaces of $S$ and $T$ both have dimension $m$, and the dimension of $V$ is $(m+n)$, we have
\begin{align*}
        \op{dim}\op{range}S = \op{dim}\op{range}T = n .
\end{align*}
    Now suppose $w \in \op{range}S$.
    Then there exists $u \in V$ such that $Su = w$.
    Since $u \in V$, there exist $a_1,\dots,a_m,b_1,\dots,b_n \in \F$ such that\
\begin{align*}
        u &= a_1u_1 + \dots + a_mu_m + b_1v_1 + \dots + b_nv_n .
\end{align*}
    Applying $S$ to both sides,
\begin{align*}
        Su &= b_1Sv_1 + \dots + b_nSv_n .
\end{align*}
    The $a_jSu_j$ terms disappear because each $u_j \in \op{null}S$.
    Hence any vector in the range of $S$ can be written as a linear combination of $Sv_1,\dots,Sv_n$.
    So this list is a basis of $\op{range}S$ because it is a spanning list of the right length.

    Repeat this logic with $Tx_1,\dots,Tx_n$ instead of $Sv_1,\dots,Sv_n$ to see that $Tx_1,\dots,Tx_n$ is a basis of $\op{range}T$.

    Suppose we extend $Sv_1,\dots,Sv_n$ to a basis of $W$ by appending $y_1,\dots,y_m \in \F$.
    Suppose we extend $Tx_1,\dots,Tx_n$ to a basis of $W$ by appending $z_1,\dots,z_m \in \F$.

    Now that we have two bases of $W$, we can define an invertible $E_2 \in \L(W)$ such that for $k=1,\dots,n$ and $j=1,\dots,m$,
\begin{align*}
        E_2(Tx_k) = Sv_k \quad\text{and}\quad E_2z_j = y_j .
\end{align*}
    To complete our proof, we need to show that $S = E_2TE_1$.
    Suppose $v \in V$.
    Then there exist $c_1,\dots,c_m,d_1,\dots,d_m \in \F$ such that
\begin{align*}
        v &= c_1u_1 + \dots + c_mu_m + d_1v_1 + \dots + d_nv_n .
\end{align*}
    Applying $S$ to $v$,
\begin{align*}
        Sv &= S(c_1u_1 + \dots + c_mu_m + d_1v_1 + \dots + d_nv_n) \\
           &= d_1Sv_1 + \dots + d_nSv_n .
\end{align*}
    Applying $E_2TE_1$ to $v$,
\begin{align*}
        E_2TE_1v &= E_2T(c_1E_1u_1 + \dots + c_mE_1u_m + d_1E_1v_1 + \dots + d_nE_1v_n) \\
                 &= E_2T(c_1w_1 + \dots + c_mw_m + d_1x_1 + \dots + d_nx_n) \\
                 &= E_2(c_1Tw_1 + \dots + c_mTw_m + d_1Tx_1 + \dots + d_nTx_n) \\
                 &= E_2(d_1Tx_1 + \dots + d_nTx_n) \\
                 &= d_1E_2Tx_1 + \dots + d_nE_2Tx_n \\
                 &= d_1Sv_1 + \dots + d_nSv_n \\
                 &= Sv .
\end{align*}
    Thus $S = E_2TE_1$ as desired.
\end{document}
