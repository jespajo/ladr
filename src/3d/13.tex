\documentclass[a5paper]{article}
\usepackage{amsmath}
\usepackage{amssymb}
\usepackage[top=1cm,right=1cm,bottom=2cm,left=1cm]{geometry}
\setlength\parindent{0pt}
\setlength\parskip{1em}
%
\begin{document}
\newcommand   \C           {\mathbf{C}}
\newcommand   \R           {\mathbf{R}}
\renewcommand \L           {\mathcal{L}}
\newcommand   \F           {\mathbf{F}}
\renewcommand \P           {\mathcal{P}}
\newcommand   \M           {\mathcal{M}}
\newcommand   \E           {\mathcal{E}}
\newcommand   \op          {\operatorname}

    3.D.13.
    Show that the result in Exercise 12 can fail without the hypothesis that $V$ is finite-dimensional.

    Define $S,T,U \in \L(\F^{\infty})$ as follows.
    $S$ and $T$ are the backwards-shift linear map.
    $U$ is the forward-shift-by-two linear map.
    So
\begin{align*}
        S(x_1,x_2,x_3,\dots) &= (x_2,x_3,\dots) . \\
        T(x_1,x_2,x_3,\dots) &= (x_2,x_3,\dots) . \\
        U(x_1,x_2,x_3,\dots) &= (0,0,x_1,x_2,x_3,\dots) .
\end{align*}
    Then
\begin{align*}
        STU(x_1,x_2,x_3,\dots) &= ST(0,0,x_1,x_2,x_3,\dots) \\
                               &= S(0,x_1,x_2,x_3,\dots) \\
                               &= (x_1,x_2,x_3,\dots) .
\end{align*}
    Thus $STU = I$.

    But
\begin{align*}
        TUS(x_1,x_2,x_3,\dots) &= TU(x_2,x_3,\dots) \\
              &= T(0,0,x_2,x_3,\dots) \\
              &= (0,x_2,x_3,\dots) .
\end{align*}
    So $TUS \neq I$.
    Hence $US$ is not the inverse of $T$.
\end{document}
