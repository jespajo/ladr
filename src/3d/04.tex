\documentclass[a5paper]{article}
\usepackage{amsmath}
\usepackage{amssymb}
\usepackage[top=1cm,right=1cm,bottom=2cm,left=1cm]{geometry}
\setlength\parindent{0pt}
\setlength\parskip{1em}
%
\begin{document}
\newcommand   \C           {\mathbf{C}}
\newcommand   \R           {\mathbf{R}}
\renewcommand \L           {\mathcal{L}}
\newcommand   \F           {\mathbf{F}}
\renewcommand \P           {\mathcal{P}}
\newcommand   \M           {\mathcal{M}}
\newcommand   \op          {\operatorname}

    3.D.4.
    Suppose $V$ is finite-dimensional and $\op{dim}V > 1$.
    Prove that the set of noninvertible linear maps from $V$ to itself is not a subspace of $\L(V)$.

    Suppose $v_1,\dots,v_n$ is a basis of $V$.
    Thus $n > 1$.

    %Define $T \in \L(V)$ such that $Tv_1 = 0$ and $Tv_j = v_j$ for $j = 2,\dots,n$.
    Define $T \in \L(V)$ such that $Tv_1 = 0$ and $Tv_2 = v_2, \dots, Tv_n = v_n$.

    %Define $S \in \L(V)$ such that $Sv_1 = v_1$ and $Sv_j = 0$ for $j = 2,\dots,n$.
    Define $S \in \L(V)$ such that $Sv_1 = v_1$ and $Sv_2 = \dots = Sv_n = 0$.

    These are valid definitions of linear maps because $n > 1$ and because they follow the structure given in the linear map lemma.

    Clearly $v_1 \neq 0$ because it is in a basis of $V$.
    So, since $v_1 \in \op{null}T$, we have $\op{null}T \neq \{0\}$.
    Thus $T$ is not injective.
    Hence $T$ is noninvertible.

    Similarly, $v_2 \in \op{null}S$, so, following the same logic, $S$ is noninvertible.

    The sum of $S$ and $T$ is the identity map on $V$.
    To see this, suppose $v \in V$.
    Then there exist $a_1,\dots,a_n \in \F$ such that
\begin{align*}
        v &= a_1 v_1 + \dots + a_n v_n .
\end{align*}
    Then
%\begin{align*}
%        (S + T) v   &= Sv + Tv \\
%                    &= S(a_1v_1 + \dots + a_nv_n) + T(a_1v_1 + \dots + a_nv_n) \\
%                    &= a_1Sv_1 + \dots + a_nSv_n + a_1Tv_1 + \dots + a_nTv_n \\
%                    &= a_1v_1 + \dots + a_nv_n \\
%                    &= v .
%\end{align*}
\begin{align*}
        (S + T) v   &= (S + T)( a_1 v_1 + \dots + a_n v_n ) \\
                    &= a_1(S + T)v_1 + \dots + a_n(S + T)v_n \\
                    &= a_1(Sv_1 + Tv_1) + \dots + a_n(Sv_n + Tv_n) \\
                    &= a_1(v_1 + 0) + a_2(0 + v_2) + \dots + a_n(0 + v_n) \\
                    &= a_1v_1 + \dots + a_nv_n \\
                    &= v .
\end{align*}
    Thus $S + T = I$.
    $I$ is invertible because it has an inverse, $I$ (since $II = I$).

    To recap, $S$ and $T$ are both noninvertible linear maps from $V$ to itself.
    Their sum is an invertible linear map.
    Thus the set of noninvertible linear maps from $V$ to itself is not closed under addition.
    Therefore it is not a subspace of $\L(V)$.
\end{document}
