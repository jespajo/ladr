\documentclass[a5paper]{article}
\usepackage{amsmath}
\usepackage{amssymb}
\usepackage[top=1cm,right=1cm,bottom=2cm,left=1cm]{geometry}
\setlength\parindent{0pt}
\setlength\parskip{1em}
%
\begin{document}
\newcommand   \C           {\mathbf{C}}
\newcommand   \R           {\mathbf{R}}
\renewcommand \L           {\mathcal{L}}
\newcommand   \F           {\mathbf{F}}
\renewcommand \P           {\mathcal{P}}
\newcommand   \M           {\mathcal{M}}
\newcommand   \E           {\mathcal{E}}
\newcommand   \op          {\operatorname}

    3.D.15.
    Suppose $T \in \L(V)$ and $v_1,\dots,v_m$ is a list in $V$ such that $Tv_1,\dots,Tv_m$ spans $V$.
    Prove that $v_1,\dots,v_m$ spans $V$.

    Since $Tv_1,\dots,Tv_m$ spans $V$, we can reduce this list to a basis of $V$.
    Let $Tu_1,\dots,Tu_n$ denote the vectors in the basis.

    Suppose $a_1,\dots,a_n \in \F$ such that
\begin{align*}
        0 &= a_1u_1 + \dots + a_nu_n .
\end{align*}
    Applying $T$ to both sides,
\begin{align*}
        0 &= T( a_1u_1 + \dots + a_nu_n ) \\
          &= a_1Tu_1 + \dots + a_nTu_n .
\end{align*}
    Since this is a linear combination of a linearly independent list, this implies $a_1=\dots=a_n=0$.

    Thus $u_1,\dots,u_n$ is a linearly independent list.

    Hence $u_1,\dots,u_n$ is a basis of $V$ because it is a linearly independent list of the right length.

    So $v_1,\dots,v_m$ spans $V$ because it contains all of the $u$'s, and the $u$'s span $V$.
\end{document}
