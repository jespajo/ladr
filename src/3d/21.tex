\documentclass[a5paper]{article}
\usepackage{amsmath}
\usepackage{amssymb}
\usepackage[top=1cm,right=1cm,bottom=2cm,left=1cm]{geometry}
\setlength\parindent{0pt}
\setlength\parskip{1em}
\begin{document}
\newcommand   \C           {\mathbf{C}}
\newcommand   \R           {\mathbf{R}}
\renewcommand \L           {\mathcal{L}}
\newcommand   \F           {\mathbf{F}}
\renewcommand \P           {\mathcal{P}}
\newcommand   \M           {\mathcal{M}}
\newcommand   \E           {\mathcal{E}}
\newcommand   \op          {\operatorname}
\newcommand   \A           {\mathcal{A}}
\newcommand   \Q           {\mathcal{Q}}

    3.D.21.
    Suppose $n$ is a positive integer and $A_{j,k} \in \F$ for all $j,k=1,\dots,n$.
    Prove the following are equivalent (note that in both parts below, the number of equations equals the number of variables).\\
    (a) The trivial solution $x_1=\dots=x_n=0$ is the only solution to the homogeneous system of equations
\begin{align*}
        \sum_{k=1}^n A_{1,k}x_k &= 0 \\
        &\vdots \\
        \sum_{k=1}^n A_{n,k}x_k &= 0.
\end{align*}
    (b) For every $c_1,\dots,c_n \in \F$, there exists a solution to the system of equations
\begin{align*}
        \sum_{k=1}^n A_{1,k}x_k &= c_1 \\
        &\vdots \\
        \sum_{k=1}^n A_{n,k}x_k &= c_n.
\end{align*}

    Consider $A$ as an $n$-by-$n$ matrix such that $A_{j,k}$ is the entry in row $j$, column $k$ of $A$.

    Suppose $T \in \L(\F^n)$ such that $\M(T)=A$, with $\M(T)$ defined with respect to the standard basis of $\F^n$.
    The existence and uniqueness of $T$ come from the fact that $\M$ is an isomorphism.

    We can restate both parts of the exercise as statements about the equation
\begin{align*}
        A
        \begin{pmatrix}
            x_1 \\
            \vdots \\
            x_n
        \end{pmatrix}
        =
        \begin{pmatrix}
            c_1 \\
            \vdots \\
            c_n
        \end{pmatrix} .
\end{align*}
    Since linear maps act like matrix multiplication, we can also write this equation as
\begin{align*}
        T(x_1,\dots,x_n) = (c_1,\dots,c_n) .
\end{align*}
    Part (a) is the statement that if the $c$'s are all $0$, then the $x$'s are all $0$.
    In other words, $T$ is injective.

    Part (b) is the statement that no matter what values we are given for the $c$'s, we can find values for the $x$'s to satisfy the equation.
    In other words, $T$ is surjective.

    Since $T$ is a linear map whose domain and target vector spaces have the same finite dimension, the injectivity and surjectivity of $T$ are equivalent.
    Hence (a) and (b) are equivalent.
\end{document}
