\documentclass[a5paper]{article}
\usepackage{amsmath}
\usepackage{amssymb}
\usepackage[top=1cm,right=1cm,bottom=2cm,left=1cm]{geometry}
\setlength\parindent{0pt}
\setlength\parskip{1em}
%
\begin{document}
\newcommand   \C           {\mathbf{C}}
\newcommand   \R           {\mathbf{R}}
\renewcommand \L           {\mathcal{L}}
\newcommand   \F           {\mathbf{F}}
\renewcommand \P           {\mathcal{P}}
\newcommand   \M           {\mathcal{M}}
\newcommand   \E           {\mathcal{E}}
\newcommand   \op          {\operatorname}

    3.D.11.
    Suppose $V$ is finite-dimensional and $S,T \in \L(V)$.
    Prove that
\begin{align*}
        ST\ \text{is invertible}\ \Longleftrightarrow\ S\ \text{and}\ T\ \text{are invertible} .
\end{align*}
    \hfill

    Suppose $S$ and $T$ are invertible.
    Then
\begin{align*}
        (T^{-1} S^{-1}) (ST) = T^{-1} (S^{-1}S) T = T^{-1}I T = T^{-1} T = I .
\end{align*}
    Thus $T^{-1}S^{-1}$ is the inverse of $ST$, completing the proof in one direction.

    To prove in the other direction, suppose $ST$ is invertible.
    Then $ST$ is surjective.
    So
\begin{align*}
        \op{dim}\op{range}ST = \op{dim}V .
\end{align*}
    Using the result of exercise 3.B.23,
\begin{align*}
        \op{dim}\op{range}ST  &\le \op{min}\{ \op{dim}\op{range}S, \op{dim}\op{range}T \} .
\end{align*}
    Thus
\begin{align*}
        \op{dim}\op{range}S \ge \op{dim}V \quad\text{and}\quad \op{dim}\op{range}T \ge \op{dim}V .
\end{align*}
    The ranges of $S$ and $T$ are subspaces of $V$.
    So their dimensions cannot be greater than the dimension of $V$.
    Hence
\begin{align*}
        \op{dim}\op{range}S = \op{dim}V \quad\text{and}\quad \op{dim}\op{range}T = \op{dim}V .
\end{align*}
    Thus $S$ and $T$ are surjective.
    Hence they are invertible because they are surjective operators on a finite-dimensional vector space.
\end{document}
