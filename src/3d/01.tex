\documentclass[a5paper]{article}
\usepackage{amsmath}
\usepackage{amssymb}
\usepackage[top=1cm,right=1cm,bottom=2cm,left=1cm]{geometry}
\setlength\parindent{0pt}
\setlength\parskip{1em}
%
\begin{document}
\newcommand   \C           {\mathbf{C}}
\newcommand   \R           {\mathbf{R}}
\renewcommand \L           {\mathcal{L}}
\newcommand   \F           {\mathbf{F}}
\renewcommand \P           {\mathcal{P}}
\newcommand   \M           {\mathcal{M}}
\newcommand   \op          {\operatorname}


    3.D.1.
    Suppose $T \in \L(V,W)$ is invertible.
    Show that $T^{-1}$ is invertible and
\begin{align*}
        \big(T^{-1}\big)^{-1} &= T .
\end{align*}
    Since $T$ is invertible, $T^{-1}$ is the unique element of $\L(W,V)$ such that
\begin{align*}
        T^{-1} T = I \ \ \text{and}\ \ T T^{-1} = I .
\end{align*}
    The above equations imply that $T^{-1}$ is invertible because it has an inverse, $T$.
\end{document}
