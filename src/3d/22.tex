\documentclass[a5paper]{article}
\usepackage{amsmath}
\usepackage{amssymb}
\usepackage[top=1cm,right=1cm,bottom=2cm,left=1cm]{geometry}
\setlength\parindent{0pt}
\setlength\parskip{1em}
\begin{document}
\newcommand   \C           {\mathbf{C}}
\newcommand   \R           {\mathbf{R}}
\renewcommand \L           {\mathcal{L}}
\newcommand   \F           {\mathbf{F}}
\renewcommand \P           {\mathcal{P}}
\newcommand   \M           {\mathcal{M}}
\newcommand   \E           {\mathcal{E}}
\newcommand   \op          {\operatorname}
\newcommand   \A           {\mathcal{A}}
\newcommand   \Q           {\mathcal{Q}}

    3.D.22.
    Suppose $T \in \L(V)$ and $v_1,\dots,v_n$ is a basis of $V$.
    Prove that
\begin{align*}
        \M\big(T, (v_1,\dots,v_n) \big)\ \text{is invertible}\ \Longleftrightarrow\ T\ \text{is invertible}.
\end{align*}
    First suppose $A = \M\big(T, (v_1,\dots,v_n) \big)$ is invertible.

    Then there exists an $n$-by-$n$ matrix $A^{-1}$ such that $AA^{-1} = I$.
    Let $S \in \L(V)$ denote the linear operator on $V$ such that $A^{-1} = \M\big( S, (v_1,\dots,v_n) \big)$.

    $T$ is surjective.
    To see this, suppose $v \in V$.
    Then
\begin{align*}
        \M(v) &= \M(Iv) \\
              &= \M(I)\M(v) \\
              &= A A^{-1} \M(v) \\
              &= \M(T) \M(S) \M(v) \\
              &= \M(T) \M(Sv) \\
              &= \M\big(T(Sv)\big) .
\end{align*}
    Since $\M$ is an isomorphism, $\M$ is injective.
    Hence the above equation implies that $v=T(Sv)$.
    Thus $v \in \op{range}T$.

    Since $T$ is a surjective operator on a finite-dimensional vector space, $T$ is invertible, completing the proof in one direction.

    To prove in the other direction, suppose $T$ is invertible.

    Then $Tv_1,\dots,Tv_n$ is a basis of $V$.
    To see this, suppose $a_1,\dots,a_n \in \F$ such that $a_1Tv_1 + \dots + a_nTv_n = 0$.
    Then $T(a_1v_1 + \dots + a_nv_n) = 0$.
    Since $T$ is invertible, $T$ is injective, so $\op{null}T = \{0\}$.
    Thus $a_1v_1 + \dots + a_nv_n = 0$.
    So we have $a_1=\dots=a_n=0$.
    Hence the list $Tv_1,\dots,Tv_n$ is linearly independent.
    Therefore the list is a basis of $V$ because it is also the right length.

    Let $A = \M\big(T, (v_1,\dots,v_n) \big)$.
    For each $k=1,\dots,n$, the $k$th column of $A$ implies that
\begin{align*}
        Tv_k = A_{1,k}v_1 + \dots + A_{n,k}v_n .
\end{align*}
    Replacing the left side of the equation with $I(Tv_k)$ shows that $A$ is also the matrix of the identity operator with respect to two different bases:
\begin{align*}
        A = \M\big( I, (Tv_1,\dots,Tv_n), (v_1,\dots,v_n) \big) .
\end{align*}
    We know that the matrix of the identity operator with respect to any two bases is invertible (from 3.82).
    Thus $A$ is invertible, as desired.
\end{document}
