\documentclass[a5paper]{article}
\usepackage{amsmath}
\usepackage{amssymb}
\usepackage[top=1cm,right=1cm,bottom=2cm,left=1cm]{geometry}
\setlength\parindent{0pt}
\setlength\parskip{1em}
%
\begin{document}
\newcommand   \C           {\mathbf{C}}
\newcommand   \R           {\mathbf{R}}
\renewcommand \L           {\mathcal{L}}
\newcommand   \F           {\mathbf{F}}
\renewcommand \P           {\mathcal{P}}
\newcommand   \M           {\mathcal{M}}
\newcommand   \op          {\operatorname}

    3.D.6.
    Suppose that $W$ is finite-dimensional and $S,T \in \L(V,W)$.
    Prove that $\op{null}S = \op{null}T$ if and only if there exists an invertible $E \in \L(W)$ such that $S = ET$.

    To prove in one direction, suppose there exists an invertible $E \in \L(W)$ such that $S = ET$.
    We must prove $\op{null}S = \op{null}T$.

    To prove $\op{null}S \subseteq \op{null}T$, suppose $v \in \op{null}S$.
    Then $Sv = 0$.
    So $ETv = 0$.

    Since $E$ is invertible and $W$ is finite-dimensional, $E$ is injective.
    So $\op{null}E = \{0\}$.
    Hence $Tv = 0$.
    Thus $v \in \op{null}T$.
    So $\op{null}S \subseteq \op{null}T$.

    To prove $\op{null}T \subseteq \op{null}S$, suppose $u \in \op{null}T$.
    Then $Tu = 0$.
    Hence
\begin{align*}
        Su = ETu = E0 = 0 .
\end{align*}
    Thus $u \in \op{null}S$.
    So $\op{null}T \subseteq \op{null}S$.

    Hence $\op{null}S = \op{null}T$, completing our proof in one direction.

    To prove in the other direction, suppose $\op{null}S = \op{null}T$.
    We must prove there exists an invertible $E \in \L(W)$ such that $S = ET$.

    Since $W$ is finite-dimensional, $\op{range}S$ is finite-dimensional.
    So $\op{range}S$ has a basis.
    Suppose $Sv_1,\dots,Sv_n$ is a basis of $\op{range}S$, with some $v_1,\dots,v_n \in V$.

    The list $Tv_1,\dots,Tv_n$ is linearly independent.
    To see this, suppose $a_1,\dots,a_n \in \F$ such that
\begin{align*}
        a_1 Tv_1 + \dots + a_n T v_n = 0 .
\end{align*}
    Then
\begin{align*}
        T(a_1 v_1 + \dots + a_n v_n) = 0 .
\end{align*}
    Thus $a_1v_1+\dots+a_nv_n \in \op{null}T$.

    Since $\op{null}T = \op{null}S$, this means $a_1v_1+\dots+a_nv_n \in \op{null}S$.
    Thus
\begin{align*}
        0 &= S(a_1 v_1 + \dots + a_n v_n) \\
          &= a_1 Sv_1 + \dots + a_n Sv_n .
\end{align*}
    Since $Sv_1,\dots,Sv_n$ is a linearly independent list, we conclude $a_1=\dots=a_n=0$.
    Thus $Tv_1,\dots,Tv_n$ is linearly independent.

    %In fact $Tv_1,\dots,Tv_n$ is a basis of $\op{range}T$, but we don't need this for our proof, but here it is:
    %Since $Tv_1,\dots,Tv_n$ is a linearly independent list of vectors in $\op{range}T$ of length $n = \op{dim}\op{range}S$, we conclude $\op{dim}\op{range}T \ge \op{dim}\op{range}S$.
    %We can repeat the same steps with the roles of $S$ and $T$ interchanged to show that $\op{dim}\op{range}S \ge \op{dim}\op{range}T$.
    %Thus $\op{dim}\op{range}T = \op{dim}\op{range}S$.
    %Hence $Tv_1,\dots,Tv_n$ is a basis of $\op{range}T$ because it is a linearly independent list of the right length.

    Suppose we extend $Tv_1,\dots,Tv_n$ to a basis of $W$ by appending $w_1,\dots,w_m \in W$.
    Suppose we extend $Sv_1,\dots,Sv_n$ to a basis of $W$ by appending $u_1,\dots,u_m \in W$.
    Now we have two bases of $W$:
\begin{align*}
        Tv_1,\dots,Tv_n,w_1,\dots,w_m \quad\text{and}\quad Sv_1,\dots,Sv_n,u_1,\dots,u_m .
\end{align*}

    Define $E \in \L(W)$ as the linear map that takes each vector in the first basis to the corresponding vector in the second basis.
    In other words, for $j = 1,\dots,n$ and for $k = 1,\dots,m$,
\begin{align*}
        ETv_j = Sv_j \quad\text{and}\quad Ew_k = u_k .
\end{align*}
    $E$ is linear because we defined it using the structure from the linear map lemma.
    So we need to show that $S = ET$ and that $E$ is invertible.

    To see that $S = ET$, suppose $v \in V$.
    We cannot define $v$ as a linear combination of a basis of $V$, because we do not know whether $V$ is finite-dimensional.
    But we can find $Sv$ and $Tv$ in terms of our bases of $W$, as follows.

    Since $Sv \in \op{range}S$, there exist $b_1,\dots,b_n \in \F$ such that
\begin{align*}
        Sv = b_1 Sv_1 + \dots + b_n Sv_n .
\end{align*}
    Subtracting $Sv$ from both sides,
\begin{align*}
        0 &= b_1 Sv_1 + \dots + b_n Sv_n - Sv \\
          &= S( b_1 v_1 + \dots + b_n v_n - v ) .
\end{align*}
    Thus $b_1v_1+\dots+b_nv_n-v \in \op{null}S$.

    Since $\op{null}S = \op{null}T$, this means $b_1v_1+\dots+b_nv_n-v \in \op{null}T$.
    Thus
\begin{align*}
        0 &= T( b_1 v_1 + \dots + b_n v_n - v ) \\
          &= b_1 T v_1 + \dots + b_n T v_n - Tv .
\end{align*}
    Adding $Tv$ to both sides,
\begin{align*}
        Tv &= b_1 T v_1 + \dots + b_n T v_n .
\end{align*}

    Hence
\begin{align*}
        Sv &= b_1Sv_1 + \dots + b_nSv_n \\
           &= b_1ETv_1 + \dots + b_nETv_n \\
           &= E(b_1Tv_1 + \dots + b_nTv_n) \\
           &= ETv .
\end{align*}
    Thus $S = ET$.

    To complete our proof, we need to show that $E$ is invertible.
    The easiest way to do this is to show that $E$ is surjective, which is sufficient because $E \in \L(W)$ and $W$ is finite-dimensional.

    Suppose $w \in W$.
    Then there exist $c_1,\dots,c_n,d_1,\dots,d_m \in \F$ such that
\begin{align*}
        w &= c_1Sv_1 + \dots + c_nSv_n + d_1u_1 + \dots + d_mu_m \\
          &= c_1ETv_1 + \dots + c_nETv_n + d_1Ew_1 + \dots + d_mEw_m \\
          &= E(c_1Tv_1 + \dots + c_nTv_n + d_1w_1 + \dots + d_mw_m ) .
\end{align*}
    Thus $w \in \op{range}E$.
    Hence $E$ is surjective.
    Hence $E$ is invertible, as desired.
\end{document}
