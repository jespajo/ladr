\documentclass[a5paper]{article}
\usepackage{amsmath}
\usepackage{amssymb}
\usepackage[top=1cm,right=1cm,bottom=2cm,left=1cm]{geometry}
\setlength\parindent{0pt}
\setlength\parskip{1em}
%
\begin{document}
\newcommand   \C           {\mathbf{C}}
\newcommand   \R           {\mathbf{R}}
\renewcommand \L           {\mathcal{L}}
\newcommand   \F           {\mathbf{F}}
\renewcommand \P           {\mathcal{P}}
\newcommand   \M           {\mathcal{M}}
\newcommand   \op          {\operatorname}

    3.D.9.
    Suppose $V$ is finite-dimensional and $T: V \rightarrow W$ is a surjective linear map of $V$ onto $W$.
    Prove that there is a subspace $U$ of $V$ such that $T\vert_U$ is an isomorphism of $U$ onto $W$.
    \\

    Suppose $v_1,\dots,v_m$ is a basis of the null space of $T$.
    Extend this list to a basis of $V$ by appending $u_1,\dots,u_n$.

    $Tu_1,\dots,Tu_n$ is a basis of $W$.
    We can see this from the fact that it is a spanning list of the right length.

    To see that it is the right length, consider the fundamental theorem of linear maps:
\begin{align*}
        \op{dim}\op{range}T &= \op{dim}V - \op{dim}\op{null}T \\
                            &= (m + n) - m \\
                            &= n .
\end{align*}
    Since $T$ is surjective, $\op{range}T = W$.
    Thus $\op{dim}W = n$.

    To see that $Tu_1,\dots,Tu_n$ spans $W$, suppose $w \in W$.
    Then because $T$ is surjective, there exists $v \in V$ and $a_1,\dots,a_m,b_1,\dots,b_n \in \F$ such that
\begin{align*}
        w   &= Tv \\
            &= T(a_1v_1 + \dots + a_mv_m + b_1u_1 + \dots + b_nu_n) \\
            &= a_1Tv_1 + \dots + a_mTv_m + b_1Tu_1 + \dots + b_nTu_n \\
            &= b_1Tu_1 + \dots + b_nTu_n .
\end{align*}
    Thus $w$ can be written as a linear combination of $Tu_1,\dots,Tu_n$.
    Thus $Tu_1,\dots,Tu_n$ is a basis of $W$ because it is a spanning list of the right length.

    Define $U$ as the span of $u_1,\dots,u_n$.

    $T\vert_U$ is surjective.
    To see this, add another line to the equation above:
\begin{align*}
        w   &= T(b_1u_1 + \dots + b_nu_n) .
\end{align*}
    Thus any vector in $W$ can be written as $T$ applied to a vector in $U$.
    Hence restricting $T$'s domain to $U$ does not decrease its range.

    Thus $T\vert_U$ is an isomorphism because it is a surjective linear map whose domain and target are finite-dimensional vector spaces with the same dimension:
\begin{align*}
        \op{dim}U = \op{dim}W = n .
\end{align*}
\end{document}
