\documentclass[a5paper]{article}
\usepackage{amsmath}
\usepackage{amssymb}
\usepackage[top=1cm,right=1cm,bottom=2cm,left=1cm]{geometry}
\setlength\parindent{0pt}
\setlength\parskip{1em}
\begin{document}
\newcommand   \C           {\mathbf{C}}
\newcommand   \R           {\mathbf{R}}
\renewcommand \L           {\mathcal{L}}
\newcommand   \F           {\mathbf{F}}
\renewcommand \P           {\mathcal{P}}
\newcommand   \M           {\mathcal{M}}
\newcommand   \E           {\mathcal{E}}
\newcommand   \op          {\operatorname}
\newcommand   \A           {\mathcal{A}}
\newcommand   \Q           {\mathcal{Q}}

    3.D.23.
    Suppose that $u_1,\dots,u_n$ and $v_1,\dots,v_n$ are bases of $V$.
    Let $T \in \L(V)$ be such that $Tv_k=u_k$ for each $k=1,\dots,n$.
    Prove that
\begin{align*}
        \M\big( T, (v_1,\dots,v_n) \big) = \M \big( (u_1,\dots,u_n), (v_1,\dots,v_n) \big) .
\end{align*}
    Let $A=\M\big( T, (v_1,\dots,v_n) \big)$.

    For each $k=1,\dots,n$, the $k$th column of $A$ implies that
\begin{align*}
        Tv_k &= A_{1,k}v_1 + \dots + A_{n,k}v_n .
\end{align*}
    Since $Tv_k=u_k$, we can replace the left side of the equation with $I(u_k)$.
    Then the equation shows that $A$ is the matrix of the identity operator with respect to two different bases:
\begin{align*}
        A = \M\big( I, (u_1,\dots,u_n), (v_1,\dots,v_n) \big)
\end{align*}
    as desired.
\end{document}
