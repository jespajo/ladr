\documentclass[a5paper]{article}
\usepackage{amsmath}
\usepackage{amssymb}
\usepackage[top=1cm,right=1cm,bottom=2cm,left=1cm]{geometry}
\setlength\parindent{0pt}
\setlength\parskip{1em}
%
\begin{document}
\newcommand   \C           {\mathbf{C}}
\newcommand   \R           {\mathbf{R}}
\renewcommand \L           {\mathcal{L}}
\newcommand   \F           {\mathbf{F}}
\renewcommand \P           {\mathcal{P}}
\newcommand   \M           {\mathcal{M}}
\newcommand   \op          {\operatorname}

    3.D.2.
    Suppose $T \in \L(U,V)$ and $S \in \L(V,W)$ are both invertible linear maps.
    Prove that $ST \in \L(U,W)$ is invertible and that $(ST)^{-1} = T^{-1} S^{-1}$.

    Since $T$ is invertible, $T^{-1} \in \L(V,U)$ exists.
    Since $S$ is invertible, $S^{-1} \in \L(W,V)$ exists.
    Thus $T^{-1}S^{-1} \in \L(W,U)$ makes sense because $S^{-1}$ maps to the domain of $T^{-1}$.

    For $w \in W$,
\begin{align*}
        (ST) (T^{-1}S^{-1}) w &= S (T T^{-1}) S^{-1} w \\
                                &= S I S^{-1} w \\
                                &= S S^{-1} w \\
                                &= I w \\
                                &= w .
\end{align*}
    For $u \in U$,
\begin{align*}
        (T^{-1}S^{-1}) (ST) u &= T^{-1} (S S^{-1}) T u \\
                              &= T^{-1} I T u \\
                              &= T^{-1} T u \\
                              &= I u \\
                              &= u .
\end{align*}
\end{document}
