\documentclass[a5paper]{article}
\usepackage{amsmath}
\usepackage{amssymb}
\usepackage[top=1cm,right=1cm,bottom=2cm,left=1cm]{geometry}
\setlength\parindent{0pt}
\setlength\parskip{1em}
%
\begin{document}
\newcommand   \C           {\mathbf{C}}
\newcommand   \R           {\mathbf{R}}
\renewcommand \L           {\mathcal{L}}
\newcommand   \F           {\mathbf{F}}
\renewcommand \P           {\mathcal{P}}
\newcommand   \M           {\mathcal{M}}
\newcommand   \op          {\operatorname}

    3.D.7.
    Suppose that $V$ is finite-dimensional and $S,T \in \L(V,W)$.
    Prove that $\op{range}S = \op{range}T$ if and only if there exists an invertible $E \in \L(V)$ such that $S = TE$.

    To prove in one direction, suppose there exists an invertible $E \in \L(V)$ such that $S = TE$.
    We must prove that $\op{range}S = \op{range}T$.

    To see that $\op{range}S \subseteq \op{range}T$, suppose $w \in \op{range}S$.
    Then there exists $v \in V$ such that $Sv = w$.
    Thus $TEv = w$.
    So $w \in \op{range}T$.
    Thus $\op{range}S \subseteq \op{range}T$.

    To see that $\op{range}T \subseteq \op{range}S$, suppose $w \in \op{range}T$.
    Then there exists $v \in V$ such that $Tv = w$.
    Since $E$ is invertible and $V$ is finite-dimensional, $E$ is surjective.
    So there exists $u \in V$ such that $Eu = v$.
    Thus
\begin{align*}
        w = Tv = TEu = Su .
\end{align*}
    So $w \in \op{range}S$.
    Thus $\op{range}T \subseteq \op{range}S$.

    Hence $\op{range}S = \op{range}T$.



    To prove in the other direction, suppose $\op{range}S = \op{range}T$.
    We must prove there exists an invertible $E \in \L(V)$ such that $S = TE$.

    Suppose $u_1,\dots,u_m$ is a basis of $\op{null}S$.
    Such a basis exists because $\op{null}S$ is a subspace of $V$, a finite-dimensional vector space.

    Extend this list to a basis of $V$ by appending $v_1,\dots,v_n \in V$.

    $Sv_1,\dots,Sv_n$ is a linearly independent list of vectors in $\op{range}S$.
    To see this, suppose $a_1,\dots,a_n \in \F$ such that $a_1Sv_1+\dots+a_nSv_n=0$.
    Then
\begin{align*}
        0 &= S(a_1v_1 + \dots + a_nv_n)
\end{align*}
    so $a_1v_1+\dots+a_nv_n$ is a vector in $\op{null}S$.
    So there exist $b_1,\dots,b_m \in \F$ such that
\begin{align*}
        b_1u_1+\dots+b_mu_m &= a_1v_1+\dots+a_nv_n .
\end{align*}
    Moving the right side over, we have
\begin{align*}
        0 &= a_1v_1+\dots+a_nv_n - b_1u_1-\dots-b_mu_m .
\end{align*}
    This is a linear combination of a linearly independent list, so $a_1=\dots=a_n=b_1=\dots=b_n=0$.
    Thus $Sv_1,\dots,Sv_n$ is linearly independent.

    From the fundamental theorem of linear maps, we have
\begin{align*}
        \op{dim}V &= \op{dim}\op{null}S + \op{dim}\op{range}S \\
            m + n &= m + \op{dim}\op{range}S \\
                n &= \op{dim}\op{range}S .
\end{align*}
    Hence $Sv_1,\dots,Sv_n$ is a basis of $\op{range}S$ because it is a linearly independent list of the right length.

    Suppose $w_1,\dots,w_m$ is a basis of $\op{null}T$.
    The fundamental theorem of linear maps tells us that $m$ is the correct number of elements:
\begin{align*}
        \op{dim}\op{null}T = \op{dim}V - \op{dim}\op{range}T = (m+n) - n &= m .
\end{align*}
    Next, for each $k=1,\dots,n$, since $Sv_k \in \op{range}S$ and $\op{range}T = \op{range}S$, $Sv_k \in \op{range}T$.
    Hence there exists $x_k \in V$ such that
\begin{align*}
        Sv_k = Tx_k .
\end{align*}
    We will show that $w_1,\dots,w_m,x_1,\dots,x_n$ is a basis of $V$ by showing that it is a linearly independent list of the right length.
    Suppose $c_1,\dots,c_m,d_1,\dots,d_n \in \F$ such that
\begin{align*}
        0 &= c_1w_1 + \dots + c_mw_m + d_1x_1 + \dots d_nx_n .
\end{align*}
    Applying $T$ to both sides,
\begin{align*}
         0 &= T(c_1w_1 + \dots + c_mw_m + d_1x_1 + \dots d_nx_n) \\
           &= c_1Tw_1 + \dots + c_mTw_m + d_1Tx_1 + \dots d_nTx_n \\
           &= d_1Tx_1 + \dots d_nTx_n \\
           &= d_1Sv_1 + \dots d_nSv_n .
\end{align*}
    The $c_jTw_j$ terms disappeared because each $w_j \in \op{null}T$.

    Thus $d_1=\dots=d_n=0$ because we have a linear combination of a linearly independent list of vectors equal to zero.
    Hence our equation becomes
\begin{align*}
        0 &= c_1w_1 + \dots + c_mw_m
\end{align*}
    so $c_1=\dots=c_m=0$ for the same reason.
    Thus $w_1,\dots,w_m,x_1,\dots,x_n$ is linearly independent, as desired, which means it is a basis of $V$.

    To recap, our steps so far have established two bases of $V$,
\begin{align*}
        u_1,\dots,u_m,v_1,\dots,v_n \quad\text{and}\quad w_1,\dots,w_m,x_1,\dots,x_n,
\end{align*}
    where $u_1,\dots,u_m$ is a basis of $\op{null}S$, $w_1,\dots,w_m$ is a basis of $\op{null}T$ and $Sv_1,\dots,Sv_n$, which is the same as $Tx_1,\dots,Tx_n$, is a basis of $\op{range}S$ and $\op{range}T$.

    Now define $E \in \L(V)$ as a linear map that takes each vector in the first basis to the corresponding vector in the second basis.
    In other words, for $j=1,\dots,m$ and $k=1,\dots,n$,
\begin{align*}
        Eu_j = w_j \quad\text{and}\quad Ev_k = x_k .
\end{align*}
    This is a valid definition of a linear map because it defines $E$ completely on a basis of $V$ ($u_1,\dots,u_m,v_1,\dots,v_m$).
    So to complete our proof, we need to show that $S=TE$ and that $E$ is invertible.

    Suppose $v \in V$.
    Then there exist $p_1,\dots,p_m,q_1,\dots,q_n \in \F$ such that
\begin{align*}
        v &= p_1u_1 + \dots + p_mu_m + q_1v_1 + \dots + q_nv_n .
\end{align*}
    Applying $S$ to $v$,
\begin{align*}
        Sv &= S(p_1u_1 + \dots + p_mu_m + q_1v_1 + \dots + q_nv_n) \\
           &= p_1Su_1 + \dots + p_mSu_m + q_1Sv_1 + \dots + q_nSv_n \\
           &= q_1Sv_1 + \dots + q_nSv_n .
\end{align*}
    Applying $TE$ to $v$,
\begin{align*}
        TEv &= T(p_1Eu_1 + \dots + p_mEu_m + q_1Ev_1 + \dots + q_nEv_n) \\
            &= T(p_1w_1 + \dots + p_mw_m + q_1x_1 + \dots + q_nx_n) \\
            &= p_1Tw_1 + \dots + p_mTw_m + q_1Tx_1 + \dots + q_nTx_n \\
            &= q_1Tx_1 + \dots + q_nTx_n \\
            &= q_1Sv_1 + \dots + q_nSv_n \\
            &= Sv .
\end{align*}
    Thus $S = TE$.

    Finally, it is easy to see that $E$ is invertible because $E$ maps each vector in one basis of $V$ ($u_1,\dots,u_m,v_1,\dots,v_n$) to a vector in another basis of $V$ ($w_1,\dots,w_m,x_1,\dots,x_n$).
    Any $w \in V$ is expressible as a linear combination of the second basis, so we could take the coefficients from this linear combination and use them in a linear combination of the first basis to find a vector that $E$ maps to $w$.
    Thus $E$ is surjective.
    Hence $E$ is invertible because it is a surjective operator on a finite-dimensional vector space.
\end{document}
