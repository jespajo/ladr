\documentclass[a5paper]{article}
\usepackage{amsmath}
\usepackage{amssymb}
\usepackage[top=1cm,right=1cm,bottom=2cm,left=1cm]{geometry}
\setlength\parindent{0pt}
\setlength\parskip{1em}
%
\begin{document}
\newcommand   \C           {\mathbf{C}}
\newcommand   \R           {\mathbf{R}}
\renewcommand \L           {\mathcal{L}}
\newcommand   \F           {\mathbf{F}}
\renewcommand \P           {\mathcal{P}}
\newcommand   \M           {\mathcal{M}}
\newcommand   \op          {\operatorname}

    3.D.5.
    Suppose $V$ is finite-dimensional, $U$ is a subspace of $V$, and $S \in \L(U,V)$.
    Prove that there exists an invertible linear map $T$ from $V$ to itself such that $Tu = Su$ for every $u \in U$ if and only if $S$ is injective.

    To prove in one direction, suppose there exists an invertible linear map $T$ from $V$ to itself such that $Tu = Su$ for every $u \in U$.

    First, $T$ is injective because it is an invertible linear map whose domain and range are vector spaces of the same dimension.

    Now suppose $Su = 0$.
    Then $Tu = 0$.
    Then $u = 0$ because $T$ is injective.
    Thus $\op{null}S = \{0\}$.
    Hence $S$ is injective.


    To prove in the other direction, suppose $S$ is injective.

    Suppose $u_1,\dots,u_n$ is a basis of $U$.
    Suppose we extend this list to a basis of $V$ by appending $v_1,\dots,v_m \in V$.
    We have created our first basis of $V$:
\begin{align*}
        u_1,\dots,u_n, v_1,\dots,v_m .
\end{align*}

    Because $u_1,\dots,u_n$ is a linearly independent list of vectors in $U$, and $S$ is injective, $Su_1,\dots,Su_n$ is a linearly independent list of vectors in $V$ (a result that we showed in exercise 3.B.9).
    Suppose we extend $Su_1,\dots,Su_n$ to a basis of $V$ by appending $w_1,\dots,w_m \in V$.
    We have created our second basis of $V$:
\begin{align*}
        Su_1,\dots,Su_n, w_1,\dots,w_m .
\end{align*}

    Define $T \in \L(V)$ as the linear map that takes each vector in the first basis to the corresponding vector in the second basis.
    The validity of this definition is shown by the linear map lemma.

    For $u \in U$, there exist $a_1,\dots,a_n \in \F$ such that
\begin{align*}
        u = a_1u_1+\dots+a_nu_n .
\end{align*}
    Thus
\begin{align*}
        Tu &= T(a_1u_1+\dots+a_nu_n) \\
           &= a_1Tu_1+\dots+a_nTu_n \\
           &= a_1Su_1+\dots+a_nSu_n \\
           &= S(a_1u_1+\dots+a_nu_n) \\
           &= Su .
\end{align*}
    Thus $Tu = Su$ for $u \in U$.

    To complete our proof, we need to show that $T$ is invertible.
    Since $T$ is a linear map whose domain and range are vector spaces of the same dimension, we can show that $T$ is invertible by showing that it is surjective.

    To see that $T$ is surjective, suppose $w \in V$.
    Then there exist $b_1,\dots,b_n,c_1,\dots,c_m \in \F$ such that
\begin{align*}
        w = b_1 Su_1 + \dots + b_n Su_n + c_1 w_1 + \dots + c_m w_m .
\end{align*}
    There also exists $v \in V$ such that
\begin{align*}
        v = b_1 u_1 + \dots + b_n u_n + c_1 v_1 + \dots + c_m v_m .
\end{align*}
    Applying $T$ to $v$,
\begin{align*}
        Tv &= T (b_1u_1 + \dots + b_nu_n + c_1v_1 + \dots + c_mv_m) \\
           &= b_1 Tu_1 + \dots + b_n Tu_n + c_1 Tv_1 + \dots + c_m Tv_m \\
           &= b_1 Su_1 + \dots + b_n Su_n + c_1 w_1 + \dots + c_m w_m \\
           &= w .
\end{align*}
    Thus $w \in \op{range}T$.
    Hence $T$ is surjective, completing our proof.
\end{document}
