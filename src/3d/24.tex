\documentclass[a5paper]{article}
\usepackage{amsmath}
\usepackage{amssymb}
\usepackage[top=1cm,right=1cm,bottom=2cm,left=1cm]{geometry}
\setlength\parindent{0pt}
\setlength\parskip{1em}
\begin{document}
\newcommand   \C           {\mathbf{C}}
\newcommand   \R           {\mathbf{R}}
\renewcommand \L           {\mathcal{L}}
\newcommand   \F           {\mathbf{F}}
\renewcommand \P           {\mathcal{P}}
\newcommand   \M           {\mathcal{M}}
\newcommand   \E           {\mathcal{E}}
\newcommand   \op          {\operatorname}
\newcommand   \A           {\mathcal{A}}
\newcommand   \Q           {\mathcal{Q}}

    3.D.24.
    Suppose $A$ and $B$ are square matrices of the same size and $AB=I$.
    Prove that $BA=I$.

    Let $n$ denote the number of columns in $A$ and $B$.
    Note that all instances of $\M$ in this exercise define a matrix with respect to the standard basis of $\F^n$.

    Suppose $S,T \in \L(\F^n)$ such that $\M(S) = A$ and $\M(T) = B$.
    The existence of $S$ and $T$ follows from the surjectivity of $\M$ from $\L(\F^n)$ onto $\F^{n,n}$ (see 3.71).
    Then
\begin{align*}
        \M(I) = AB = \M(S)\M(T) = \M(ST) .
\end{align*}
    Since $\M$ is injective, this implies $ST=I$.

    Hence $TS=I$ because the domain and target vector spaces of $S$ and $T$ have the same dimension (using 3.68).
    Thus
\begin{align*}
        \M(I) = \M(TS) = \M(T)\M(S) = B A .
\end{align*}
\end{document}
