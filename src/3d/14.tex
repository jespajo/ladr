\documentclass[a5paper]{article}
\usepackage{amsmath}
\usepackage{amssymb}
\usepackage[top=1cm,right=1cm,bottom=2cm,left=1cm]{geometry}
\setlength\parindent{0pt}
\setlength\parskip{1em}
%
\begin{document}
\newcommand   \C           {\mathbf{C}}
\newcommand   \R           {\mathbf{R}}
\renewcommand \L           {\mathcal{L}}
\newcommand   \F           {\mathbf{F}}
\renewcommand \P           {\mathcal{P}}
\newcommand   \M           {\mathcal{M}}
\newcommand   \E           {\mathcal{E}}
\newcommand   \op          {\operatorname}

    3.D.14.
    Prove or give a counterexample:
    If $V$ is a finite-dimensional vector space and $R,S,T \in \L(V)$ are such that $RST$ is surjective, then $S$ is injective.

    Since $RST$ is surjective, $\op{dim}\op{range}RST = \op{dim}V$.

    We can use the result of exercise 3.B.23 to see that
\begin{align*}
        \op{dim}V = \op{dim}\op{range}RST \le \op{min}\{ \op{dim}\op{range}RS, \op{dim}\op{range}T \} .
\end{align*}
    Thus
\begin{align*}
        \op{dim}V \le \op{dim}\op{range}RS \quad\text{and}\quad \op{dim}V \le \op{dim}\op{range}T .
\end{align*}
    Since $\op{range}RS$ and $\op{range}T$ are subspaces of $V$, we also have
\begin{align*}
        \op{dim}V \ge \op{dim}\op{range}RS \quad\text{and}\quad \op{dim}V \ge \op{dim}\op{range}T .
\end{align*}
    This implies that
\begin{align*}
        \op{dim}V = \op{dim}\op{range}RS = \op{dim}\op{range}T .
\end{align*}
    Thus $RS$ is surjective.

    Applying the same logic recursively, we have
\begin{align*}
        \op{dim}V = \op{dim}\op{range}RS \le \op{min}\{ \op{dim}\op{range}R, \op{dim}\op{range}S \} .
\end{align*}
    Hence
\begin{align*}
        \op{dim}V = \op{dim}\op{range}R = \op{dim}\op{range}S = \op{dim}\op{range}T .
\end{align*}
    Thus $S$ is surjective.

    Since $S$ is an operator on a finite-dimensional vector space, this implies that $S$ is injective.
\end{document}
