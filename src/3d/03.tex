\documentclass[a5paper]{article}
\usepackage{amsmath}
\usepackage{amssymb}
\usepackage[top=1cm,right=1cm,bottom=2cm,left=1cm]{geometry}
\setlength\parindent{0pt}
\setlength\parskip{1em}
%
\begin{document}
\newcommand   \C           {\mathbf{C}}
\newcommand   \R           {\mathbf{R}}
\renewcommand \L           {\mathcal{L}}
\newcommand   \F           {\mathbf{F}}
\renewcommand \P           {\mathcal{P}}
\newcommand   \M           {\mathcal{M}}
\newcommand   \op          {\operatorname}

    3.D.3.
    Suppose $V$ is finite-dimensional and $T \in \L(V)$.
    Prove that the following are equivalent. \\
    (a) $T$ is invertible. \\
    (b) $Tv_1,\dots,Tv_n$ is a basis of $V$ for every basis $v_1,\dots,v_n$ of $V$. \\
    (c) $Tv_1,\dots,Tv_n$ is a basis of $V$ for some basis $v_1,\dots,v_n$ of $V$.

    To prove in one direction, suppose (a) is true, so $T$ is invertible.

    Because $T$ is invertible and $V$ is finite-dimensional, $T$ is injective (using 3.65).

    Now for any basis of $V$, denoted $v_1,\dots,v_n$, suppose $a_1,\dots,a_n \in \F$ such that
\begin{align*}
        a_1 Tv_1 + \dots + a_n Tv_n = 0 .
\end{align*}
    Then
\begin{align*}
        T (a_1 v_1 + \dots + a_n v_n) = 0 .
\end{align*}
    Then because $T$ is injective,
\begin{align*}
        a_1 v_1 + \dots + a_n v_n = 0 .
\end{align*}
    This implies $a_1=\dots=a_n=0$ because $v_1,\dots,v_n$ is a linearly independent list.

    Thus the only linear combination of $Tv_1,\dots,Tv_n$ equal to 0 has all coefficients of 0.
    Hence $Tv_1,\dots,Tv_n$ is a linearly independent list.
    Hence $Tv_1,\dots,Tv_n$ is a basis of $V$ because it is a linearly independent list of the right length.
    Thus we have proven (b), which implies (c) because (c) is a special case of (b).

    To prove in the other direction, suppose (c) is true, so $v_1,\dots,v_n$ is a basis of $V$ and $Tv_1,\dots,Tv_n$ is also a basis of $V$.

    Suppose $w \in V$.
    Then, because $Tv_1,\dots,Tv_n$ is a basis of $V$, there exist $b_1,\dots,b_n \in F$ such that
\begin{align*}
        w &= b_1 Tv_1 + \dots + b_n Tv_n \\
          &= T (b_1 v_1 + \dots + b_n v_n) .
\end{align*}
    So there exists a vector that $T$ takes to $w$.
    Thus $T$ is surjective.

    Using 3.65, this shows that $T$ is invertible because $T$ is surjective and $V$ is finite-dimensional.
    Thus (c) implies (a).
    Since we have already shown that (a) implies (b), this completes our proof in the other direction.
\end{document}
