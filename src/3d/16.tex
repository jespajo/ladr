\documentclass[a5paper]{article}
\usepackage{amsmath}
\usepackage{amssymb}
\usepackage[top=1cm,right=1cm,bottom=2cm,left=1cm]{geometry}
\setlength\parindent{0pt}
\setlength\parskip{1em}
%
\begin{document}
\newcommand   \C           {\mathbf{C}}
\newcommand   \R           {\mathbf{R}}
\renewcommand \L           {\mathcal{L}}
\newcommand   \F           {\mathbf{F}}
\renewcommand \P           {\mathcal{P}}
\newcommand   \M           {\mathcal{M}}
\newcommand   \E           {\mathcal{E}}
\newcommand   \op          {\operatorname}

    3.D.16.
    Prove that every linear map from $\F^{n,1}$ to $\F^{m,1}$ is given by a matrix multiplication.
    In other words, prove that if $T \in \L(\F^{n,1}, \F^{m,1})$, then there exists an $m$-by-$n$ matrix $A$ such that $Tx = Ax$ for every $x \in \F^{n,1}$.

    For compactness, let $N_1,\dots,N_n$ denote the column matrices in the standard basis of $\F^{n,1}$.
    So
\begin{align*}
        N_1 = \begin{pmatrix}
                1 \\
                0 \\
                \vdots \\
                0
            \end{pmatrix}, \quad
        N_2 = \begin{pmatrix}
                0 \\
                1 \\
                \vdots \\
                0
            \end{pmatrix}, \quad
        \dots, \quad
        N_n = \begin{pmatrix}
                0 \\
                0 \\
                \vdots \\
                1
            \end{pmatrix} .
\end{align*}
    As a linear map, $T$ is completely determined by its values on a basis of $\F^{n,1}$.
    These values are the list $TN_1, \dots, TN_n$.
    Each $TN_k$ is the $m$-by-$1$ matrix you get when you apply $T$ to the $k$th column matrix in the standard basis of $\F^{n,1}$.

    Let $A_{1,k},\dots,A_{m,k}$ denote the entries of $TN_k$.
    In other words, for $k=1,\dots,n$,
\begin{align*}
        T N_k &=    \begin{pmatrix}
                        A_{1,k} \\
                        A_{2,k} \\
                        \vdots \\
                        A_{m,k}
                    \end{pmatrix} .
\end{align*}
    Then for $x \in \F^{n,1}$ we have
\begin{align*}
    x =    \begin{pmatrix}
                x_1 \\
                x_2 \\
                \vdots \\
                x_n
            \end{pmatrix}
      = x_1N_1 + \dots + x_nN_n .
\end{align*}
    Applying $T$,
\begin{align*}
        Tx &= T(x_1N_1 + \dots + x_nN_n) \\
           &= x_1TN_1 + \dots + x_nTN_n .
\end{align*}
    Thus
\begin{align*}
        Tx &= x_1   \begin{pmatrix}
                        A_{1,1} \\
                        A_{2,1} \\
                        \vdots \\
                        A_{m,1}
                    \end{pmatrix}
            + \dots
            + x_n   \begin{pmatrix}
                        A_{1,n} \\
                        A_{2,n} \\
                        \vdots \\
                        A_{m,n}
                    \end{pmatrix} .
\end{align*}
    Thus $Tx$ is a linear combination of a list of $n$ vectors in $\F^{m,1}$, with the coefficients coming from $x$.

    Let $A$ be the matrix created by putting these vectors side by side as the columns in an $m$-by-$n$ matrix:
\begin{align*}
        A &=
            \begin{pmatrix}
                A_{1,1}\ \dots\ A_{1,n} \\
                A_{2,1}\ \dots\ A_{2,n} \\
                \vdots                    \\
                A_{m,1}\ \dots\ A_{m,n}
            \end{pmatrix} .
\end{align*}
    Using proof 3.50, $Ax$ is a linear combination of the columns of $A$, with the scalars that multiply the columns coming from $x$.

    Thus $Ax = Tx$ as desired.
\end{document}
