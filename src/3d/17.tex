\documentclass[a5paper]{article}
\usepackage{amsmath}
\usepackage{amssymb}
\usepackage[top=1cm,right=1cm,bottom=2cm,left=1cm]{geometry}
\setlength\parindent{0pt}
\setlength\parskip{1em}
%
\begin{document}
\newcommand   \C           {\mathbf{C}}
\newcommand   \R           {\mathbf{R}}
\renewcommand \L           {\mathcal{L}}
\newcommand   \F           {\mathbf{F}}
\renewcommand \P           {\mathcal{P}}
\newcommand   \M           {\mathcal{M}}
\newcommand   \E           {\mathcal{E}}
\newcommand   \op          {\operatorname}
\newcommand   \A           {\mathcal{A}}
\newcommand   \B           {\mathcal{B}}

    3.D.17.
    Suppose $V$ is finite-dimensional and $S \in \L(V)$.
    Define $\A \in \L\big(\L(V)\big)$ by
\begin{align*}
        \A(T) &= ST
\end{align*}
    for $T \in \L(V)$.

    (a) Show that $\op{dim}\op{null}\A = (\op{dim}V)(\op{dim}\op{null}S)$. \\
    (b) Show that $\op{dim}\op{range}\A = (\op{dim}V)(\op{dim}\op{range}S)$.

\hfill

    Suppose $v_1,\dots,v_m$ is a basis of the null space of $S$.
    Extend this to a basis of $V$ by appending some $v_{m+1},\dots,v_n \in V$.

    Thus $\op{dim}\op{null}S = m$ and $\op{dim}V = n$ with $m \le n$.

    Consider the standard basis of $\F^{n,n}$.
    This is the list of distinct $n$-by-$n$ matrices having $0$ in every entry except for a $1$ in one entry.
    Reduce this list by removing all matrices that contain a $1$ in any rows lower than the $m$th row.

    Let $\B$ denote the span of this reduced list of matrices.
    The list contains one matrix for every entry in the top $m$ rows of an $n$-by-$n$ matrix.
    The list is also linearly independent.
    Thus $\B$ is a subspace of $\L(V)$ with $\op{dim}\B = mn$.

    $\M\vert_{\op{null}\A}$ is an isomorphism from the null space of $\A$ onto $\B$.

    To see that $\M\vert_{\op{null}\A}$ is surjective onto $\B$, suppose $B \in \B$.
    So $B$ is an $n$-by-$n$ matrix containing only zeroes after the $m$th row.

    There exists $T \in \L(V)$ such that
\begin{align*}
        Tv_k &= B_{1,k}v_1 + \dots + B_{m,k}v_m
\end{align*}
    for $k=1,\dots,n$.

    Notice that each $Tv_k$ is a linear combination of only the first $m$ vectors in the basis of $V$.
    Applying $S$,
\begin{align*}
        S(Tv_k) &= B_{1,k}Sv_1 + \dots + B_{m,k}Sv_m \\
                &= 0 .
\end{align*}
    Thus $\op{range}T \subseteq \op{null}S$.
    Hence $T \in \op{null}\A$.

    $\M(T) = B$.
    Thus $\M\vert_{\op{null}\A}$ is surjective onto $\B$.

    The isomorphism implies that the null space of $\A$ has the same dimension as $\B$.
    Thus
\begin{align*}
        \op{dim}\op{null}\A &= m n \\
                            &= (\op{dim}\op{null}S)(\op{dim}V) ,
\end{align*}
    proving (a).

    Using the fundamental theorem of linear maps,
\begin{align*}
        \op{dim}\op{range}S &= \op{dim}V - \op{dim}\op{null}S \\
                            &= n - m .
\end{align*}
    Using the fundamental theorem again,
\begin{align*}
        \op{dim}\op{range}\A &= \op{dim}\L(V) - \op{dim}\op{null}\A \\
                             &= nn - mn \\
                             &= n (n - m) \\
                             &= (\op{dim}V) (\op{dim}\op{range}S) ,
\end{align*}
    proving (b).
\end{document}
